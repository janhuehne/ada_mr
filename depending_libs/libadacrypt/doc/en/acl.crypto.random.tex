\chapter{Crypto.Random}
This package uses ``dev/random/'' as default.
On Linux systems, ``/dev/random'' is a pseudo-device capable of producing
cryptographically secure pseudo-random bits. If you should be using an 
operatingsystem without such a pseudo-device, you can use Brian Warner's
``Entropy Gathering Daemon''(EGD) that can be found at
\textit{http://www.lothar.com/tech/crypto/}.
The EGD is a userspace implementation of the Linux kernel driver
 ``dev/random/'', that can be used for generating cryptographically secure 
pseudo-random bits.\\
The private part includes the ``Path'' variable. These variable contains the
path to a file, a named pipe or a device storing cryptographically secure 
random bits. As a result, you can even link the ACL to a cryptographically 
secure pseudo-random bit source on non-POSIX compliant operating systems such 
as Windows XP.

\section{API}
\subsection{Prozedures}
\begin{tabular}{p{\textwidth}}
  \begin{lstlisting}{}
    procedure Read(B : out Byte);
    procedure Read(W : out Word);
    procedure Read(D : out DWord);
  \end{lstlisting}
This three procedures return a random Byte, Word or DWord. \\ \ \\
  \hline\\
\end{tabular}

\begin{tabular}{p{\textwidth}}
  \begin{lstlisting}{}
    procedure Read(Byte_Array  : out Bytes);
    procedure Read(Word_Array  : out Words);
    procedure Read(DWord_Array : out DWords);
  \end{lstlisting}\\
Any of thes  prozedures fill an array with random numbers from the pseudo 
random number generator (PNRG).\\ \ \\
\end{tabular}


\subsection{Exceptions}
\begin{tabular}{p{\textwidth}}
  \begin{lstlisting}{}
    Random_Source_Does_Not_Exist_Error : exception;
  \end{lstlisting}\\
  This exception occur when no secure PRNG were  found.\\ \ \\
  \hline\\
  \begin{lstlisting}{}
    Random_Source_Read_Error : exception;
  \end{lstlisting}\\
  This exception occur when he ACL can't read random bits from  the PRNG.\\ 
  \ \\
  \hline\\
\end{tabular}

\begin{lstlisting}{}
\end{lstlisting}

