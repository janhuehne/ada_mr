\chapter{Introduction}
The Ada Crypto Library (ACL) is a free cryptographic library for Ada.
One of the two main design objectives of this library was to create
an \textbf{intuitive and clean API}. The other goal was to use a clean
programming style in order to actively support and simplify formal code
verification. Due to this, the ACL has been coded without the
following 'features':
\begin{itemize}
\item access types (pointers)
\item inline assembler
\item tagged types (objects)
\item goto statements
\end{itemize}

Goto statements and access types have been avoided as they make the
source more complex and might cause problems during formal verification
of the code. No tagged types have been used as procedures and methods may
be overwritten and the actual method that is being utilised is determined
during runtime (dynamic dispatching). This results in massive problems during
formal code verification. Inline assembler has been avoided cause it involves
a very unclean programming style and can be used for omitting the strict typing
of ADA. Due to this restrictions, the actual code quality increases with
respect to verification and security, however they also lead to decreased
performance. 
If you favor performance over clean design, you should possibly look for
anothercryptographic library. Unfortunately, there are currently no other free 
cryptographic libraries for ADA I can refer to.\\
The ACL has the status of a draft. Neither a formal code review nor a code 
audit have been performed. At the time of writing, the author of the ACL is the
only one who has dealt with the implementation's details.
For instance, there might be the possibility that sensible material, such as 
the key, might still be resident within the RAM after a program using the ACL 
has been terminated. A storage pool dealing with this restriction is currently 
being developed. Most of the other cryptographic libraries that share this 
limitation do not inform about the possible weakness. Those libraries should 
not be used as they can not be considered reliable.\\
The following paragraph subsumes the disadvantages of the ACL I know:
\begin{itemize}
\item missing cleanup of stack and heap
\item no big endian support
\item bad performance
\end{itemize}
This documentation subsumes the installation and the topological structure.
Afterwards, it solely describes the API. Each package and it's API are 
introduced within a separate chapter. Each chapter concludes with an example.\\
If you should have any questions about the ACL, encounter a bug or want to 
contribute one or several packages, feel free to contact me via email at: 
crypto@shortie.inka.de.\\

\subsection{TODO List}
\begin{itemize}
\item optimisation
\item own storage pool
\item extension (Tiger2, RSA-PSS, Poly1305-AES etc.)  
\end{itemize}

%%%%%%%%%%%%%%%%%%%%%%%%%%%%%%%%%%%%%%%%%%%%%%%%%%%%%%%%%%%%%%%%%%%%%%%%%%%

\section{Installation HOWTO}
\subsection{libadacrpyt-dev}
On a Linux system, the following packages are required for installing the ACL:
\begin{itemize}
\item tar 
\item gnat 
\item make
\item bunzip2 
\item binutils
\end{itemize}

\subsubsection{Compilation}
The following commands can be used for decompressing and compiling the ACL as 
well as the regression test included.
\begin{itemize}
\item \texttt{tar cjf acl.tar.bz2}
\item \texttt{cd crypto}
\item \texttt{make}
\item \texttt{make acltest}
\end{itemize}

\subsubsection{Testing}
Before you actually install the ACL you should run the regression test in
order to make sure the ACL properly works on your computer system. The
regression test approximately takes 30 seconds on a PII 450. The following
command sequence can be used for running the regression test.\\

\begin{itemize}
\item \texttt{cd test}
\item \texttt{./acltest}
\item \texttt{cd ..}
\end{itemize}

\subsubsection{Installation}
If no errors were encountered during the regression test, the ACL can
be installed by issuing the following command:\\
\hspace*{1cm} \texttt{su -c ``make install''}\ \\

\subsubsection{De-installation}
The following command can be used for removing the ACL: \\
\hspace*{1cm} \texttt{su -c ``make uninstall''}\ \\

\subsubsection{Recompilation}
The following commands can be used for recompiling the ACL:\\
\begin{itemize}
\item \texttt{make clean acltest\_clean}
\item \texttt{make}
\item \texttt{make acltest}
\end{itemize}


\subsection{libadacrypt}
\subsubsection{Installiation}
\subsubsection{Shared Library}
The ACL may be installed as a shared library (libacl.so) with the following 
commands:
\begin{itemize}
\item \texttt{make shared}
\item \texttt{make install-shared}
\end{itemize}

\subsubsection{De-installation}
The shared library can be de-installed with the following command:\\
\hspace*{1cm} \texttt{su -c ``make uninstall-shared''}\ \\


\subsection{Documentation}
The tetex-bin (latex) and the tetex-extra packages are needed to install this 
documentation The complete ACL documentation (german \& english) can be 
generated with  the following command:\\
\hspace*{1cm} \texttt{make doc}\ \\ 
\ \\
The documentation will be copied to  
\texttt{/usr/local/share/doc/libadacrypt-dev} with the following command:
\hspace*{1cm} \texttt{su -c ``make install-doc''}\\ 
\ \\
The documentation in \texttt{/usr/local/share/doc/libadacrypt-dev} can be
uninstalled with the following command:
\hspace*{1cm} \texttt{su -c ``make uninstall-doc''}\\ 
\ \\
The documentation can be departed with the following command:
\hspace*{1cm} \texttt{su -c ``make uninstall-doc''}\\



\subsubsection{Adaptations}
The Makefiles can be found in the \texttt{src} directory. It can be used for
adapting the following variables:
\begin{itemize}
\item LIBDIR : installation path for the shared library.
\item INSTDIR : installation path for the ACL.
\end{itemize}
Besides the installation path for the documentation also be changed. Thereto
the variable \texttt{DOCPATH} in \texttt{doc/en/Makefile} and
\texttt{doc/de/Makefile} must be changed.


%%%%%%%%%%%%%%%%%%%%%%%%%%%%%%%%%%%%%%%%%%%%%%%%%%%%%%%%%%%%%%%%%%%%%%%%%%%

\section{Layout of Directories and Packages}
\subsection{Directories}
\begin{itemize}
\item doc : documentation
\item ref : references and specifications
\item src : source code
\item test : regression test
\end{itemize}

%%%%%%%%%%%%%%%%%%%%%%%%%%%%%%%%%%%%%%%%%%%%%%%%%%%%%%%%%%%%%%%%%%%%%%%%%%%

\subsection{Package Layout}
The Ada Crypto Library (ACL) subsumes the following packages (components).
\begin{description}
\item[Crypto:]\ \\ 
  This is the root package of the ACL.
  All other ACL packages start with the \textit{Crypto} prefix.
\item[Crypto.Types:] \ \\
  The fundamental types of the ACL (such as Byte) and their corresponding basic
  functionality are located within this package.
  Utilisation of the ACL is very limited without including this package.  
\item[Crypto.Random:] \ \\
  This package implements an interface between an external random bit generator
  an the ACL.
\item[Crypto.Symmetric:]\ \\
  This is the root package for the symmetric branch.
\item[Crypto.Symmetric.Algorithm:]\ \\
  This branch of the ACL tree contains the symmetric algorithms for the
  blockciphers and the hashfunctions.\\
\item[Crypto.Symmetric.Algorithm.Oneway:]\ \\
  Each algorithm has an oneway algorithm. Symmetric oneway algorithms are
  symmetric algorithms that only work in one direction, and can
  either be used for en- or decryption of data.
\item[Crypto.Symmetric.Blockcipher:]\ \\
  This generic package can be used for generating a block cipher out of
  a symmetric algorithm.
\item[Crypto.Symmetric.Oneway\_Blockcipher:]\ \\
  This package can be used for generating a oneway block cipher out of a
  symmetric oneway algorithm.
\item[Crypto.Symmetric.MAC:]
  This is the root package of the mac branch.
\item[Crypto.Symmetric.Mode:]\ \\
  This branch subsumes multiple modes of operation for block ciphers.
\item[Crypto.Symmetric.Mode.Oneway:]\ \\
  This branch subsumes multiple modes of operation for oneway block ciphers.
\item[Crypto.Asymmetric:]\ \\
  This is the root package of the asymmetric branch.
\item[Crypto.Asymmetric.DSA:]\ \\
  This package can be used for generating and verifying digital signatures.
\item[Crypto.Asymmetric.RSA:]\ \\
  This package can be used for asymmetric en- and decryption of data.
\end{description}


\begin{figure}
  \scalebox{0.90}{%
  \pstree[nodesep=2pt]{\TR{Crypto}}{%
    \pstree{\TR{Types}}{%
      \TR{Big\_Numbers}}
    \TR{Hashfunction}
    \pstree{\TR{Symmetric}}{%
      \pstree{\TR{Algorithm}}{%
	\Tr{Oneway}}
      \TR{Blockcipher} 
      \TR{MAC}
      \TR{Oneway\_Blockcipher}
      \pstree{\TR{Mode}}{%
	\TR{Oneway}}} % end symmetric
    \TR{Random} 
    \pstree{\TR{Asymmetric}}{%
      \TR{DSA} \TR{RSA}}
}}
\end{figure}
    
