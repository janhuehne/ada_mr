\chapter{Crypto.Certificate}
\label{sec:CryptoCertificate}
Mit Hilfe eines asymmetrischen Kryptosystems k�nnen Nachrichten in einem Netzwerk digital signiert und verschl�sselt werden. F�r jede verschl�sselte �bermittlung ben�tigt der Sender allerdings den �ffentlichen Schl�ssel (Public-Key) des Empf�ngers. Um zu �berpr�fen, ob es sich tats�chlich um den Schl�ssel des Empf�ngers handelt und nicht um eine F�lschung, benutzt man digitale Zertifikate, welche die Authentizit�t eines �ffentlichen Schl�ssels und seinen zul�ssigen Anwendungs- und Geltungsbereich best�tigen. X.509 ist derzeit der wichtigste Standard f�r digitale Zertifikate. Die aktuelle Version ist X.509v3.


\section{X.509}
Durch den Einsatz des Crypto.Certificate werden Zertifikate verifiziert und im Falle einer erfolgeichen Pr�fung k�nnen sie f�r ihre Zwecke weiter benutzt werden. Andernfalls gilt das Zertifikat als nicht vertrauensw�rdig und wird nicht weiter benutzt.\\ Unter solchen umst�nden sollten in Zukunft die Kinderzertifikate, also die Zertifikate die von einem hierarchisch h�heren Signiert wurden, auch nicht mehr vertraut werden.


\section{API}
\begin{tabular}{p{\textwidth}}
\begin{lstlisting}{}
procedure Get_Cert(Filename: String);
\end{lstlisting}\\
Die Prozedur dient zum einlesen eines Zertifikats. Akzeptiert werden *.txt und *.cer Files. Allerdings k�nnen nur die CER-Files auf Echtheit �berpr�ft werden.\\
Beim Importieren wird die Struktur und die G�ltigkeitsdauer des Zertifikats �berpr�ft. Ist das Zertifikat noch aktuell, wird die Signatur mit dem �ffentlichen Schl�ssel des Austellers entschl�sselt und mit dem Zertifikat-Hash verglichen. Bei einem Wurzelzertifikat befindet sich der Public Key in dem selbigen. So wird die Korrektheit der Werte sichergestellt.\\ \ \\
\hline
\end{tabular}
\begin{tabular}{p{\textwidth}}
\begin{lstlisting}{}
procedure CertOut;
\end{lstlisting}\\
Die Prozedur \textit{CertOut} gibt das Zertifikat in Textform aus.\\ \ \\
\hline
\end{tabular}
\begin{tabular}{p{\textwidth}}
\begin{lstlisting}{}
function AnalyseSigned return Unbounded_String;
\end{lstlisting}\\
Die Funktion \textit{AnalyseSigned} entschl�sselt die Signatur des in \textit{Get\_Cert} Importierten Zertifikats und gibt einen Unbounded\_String zur�ck. Man erh�lt bei erfolgreicher Dekodierung einen Hash-Wert (die L�nge variiert je nach verwendeten Hash-Typ MD5 oder Sha-1). Ansonsten erh�lt man einen Null\_Unbounded\_String.\\ \ \\
\end{tabular}


\section{Exceptions}
\begin{tabular}{p{\textwidth}}
\begin{lstlisting}{}
Cert_Typ_Error	:	exception;
\end{lstlisting}\\
Diese Ausnahme wird geworfen wenn es sich beim Importieren um ein falsches Format handelt.\\ \ \\
\hline
\end{tabular}

\begin{tabular}{p{\textwidth}}
\begin{lstlisting}{}
Cert_Structure_Error : exception;
\end{lstlisting}\\
Diese Ausnahme wird geworfen wenn die Struktur des Zertifikats nicht stimmt.\\ \ \\
\hline
\end{tabular}

\begin{tabular}{p{\textwidth}}
\begin{lstlisting}{}
ASN1_Structure_Error : exception;
\end{lstlisting}\\
Diese Ausnahme wird im Kinderpaket crypto.certificate.asn1 geworfen, wenn die Struktur der Bin�rdaten nicht nach dem asn1 Standart vorliegen.\\ \ \\
\hline
\end{tabular}

\begin{tabular}{p{\textwidth}}
\begin{lstlisting}{}
Conversion_Error  : exception;
\end{lstlisting}\\
Diese Ausnahme wird im Kinderpaket crypto.certificate.asn1 geworfen, wenn keine Bin�rdaten vorliegen.\\ \ \\
\end{tabular}


\section{Kinderpaket}
Das Paket Crypto.Certificate.ASN1 dient lediglich zur Konvertierung eines Zertifikats im Format CER.\\
Es verf�gt �ber folgende High-Level-API\\
\begin{tabular}{p{\textwidth}}
\begin{lstlisting}{}
procedure ASN1_Out;
\end{lstlisting}\\
Mit Hilfe dieser Prozedur wird das Zertifikat in Bin�r Darstellung ausgegeben. Es muss vorher ein Zertifikat mittels \textit{Get\_Cert(Filename: String)} eingelesen sein.\\ \ \\
\hline
\end{tabular}
\\ \ \\
Die Low-Level API\\
\begin{tabular}{p{\textwidth}}
\begin{lstlisting}{}
procedure Parse(ASN1Data : Unbounded_String);;
\end{lstlisting}\\
Diese Prozedur wird direkt beim Einlesen eines Zertifikats im CER-Format benutzt und dient zum Entschl�sseln der Werte eines nach dem Standard ASN1 (mit Distinct Encoding Rules (DER)) bin�r codierten Zertifikat.\\ \ \\
\hline
\end{tabular}


\section{Anwendungsbeispiel}
\begin{lstlisting}{}
with Ada.Text_IO; use Ada.Text_IO;
with Crypto.Certificate; 
use Crypto.Certificate;

procedure Example_Cert is
   
begin
   --Zertifikat importiern
   Get_Cert("1.cer");

   --Zertifikat ausgeben
   CertOut;

end Example_Cert;
\end{lstlisting}