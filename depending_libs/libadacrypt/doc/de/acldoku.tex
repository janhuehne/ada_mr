\documentclass[11pt,a4paper]{report}

\usepackage{a4}
\usepackage{ngerman}
\usepackage{latexsym}
\usepackage[latin1]{inputenc}
\usepackage{pstricks, pst-node, pst-tree}
\usepackage{listings}

\lstset{language=[95]Ada}

\linespread{1.1}

\title{Ada Crypto Lib (ACL) aka libadacrypt-dev\\
  Version 0.1.2 (beta)\\
  Benutzer-Dokumentation }
\author{Christian Forler}
\date{\today}

%Trennung
\hyphenation{dev Gathering Ciphertext Message libacl Byte Word DWord Arrays}
\hyphenation{Crypto Ferguson Initial Hash-funktionen Private}

\begin{document}

\input ngerman.sty
% Gro�-/Kleinschreibung der deutschen Umlaute
\catcode`�=\active \def�{"a}
\catcode`�=\active \def�{"o}
\catcode`�=\active \def�{"u}
\catcode`�=\active \def�{"s}
\catcode`�=\active \def�{"A}
\catcode`�=\active \def�{"O}
\catcode`�=\active \def�{"U}

\maketitle
\tableofcontents

  \include{acleinfuehrung}
  \include{acl.crypto}
  \include{acl.crypto.types}
  \include{acl.crypto.random}
  \include{acl.crypto.symmetric}
  \include{acl.crypto.symmetric.algorithm} 
  \include{acl.crypto.symmetric.algorithm.oneway}
  \include{acl.crypto.symmetric.blockcipher}
  \include{acl.crypto.symmetric.oneway_blockcipher}
  \include{acl.crypto.symmetric.mode}
  \include{acl.crypto.symmetric.mode.oneway}
  \include{acl.crypto.hashfunction}
  \include{acl.crypto.symmetric.mac}
  \include{acl.crypto.types.big-numbers}
  \include{acl.crypto.types.big-numbers.binfield}
  \include{acl.crypto.asymmetric}
  \include{acl.crypto.asymmetric.dsa}
  \include{acl.crypto.asymmetric.rsa}
  \include{acl.crypto.types.elliptic-curves}
  \include{acl.crypto.types.elliptic-curves.databasezp}
  \include{acl.crypto.asymmetric.ecdsa}
  \include{acl.crypto.asymmetric.ecnr}
  \include{acl.crypto.symmetric.algorithm.ecdh}
  \include{acl.crypto.symmetric.algorithm.ecmqv}
  \include{acl.crypto.symmetric.algorithm.ecies}  
  \include{acl.crypto.certificate}


% Literaturdatenbank
\bibliographystyle{plain} 
\bibliography{crypto}

\end{document}

    
    


