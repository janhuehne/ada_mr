\documentclass[11pt,a4paper]{report}

\usepackage{a4}
\usepackage{ngerman}
\usepackage{latexsym}
\usepackage[latin1]{inputenc}
\usepackage{pstricks, pst-node, pst-tree}
\usepackage{listings}

\lstset{language=[95]Ada}

\linespread{1.1}

\title{Ada Crypto Lib (ACL) aka libadacrypt-dev\\
  Version 0.1.2 (beta)\\
  Benutzer-Dokumentation }
\author{Christian Forler}
\date{\today}

%Trennung
\hyphenation{dev Gathering Ciphertext Message libacl Byte Word DWord Arrays}
\hyphenation{Crypto Ferguson Initial Hash-funktionen Private}

\begin{document}

\input ngerman.sty
% Gro�-/Kleinschreibung der deutschen Umlaute
\catcode`�=\active \def�{"a}
\catcode`�=\active \def�{"o}
\catcode`�=\active \def�{"u}
\catcode`�=\active \def�{"s}
\catcode`�=\active \def�{"A}
\catcode`�=\active \def�{"O}
\catcode`�=\active \def�{"U}

\maketitle
\tableofcontents

  \chapter{Einleitung}
Bei der  Ada Crypto Library (ACL) handelt es sich um eine freie 
kryptographische Bibliothek f�r Ada. Eines der beiden Hauptziele bei der
Entwicklung dieser Bibliothek war das Design einer
\textbf{selbst sprechenden und sauberen API}. Das andere Hauptziel war es,
einen m�glichst sauberen Programmierstil zu verwenden, der die formale 
Verifikation des Codes erleichtert.
Aus diesem Grund wurde bei der ACL komplett auf folgende ``Features''
verzichtet:
\begin{itemize}
\item access types (Zeiger)
\item inline Assembler
\item tagged types (Objekte)
\item goto statements
\end{itemize}
Auf goto statements und access-types wurde verzichtet, da diese den 
Quelltext un�bersichtlicher machen und Probleme bei der Verifikation des 
Codes auftreten k�nnen. Auf tagged types wurde verzichtet, da hier
Prozeduren und Methoden �berschrieben werden k�nnen, und erst zur Laufzeit 
dynamisch ermittelt wird welche Methode verwendet wird (dynamisches 
dispatching). Hierdurch treten massive Probleme bei Verifikation des
Codes auf. Auf inline Assembler wurde verzichtet, da dies ein sehr unsauberer
Programmierstil ist und dies die starke Typisierung von ADA umgehen kann.
Was nicht gerade den wahren Ada-Weg entspricht.
Durch diese Einschr�nkungen erh�ht sie zwar die Qualit�t des Quelltextes
im Bezug auf Verifizierbarkeit und Sicherheit, f�hrt aber zu einer
schlechteren Performance. 
Wenn Sie also mehr Wert auf Performance als auf ein sauberes Design
legen, dann sollten Sie sich vielleicht nach einer anderen kryptographischen
Bibliothek umsehen. Leider sind mir keine weiteren freie kryptographische
Bibliothek f�r Ada bekannt auf die ich sie jetzt verweisen k�nnte.\\
Dies ACL hat mehr den Status eines \glqq Proof of Concept''.
Es fand weder ein ``Code-Review'' noch ein ``Code-Audit'' statt. 
Bis jetzt ist der Autor der ACL der Einzige,
der sich mit dem Code etwas n�her befasst hat. Es besteht z.B.
durchaus die M�glichkeit, da� sich sicherheitskritisches Material wie der 
Schl�ssel sich nach der Ausf�hrung eines Programms das die ACL verwendet im
RAM ``�berlebt''. Ein Storage-Pool der genau diese Aufgabe �bernimmt ist in 
Planung. Die meisten kryptographische Bibliotheken die auch dieses 
Sicherheitsproblem haben weisen nicht darauf hin. Von solchen Bibliotheken sollte 
man Abstand nehmen, da diese nicht mit offenen Karten spielen.\\
Im folgendem habe ich noch einmal alle  mir bekannten Nachteile der ACL
aufgelistet.  
\begin{itemize}
\item fehlende S�uberung des Stacks und Heaps 
\item kein ``big endian'' Support
\item schlechte Performance
\end{itemize}
In dieser Dokumentation wird kurz auf die Installation und den topologischen 
Aufbau eingegangen. Danach widmet sich diese Dokumentation ausschlie�lich der 
API. Jedes Paket und dessen API wird in einem separaten Kapitel vorgestellt.
Am Ende jedes Kapitels finden Sie ein Anwendungsbeispiel.\\
Falls sie noch Fragen zur ACL haben, einen Fehler finden oder die ACL um
ein bzw. mehrere Pakete erweitern wollen, dann k�nnen  sie
mich per E-Mail unter folgender Adresse kontaktieren: crypto@shortie.inka.de.\\

\subsubsection{TODO-Liste}
\begin{itemize}
\item Optimierung
\item Eigener Storage Pool
\item Erweiterung (Tiger2, RSA-PSS, Poly1305-AES usw.)   
\item Fallunterscheidung f�r die Anwendung des RNG (Windows/Linux)
\end{itemize}

%%%%%%%%%%%%%%%%%%%%%%%%%%%%%%%%%%%%%%%%%%%%%%%%%%%%%%%%%%%%%%%%%%%%%%%%%%%

\section{Installation HOWTO}
\subsection{libadacrpyt-dev}
\subsubsection{Abh�ngigkeiten}
Unter Linux ben�tigen Sie folgende Pakete um die ACL zu installieren:
\begin{itemize}
\item tar 
\item gnat 
\item make
\item bunzip2 
\item binutils
\end{itemize}

\subsubsection{Kompilieren}
Mit der folgenden Befehlssequenzen entpacken und kompilieren sie die ACL sowie 
den beiliegenden Regressionstest.
\begin{itemize}
\item \texttt{tar cjf acl.tar.bz2}
\item \texttt{cd crypto}
\item \texttt{make}
\item \texttt{make acltest}
\end{itemize}

\subsubsection{Testen}
Bevor Sie die ACL installieren sollten Sie unbedingt den Regressionstest
durchf�hren um sicherzustellen, das die ACL auf ihrem System einwandfrei 
arbeitet. Der Regressionstest dauert auf einem PII 450 ungef�hr 30 Sekunden.
Mit dem folgender Befehlssequenz f�hren Sie den Regressiontest aus.\\
\begin{itemize}
\item \texttt{cd test}
\item \texttt{./acltest}
\item \texttt{cd ..}
\end{itemize}

\subsubsection{Installieren}
Wenn bei dem Test kein Fehler auftrat k�nnen sie die ACL mit folgendem
Befehl installieren:\\
\hspace*{1cm} \texttt{su -c ``make install''}\ \\

\subsubsection{Deinstallieren}
Mit dem folgenden Befehl k�nnen sie ACL wieder deinstallieren:\\
\hspace*{1cm} \texttt{su -c ``make uninstall''}\ \\

\subsubsection{Rekompilieren}
Mit der folgenden Befehlssequenzen k�nnen sie ACL neu kompilieren:\\
\begin{itemize}
\item \texttt{make clean}
\item \texttt{make clean-acltest}
\item \texttt{make}
\item \texttt{make acltest}
\end{itemize}

\subsection{libadacrypt}
\subsubsection{Installieren}
Sie k�nnen die ACL auch mit den folgenden Befehlen als shared library 
(libacl.so) installieren:
\begin{itemize}
\item \texttt{make shared}
\item \texttt{make install-shared}
\end{itemize}

\subsubsection{deinstallieren}
Mit dem folgenden Befehl k�nnen sie ACL wieder deinstallieren:\\
\hspace*{1cm} \texttt{su -c ``make uninstall-shared''}\ \\

\subsection{Dokumentation}
Um diese Dokumentation zu erstellen wird das Paket tetex-bin (latex) und
tetex-extra ben�tigt. Wenn n�tig sollte sie diese beiden Pakete installieren.
Mit dem folgenden beiden Befehlen wird die komplette Dokumentation (de+en) 
erstellt und nach \texttt{/usr/local/share/doc/libadacrypt-dev} kopiert.
\begin{itemize}
\item \texttt{make doc}
\item \texttt{su -c ``make install-doc''}
\end{itemize}\ \\
Mit den folgenden beiden Befehlen k�nnen sie die Dokumentationen unter 
 \texttt{/usr/local/share/doc/libadacrypt-dev} deinstallieren und die 
Dokumentation \glqq l�schen''.
\begin{itemize}
\item \texttt{make clean-doc}
\item \texttt{su -c ``make uninstall-doc''}
\end{itemize}


\subsection{Anpassungen}
Im Unterverzeichnis \texttt{src} befindet sich eine Datei \texttt{Makefile}. 
In dieser k�nnen Sie bei den folgende Variablen Anpassungen vornehmen:
\begin{itemize}
\item LIBDIR :  Installationspfad der shared library.
\item INSTDIR : Installationspfad der ACL.
\end{itemize}\ \\
Ausserdem k�nnen sie noch bei den Makefiles in den Unterverzeichnissen von 
\texttt{doc} durch Anpassung der Variablen \texttt{DOCPATH} vornemen.
Diese Variable enth�lt den Installationspfad der entsprechenden Dokumentation.
 

\subsection{Benchmark}
In dem Unterverzeichnis \texttt{bench} befindet sich ein Benchmark, der in einer
sp�teren Version Aussagen �ber die Performance der ACL auf dem eigenen System 
erm�glichen soll. Der Aufbau orientiert sich an dem des Regressionstests. Bisher
wurde eine M�glichkeit zum Messen der Multiplikationsalgorithmen aus dem Package
\texttt{acl.crypto.types.big-numbers} umgesetzt. Der Benchmark generiert dazu
Zufallszahlen, aus denen die Faktoren bestimmt werden. Das Produkt wird durch die
jeweils durch alle implementierten Algorithmen bestimmt und die zur Berechnung
ben�tigte Zeit ausgegeben.\\
Kompilieren und Ausf�hren l�sst sich der Benchmark durch

\begin{itemize}
\item \texttt{cd bench}
\item \texttt{make}
\item \texttt{./aclbench}
\end{itemize}

Durch einen Aufruf des Benchmarks mit Kommandzeilenparameter \texttt{-CSV} lassen
sich die Messungen kommasepariert ausgeben. Dadurch kann man die Ergebnisse 
einfacher in ein Tabellenkalkulationsprogramm importieren und auswerten.\\
Unter Linux:

\begin{itemize}
\item \texttt{./aclbench -CSV > Dateiname.txt}
\end{itemize}

Im Unterverzeichnis \texttt{bench} befindet sich eine Datei \texttt{Makefile}. 
In dieser k�nnen Sie bei den folgende Variablen Anpassungen vornehmen:\\

Kompilieren des Benchmarks ohne Optimierung:
\begin{lstlisting}{}
aclbench.o :
   $(CC) $(CFLAGS) aclbench
   $(BB) $(BFLAGS) aclbench.ali
   $(LL) $(LFLAGS) aclbench.ali
\end{lstlisting}

Kompilieren des Benchmarks mit O3 Optimierung:
\begin{lstlisting}{}
aclbench.o :
   $(CC) $(CFLAGS_O3) aclbench
   $(BB) $(BFLAGS_O3) aclbench.ali
   $(LL) $(LFLAGS_O3) aclbench.ali
\end{lstlisting}

Kompilieren des Benchmarks mit Symbolen f�r \texttt{gdb} oder \texttt{gnatgdb} :
\begin{lstlisting}{}
aclbench.o :
   $(CC) $(CFLAGS_DBUG) aclbench
   $(BB) $(BFLAGS_DBUG) aclbench.ali
   $(LL) $(LFLAGS_DBUG) aclbench.ali
\end{lstlisting}

Kompilieren des Benchmarks mit Symbolen f�r \texttt{gprof} oder \texttt{kprof}:
\begin{lstlisting}{}
aclbench.o :
   $(CC) $(CFLAGS_GPROF) aclbench
   $(BB) $(BFLAGS_GPROF) aclbench.ali
   $(LL) $(LFLAGS_GPROF) aclbench.ali
\end{lstlisting}

Kompilieren des Benchmarks mit Symbolen f�r \texttt{gcov} oder \texttt{ggcov}:
\begin{lstlisting}{}
aclbench.o :
   $(CC) $(CFLAGS_GCOV) aclbench
   $(BB) $(BFLAGS_GCOV) aclbench.ali
   $(LL) $(LFLAGS_GCOV) aclbench.ali
\end{lstlisting}

%%%%%%%%%%%%%%%%%%%%%%%%%%%%%%%%%%%%%%%%%%%%%%%%%%%%%%%%%%%%%%%%%%%%%%%%%%%

\section{Verzeichnis- und Paket-Struktur}
\subsection{Verzeichnis-Struktur}
\begin{itemize}
\item doc : Dokumentation
\item ref : Referenzen und Spezifikationen
\item src : Quellcode 
\item test : Regressionstest
\item bench : Benchmark
\end{itemize}

%%%%%%%%%%%%%%%%%%%%%%%%%%%%%%%%%%%%%%%%%%%%%%%%%%%%%%%%%%%%%%%%%%%%%%%%%%%

\subsection{Paket-Struktur}
Die Ada Crypto Library (ACL) besteht aus den folgenden Paketen (Komponenten).
\begin{description}
\item[Crypto:]\ \\ 
  Dies ist das Wurzelpaket der ACL.
  Alle anderen Pakete der ACL beginnen mit dem 
Pr�fix \textit{Crypto.}
\item[Crypto.Types:] \ \\
  In diesem Paket befinden sich grundlegende Typen der ACL (z.B. Byte)
  und deren  Basisfunktionen.
  Ein Einsatz der ACL ohne Einbindung dieses Paketes ist nur sehr begrenzt 
  m�glich.
\item[Crypto.Types.Big\_Numbers:] \ \\
  Dieses generische Packet unterst�tzt die Arithmetik in $Z_p$ oder 
  $GF(2^m)$. Der Basistyp hierf�hr ist eine modulare n-Bit Zahl.
  Dieses Packet wird f�r die asymetrische Kryptrographie ben�tigt.
\item[Crypto.Types.Elliptic\_Curves:] \ \\
  Dieses generischen Pakete ist das Wurzelpacket f�r elliptische Kurven.
\item[Crypto.Random:] \ \\
  Dieses Paket ist eine Schnittstelle zwischen einem externen
  Pseudozufallsbitgenerator  und der ACL.
\item[Crypto.Symmetric:]\ \\
  Die ist das Wurzelpaket f�r den symmetrische Zweig.
\item[Crypto.Symmetric.Algorithm:]\ \\
  In diesem Zweig des ACL-Baums befinden sich die symmetrische Algorithmen
  f�r symmetrische Blockchiffren und Hashfunktionen.
\item[Crypto.Symmetric.Algorithm.Oneway:]\ \\
  Jeder Algorithmus verf�gt �ber einen Oneway-Algorithmus.
  Symmetrisch Oneway-Algorithmen sind einfach symmetrische Algorithmen die nur 
  in eine Richtung, entweder Ver- oder Entschl�sseln arbeiten.
\item[Crypto.Symmetric.Blockcipher:]\ \\
  Mit Hilfe dieses generische Paketes k�nnen Sie aus einem symmetrischen
  Algorithmus eine Blockchiffre generieren.
\item[Crypto.Symmetric.Oneway\_Blockcipher:]\ \\
  Mit Hilfe dieses generische Paketes k�nnen Sie aus einem symmetrischen
  Oneway-Algorithmus eine Einweg-Blockchiffre generieren.
\item[Crypto.Symmetric.MAC:]
  Dies ist das Wurzelpacket f�r MACs.
\item[Crypto.Symmetric.Mode:]\ \\
  In diesem Zweig befinden sich verschiedene Betriebsmodi f�r Blockchiffren.
\item[Crypto.Symmetric.Mode.Oneway:]\ \\
  In diesem Zweig befinden sich verschiedene Betriebsmodi f�r 
  Einweg-Blockchiffren.
\item[Crypto.Asymmetric:]\ \\
  Dies ist das Wurzelpaket des asymmetrischen Zweigs.
\item[Crypto.Asymmetric.DSA:]\ \\
  Mit Hilfe dieses Pakets k�nnen digitale Unterschriften erstellt und 
  verifiziert werden.
\item[Crypto.Asymmetric.RSA:]\ \\
  Mit Hilfe dieses Paketes lassen sich Daten  asymmetrisch ver- bzw.
  entschl�sseln
\item[Crypto.Hashfunction:]
  Mit Hilfe dieses generische Paketes k�nnen Sie aus einem entsprechenden
  symmetrischen Algorithmus eine Hashfunktion generieren.\\
\end{description}


\begin{figure}
 \scalebox{0.75}{%
  \pstree[nodesep=2pt]{\TR{Crypto}}{%
    \pstree{\TR{Types}}{%
      \TR{Big\_Numbers} \TR{Elliptic\_Curves}}
    \TR{Hashfunction}
    \pstree{\TR{Symmetric}}{%
      \pstree{\TR{Algorithm}}{%
	\Tr{Oneway}}
      \TR{Blockcipher} 
      \TR{MAC}
      \TR{Oneway\_Blockcipher}
      \pstree{\TR{Mode}}{%
	\TR{Oneway}}} % end symmetric
    \TR{Random} 
    \pstree{\TR{Asymmetric}}{%
      \TR{DSA} \TR{RSA}}
}}
\end{figure}
    

  \chapter{Crypto}

Crypto is an empty root package. This package does not contain any code.
It's sole purpose is to provide an unique namespace for all ACL packages.
As all packages are descendants of this package, their names start with
the prefix Crypto (e.g. Crypto.Foo.Bar).


 

  \chapter{Crypto.Types}
Dieses Paket stellt die elementaren Typen der ACL und dessen  Basisfunktionen 
zur Verf�gung.\\ \ \\
{\large \textbf{WICHTIG:} \\
  Bei der Verwendung der ACL sollten Sie unbedingt dieses Paket mittels
  \begin{lstlisting}{}
    with Crypto.Types;
  \end{lstlisting}
  importieren}

%%%%%%%%%%%%%%%%%%%%%%%%%%%%%%%%%%%%%%%%%%%%%%%%%%%%%%%%%%%%%%%%%%%%%%%%%%%
%%%%%%%%%%%%%%%%%%%%%%%%%%%%%%%%%%%%%%%%%%%%%%%%%%%%%%%%%%%%%%%%%%%%%%%%%%%
%%%%%%%%%%%%%%%%%%%%%%%%%%%%%%%%%%%%%%%%%%%%%%%%%%%%%%%%%%%%%%%%%%%%%%%%%%%

\section{Elementare Typen}
Bei den elementaren bzw. prim�ren Typen handelt es sich ausschlie�lich um
modulare Typen. D.h. ein Variablen�berlauf bzw -unterlauf f�hrt nicht zu 
einer Exception. Wenn das Ergebnis einer Operation nicht im Wertebereich des
modularen Types liegt, dann wird $n:=2**Type'Size=Type'Last+1$ solange 
addiert bzw. subtrahiert, bis das Ergebnis wieder im Wertebereich des 
modularen Types liegt.\\

\begin{lstlisting}{}
type Bit is mod 2 ** 1;
for Bit'Size use 1;

type Byte is mod 2 ** 8;
for Byte'Size use  8;

type Word is mod 2 ** 32;
for Word'Size use 32;

type DWord is mod 2 ** 64;
for DWord'Size use 64;
\end{lstlisting}

%%%%%%%%%%%%%%%%%%%%%%%%%%%%%%%%%%%%%%%%%%%%%%%%%%%%%%%%%%%%%%%%%%%%%%%%%%%
%%%%%%%%%%%%%%%%%%%%%%%%%%%%%%%%%%%%%%%%%%%%%%%%%%%%%%%%%%%%%%%%%%%%%%%%%%%

\subsubsection{Beispiel}
\begin{lstlisting}[frame=brtl]{Modulare Typen}
with Crypto.Types;
with Ada.Text_IO;
use Crypto.Types;

procedure Example is
   -- Byte hat einen Wertebereich von 0..255
   A, B : Byte;
begin
   A := 100;
   B := A + 250; -- Ueberlauf bei B
   A := A - 250; -- Unterlauf bei A
   Ada.Text_IO.Put_Line("A: " &  A'IMG);
   Ada.Text_IO.Put_Line("B: " &  B'IMG);
end example;
\end{lstlisting}\ \\
\textit{Ausgabe des Programmes:\\}
\texttt{A:  106\\B:  94}


%%%%%%%%%%%%%%%%%%%%%%%%%%%%%%%%%%%%%%%%%%%%%%%%%%%%%%%%%%%%%%%%%%%%%%%%%%%
%%%%%%%%%%%%%%%%%%%%%%%%%%%%%%%%%%%%%%%%%%%%%%%%%%%%%%%%%%%%%%%%%%%%%%%%%%%
%%%%%%%%%%%%%%%%%%%%%%%%%%%%%%%%%%%%%%%%%%%%%%%%%%%%%%%%%%%%%%%%%%%%%%%%%%%


\section{Abgeleitete Typen}
Abgeleitete Typen sind Typen die von elementaren Typen abgeleitet werden.
In der Regel handelt es sich dabei um Felder die aus Elementaren Typen
bestehen. \textbf{Alle nicht privaten Felder, die aus elementaren Typen
  bestehen, werden in der ACL als n-Bit Zahlen interpretiert}, wobei das 
erste Element (First) als das h�chstwertigste und das letzte Element (Last)
als der niederwertigster Teil dieser Zahl interpretiert wird.
D.h. Die Wertigkeit eines Elements sinkt mit der H�he des Index.

\begin{tabular}{p{\textwidth}}
\subsubsection{Bit}
\begin{lstlisting}{Bits}
  type Bits is array (Integer range <>) of Bit;
\end{lstlisting}\ \\ \ \\
\hline 
\end{tabular}


\begin{tabular}{p{\textwidth}}
\subsubsection{Bytes} 
\begin{lstlisting}{Bytes}
  type Bytes is array (Integer range <>) of Byte;
  
  subtype Byte_Word  is Bytes (0 .. 3);
  subtype Byte_DWord is Bytes (0 .. 7);

  subtype B_Block32  is Bytes (0 ..  3);
  subtype B_Block48  is Bytes (0 ..  5);
  subtype B_Block56  is Bytes (0 ..  6);
  subtype B_Block64  is Bytes (0 ..  7);
  subtype B_Block128 is Bytes (0 .. 15);
  subtype B_Block160 is Bytes (0 .. 19); 
  subtype B_Block192 is Bytes (0 .. 23);
  subtype B_Block256 is Bytes (0 .. 31);
\end{lstlisting}
Ein B\_BlockN besteht immer aus einem N Bit-Array das in N/8 Bytes
unterteilt ist.
Der subtype B\_Block256 ist z.B. ein 256-Bitstring der in 32 (256/8) Bytes
unterteilt ist. In Ada l�sst sich dies durch ein  32-elementiges
Byte-Array abbilden.\\ \ \\
\hline
\end{tabular}

%%%%%%%%%%%%%%%%%%%%%%%%%%%%%%%%%%%%%%%%%%%%%%%%%%%%%%%%%%%%%%%%%%%%%%%%%%%

\begin{tabular}{p{\textwidth}}
  \subsubsection{Words} 
  \begin{lstlisting}{Words}
  type Words   is array (Integer range <>) of Word;
  
  subtype W_Block128  is Words(0 .. 3);
  subtype W_Block160  is Words(0 .. 4);
  subtype W_Block192  is Words(0 .. 5);
  subtype W_Block256  is Words(0 .. 7);
  subtype W_Block512  is Words(0 .. 15);
  \end{lstlisting}
  Ein W\_BlockN besteht immer aus einem N Bit-Array das in N/32 Words
  unterteilt ist.
  Der subtype W\_Block256 ist z.B. ein 256-Bitstring der in 8 (256/32) Word
  Elementen unterteilt ist. In Ada l�sst sich dies durch ein  8-elementiges
  Word-Array abbilden.\\ \ \\
\hline
\end{tabular}

%%%%%%%%%%%%%%%%%%%%%%%%%%%%%%%%%%%%%%%%%%%%%%%%%%%%%%%%%%%%%%%%%%%%%%%%%%%
  
\begin{tabular}{p{\textwidth}}
  \subsubsection{DWords} 
  \begin{lstlisting}{DWords}
  type DWords   is array (Integer range <>) of DWord  
    
  subtype DW_Block256   is DWords(0 ..  3);
  subtype DW_Block384   is DWords(0 ..  5);
  subtype DW_Block512   is DWords(0 ..  7);
  subtype DW_Block1024  is DWords(0 .. 15);
  \end{lstlisting}
  Ein DW\_BlockN besteht immer aus einem N Bit-Array das in N/64 Words
  unterteilt ist.
  Der subtype DW\_Block256 ist z.B. ein 256-Bitstring der in 4 (256/64) DWord
  Elementen unterteilt ist. In Ada l�sst sich dies durch ein 4-elementiges
  DWord-Array abbilden.\\ \ \\
\hline
\end{tabular}

%%%%%%%%%%%%%%%%%%%%%%%%%%%%%%%%%%%%%%%%%%%%%%%%%%%%%%%%%%%%%%%%%%%%%%%%%%%

\begin{tabular}{p{\textwidth}}
  \subsubsection{Strings} 
  \begin{lstlisting}{Strings}
    subtype Hex_Byte  is String (1.. 2);
    subtype Hex_Word  is String (1.. 8);
    subtype Hex_DWord is String( 1..16);
  \end{lstlisting}\ \\
  \hline
\end{tabular}


%%%%%%%%%%%%%%%%%%%%%%%%%%%%%%%%%%%%%%%%%%%%%%%%%%%%%%%%%%%%%%%%%%%%%%%%%%%
  
\begin{tabular}{p{\textwidth}}
\subsubsection{Nachrichtenbl�cke} 
\begin{lstlisting}{Nachrichtenbloecke}
subtype Message_Block_Length512  is Natural range 0 ..  64;
subtype Message_Block_Length1024 is Natural range 0 .. 128;

\end{lstlisting}
Die Message\_Block\_Length Typen gibt die L�nge in Zeichen (Bytes) einer 
Nachricht innerhalb eines 512- bzw. 1024-Bit Nachrichtenblockes M an.
Wenn man z.B. eine 1152-Bit lange Nachricht in 512-Bit Bl�cke unterteilt, dann
erh�lt man drei 512-Bit Bl�cke. Wobei die Nachrichtenl�nge im letzten Block 16
($1280 - 2 \cdot 512 = 128 / 8 = 16$) ist.
Die restlichen 384-Bit des letzten Nachrichtenblockes sind \glqq leer \grqq, 
d.h. sie enthalten keine Teile der urspr�nglichen Nachricht.
Eine Variable des Types M\_Length512 w�rde man in diesem
Fall auf 32 setzen. Diese Variablen-Typen werden also f�r das Auff�llen von
Nachrichtenbl�cken (padding) ben�tigt. Mehr Informationen zum Thema ``Padding''
gibt es im Kapitel �ber Hashfunktionen (\ref{hash}).\\ \ \\
\end{tabular}\ \\

%%%%%%%%%%%%%%%%%%%%%%%%%%%%%%%%%%%%%%%%%%%%%%%%%%%%%%%%%%%%%%%%%%%%%%%%%%%
%%%%%%%%%%%%%%%%%%%%%%%%%%%%%%%%%%%%%%%%%%%%%%%%%%%%%%%%%%%%%%%%%%%%%%%%%%%

\section{Funktionen und Prozeduren}
\subsection{Bitverschiebungs Funktionen}
\begin{lstlisting}{Types-functions}
   function Shift_Left   (Value  : Byte;
                          Amount : Natural) 
			  return Byte;

   function Shift_Right  (Value  : Byte;
                          Amount : Natural) 
			  return Byte;

   function Rotate_Left  (Value  : Byte;
                          Amount : Natural)
			  return Byte;

   function Rotate_Right (Value  : Byte;
                          Amount : Natural)
			  return Byte;


   function Shift_Left   (Value : Word;
                          Amount : Natural) 
                          return Word;

   function Shift_Right  (Value  : Word; 
                          Amount : Natural) 
			  return Word;

   function Rotate_Left  (Value  : Word; 
                          Amount : Natural) 
			  return Word;

   function Rotate_Right (Value  : Word; 
                          Amount : Natural) 
			  return Word;


   function Shift_Left   (Value  : DWord; 
                          Amount : Natural) 
			  return DWord;

   function Shift_Right  (Value  : DWord;
                          Amount : Natural)
			  return DWord;

   function Rotate_Left  (Value  : DWord; 
                          Amount : Natural)
                           return DWord;

   function Rotate_Right (Value  : DWord;
                          Amount : Natural) 
			  return DWord;
\end{lstlisting}\ \\

%%%%%%%%%%%%%%%%%%%%%%%%%%%%%%%%%%%%%%%%%%%%%%%%%%%%%%%%%%%%%%%%%%%%%%%%%%%
%%%%%%%%%%%%%%%%%%%%%%%%%%%%%%%%%%%%%%%%%%%%%%%%%%%%%%%%%%%%%%%%%%%%%%%%%%%

\subsection{Basis Funktionen}\ 
\begin{tabular}{p{\textwidth}}
  \textbf{\grqq XOR''}\ \\
  \begin{lstlisting}{}
    function "xor" (Left, Right : Bytes)   return Bytes;
    function "xor" (Left, Right : Words)   return Words;
    function "xor" (Left, Right : DWords) return DWords;
  \end{lstlisting}
  \underline{Vorbedingung:}\\ Left'Length = Right'Length\\
  \underline{Exception:}\\
  Verletzung der Vorbedingung :  Constraint\_Byte/Word/DWord\_Error\\ \ \\
  Diese Funktionen verkn�pfen die beiden Eingabefelder feldweise mit
  der XOR-Operation. D.h. Left(Left'First) wird z.B. mit
  Right(Right'First) XOR-verkn�pft und Left(Left'Last) wird mit
  Right(Right'Last) XOR-verkn�pft.\\ \ \\
  \hline\\
\end{tabular}

%%%%%%%%%%%%%%%%%%%%%%%%%%%%%%%%%%%%%%%%%%%%%%%%%%%%%%%%%%%%%%%%%%%%%%%%%%%

 \textbf{``+''}
 \begin{lstlisting}{}
     function "+" (Left : Bytes;  Right : Byte)   return Bytes;
     function "+" (Left : Byte;   Right : Bytes)  return Bytes;
     function "+" (Left : Words;  Right : Word)   return Words;
     function "+" (Left : Word;   Right : Words)  return Words;
     function "+" (Left : Words;  Right : Byte)   return Words;
     function "+" (Left : DWords; Right : DWord)  return DWords;
     function "+" (Left : DWord;  Right : DWords) return DWords;
     function "+" (Left : DWords; Right : Byte)   return DWords;
 \end{lstlisting}


\begin{tabular}{p{\textwidth}}
  Bei diesen Funktionen werden Left und Right als Zahlen interpretiert wobei
  Left(Left'First) bzw. Right(Right'First) das niederwertigste 
  Byte/Word/ DWord und Left(Left'Last) bzw. Right(Right'Last) das
  h�chstwertigste Byte /Word/DWord einer Zahl repr�sentiert.
  Das Ergebnis dieser Funktion ist die Summe der beiden Zahlen.\\ \ \\
  \textbf{Beispiel:}
\begin{lstlisting}{}
procedure plus is 
  A : Byte := 200;
  B : Bytes(0..1) := (0 => 100, 1 => 16); 
  begin
  B := A+B; -- B := 2#11001000#+2#10000_01100100# 
  -- B(0) = 2#101100# = 44
  -- B(1) = 2#010001# = 17
end plus;
\end{lstlisting}
\end{tabular}

%%%%%%%%%%%%%%%%%%%%%%%%%%%%%%%%%%%%%%%%%%%%%%%%%%%%%%%%%%%%%%%%%%%%%%%%%%%
%%%%%%%%%%%%%%%%%%%%%%%%%%%%%%%%%%%%%%%%%%%%%%%%%%%%%%%%%%%%%%%%%%%%%%%%%%%

\subsection{Umwandlungs-Funktionen}\ \\
Im folgenden gilt:\\ \ \\
$\mathtt{W : Word  \equiv B0||B1||B2||B3}$\\
$\mathtt{D : DWord \equiv B0||B1||B2||B3||B4||B5||B6||B7}$\\ \ \\
wobei B0 hier jeweils dem h�chstwertigsten Byte und B3 bzw. B7 dem 
niederwertigsten Byte von W bzw. D entspricht.

\subsubsection{ByteN}
\begin{lstlisting}{}
  function Byte0 (W : Word)  return Byte;
  function Byte1 (W : Word)  return Byte;
  function Byte2 (W : Word)  return Byte;
  function Byte3 (W : Word)  return Byte; 

  function Byte0 (D : DWord) return Byte;
  function Byte1 (D : DWord) return Byte;
  function Byte2 (D : DWord) return Byte;
  function Byte3 (D : DWord) return Byte;
  function Byte4 (D : DWord) return Byte;
  function Byte5 (D : DWord) return Byte;
  function Byte6 (D : DWord) return Byte;
  function Byte7 (D : DWord) return Byte;
\end{lstlisting}
\begin{tabular}{p{\textwidth}}
  Diese Funktionen geben das entprechende N-te Byte von W bzw. D zur�ck.
  Man beachte, das hier $\mathtt{B0 \equiv Byte0}$ usw. entspricht.\\ \ \\
  \hline\\
\end{tabular}

%%%%%%%%%%%%%%%%%%%%%%%%%%%%%%%%%%%%%%%%%%%%%%%%%%%%%%%%%%%%%%%%%%%%%%%%%%%

\begin{tabular}{p{\textwidth}}   
  \textbf{To\_Bytes}\ \\
  \begin{lstlisting}{}
    function To_Bytes(X : Word)  return Byte_Word;
    function To_Bytes(X : DWord) return Byte_DWord; 
    
    function To_Bytes(Word_Array  : Words)  return Bytes;
    function To_Bytes(DWord_Array : DWords) return Bytes;
    
    function To_Bytes(Message : String) return Bytes;
  \end{lstlisting}\\
  Die ersten beiden Funktion wandelt ein Word bzw. DWord in ein Byte\_Word bzw.
  Byte\_DWord um.
  Dabei wird das h�chstwertigste Byte von X zum ersten Element des
  R�ckgabewertes (First) und das niederwertigste Byte von X zum letzten 
  Element des R�ckgabewertes.\\ \ \\ 
  \textbf{Beispiel:}
  \begin{lstlisting}{}
    D : DWord := 16#AA_BB_CC_DD_EE_FF_11_22#;
    B : Byte_DWord := To_Bytes(D);
    -- B(0) := 16#AA#; B(1) := 16#BB#; B(2) := 16#CC#;
    -- B(3) := 16#DD#; B(4) := 16#EE#; B(5) := 16#FF#; 
    -- B(6) := 16#11#; B(7) := 16#22#;
  \end{lstlisting}\\
  Die n�chsten beiden Funktion wandelt ein Word-Array, bzw. DWord-Array in ein
  Byte-Array um.  Dabei wird das h�chstwertigste Byte des ersten
  Arrrayelemntes zum ersten Element des Byte-Arrays und das niederwertigste
  Byte vom letzten  Arrrayelement zum letzten Element des Byte-Arrays.\\ \ \\
  
  Die letzte Funktion wandelt einen (ASCII-) String in ein Byte-Array B um.
  F�r alle $I \in$ Message'Range gilt dabei :
  B(I) =  Character'Pos(Message(I)).\\  \ \\
\hline\\
\end{tabular}

%%%%%%%%%%%%%%%%%%%%%%%%%%%%%%%%%%%%%%%%%%%%%%%%%%%%%%%%%%%%%%%%%%%%%%%%%%%

\begin{tabular}{p{\textwidth}}
  \textbf{R\_To\_Bytes}\ \\
  \begin{lstlisting}{}
    function R_To_Bytes (X :  Word) return Byte_Word; 
    function R_To_Bytes (X : DWord) return Byte_DWord;
  \end{lstlisting}
  Diese Funktionen wandeln ein Word bzw. DWord X in ein Byte-Array 
  Byte\_Word bzw. Byte\_DWord B um.
  Dabei wird das h�chstwertigste Byte von X zu B(B'Last)
  und das niederwertigste Byte von X zu B(B'First).\\ \ \\
  \hline\\
\end{tabular}

%%%%%%%%%%%%%%%%%%%%%%%%%%%%%%%%%%%%%%%%%%%%%%%%%%%%%%%%%%%%%%%%%%%%%%%%%%%

\begin{tabular}{p{\textwidth}}
  \textbf{To\_Word}\ \\
  \begin{lstlisting}{}
    function To_Word (X : Byte_Word)  return Word;

    function To_Word (A,B,C,D : Byte) return Word;

    function To_Word (A,B,C,D : Character) return Word;  
\end{lstlisting}\\ 
  Die erste Funktion wandelt ein Byte\_Word in ein Word um.
  Dabei wird X(Byte\_Word'First) zum h�chstwertigesten Byte des Wortes und 
  X(Byte\_Word'Last) zum niederwertigsten Byte des Wortes.\\ \ \\
  Die zweite Funktion wandelt vier Bytes (A, B, C, D) zu einem Wort um.
  Dabei wird das erste Byte (A) zum h�chstwertigsten Byte des Wortes und das
  letzte Byte (D) zum niederwertigsten Byte des Wortes.\\  \ \\
  Die letzte Funktion wandelt vier Character (A, B, C, D) zu einem Wort um.
  Dabei wird die Position des erste Character (A'Pos) zum h�chstwertigsten
  Byte des Wortes und die Position das letzte Characters (D'Pos) zum 
  niederwertigsten Byte des Wortes.\\ \ \\
  \hline\\
\end{tabular}

%%%%%%%%%%%%%%%%%%%%%%%%%%%%%%%%%%%%%%%%%%%%%%%%%%%%%%%%%%%%%%%%%%%%%%%%%%%

\begin{tabular}{p{\textwidth}}   
  \textbf{R\_To\_Word}\ \\
  \begin{lstlisting}{}
    function R_To_Word (X : Byte_Word) return Word;   
  \end{lstlisting}
  Diese Funktion wandelt ein Byte\_Word in ein Word um.
  Dabei wird X(Byte\_Word'First) zum niederwertigsten Byte und 
  X(Byte\_Word'Last) zum h�chstwertigsten Byte des Resultats.\\ \ \\
  \hline\\
\end{tabular}

%%%%%%%%%%%%%%%%%%%%%%%%%%%%%%%%%%%%%%%%%%%%%%%%%%%%%%%%%%%%%%%%%%%%%%%%%%%


\begin{tabular}{p{\textwidth}}
  \textbf{To\_Words}\ \\
  \begin{lstlisting}{}
    function To_Words(Byte_Array : Bytes)  return Words;
  \end{lstlisting}
  Diese Funktion wandelt ein Byte-Array B in ein Word-Array W um.
  Dabei wird B als Zahl interpretiert bei der B(B'First) das niederwertigste
  Byte und  B(B'Last) das h�chstwertigste  Byte dieser Zahl darstellt. 
  Diese Zahl wird in ein Word-Array konvertiert, das ebenfalls
  als Zahl interpretiert wird, bei der das erste Element (First) den 
  h�chstwertigsten und das letzte Element (Last) den niederwertigsten Teil der
  Zahl darstellt.\\ \ \\
  \textbf{Beispiele:}\\
  \textbf{(i)}
\begin{lstlisting}{}
      B : Bytes(1..6) := (16#0A#, 16#14#, 16#1E#, 
                          16#28#, 16#32#, 16#3C#);
      W : Words := Byte_to_Dwords(B);
      -- W(D'First) = 16#0A_14_1E_28#
      -- W(D'Last)  = 16#32_3C_00_00#
    \end{lstlisting}\\ \ \\
  \textbf{(ii)}\\ \ \\
    \textit{Eingabe:}$B=\underbrace{B(1)||B(2)||B(3)||B(4)}_{W(0)}
    \underbrace{B(5)||B(6)||B(7)||B(8)}_{W(1)}
    \underbrace{B(9)||B(10)}_{W(2)}$\\ \ \\
    \textit{Ausgabe:} $W=W(0)||W(1)||W(2)$\\ \ \\
    \hline\\
\end{tabular}

%%%%%%%%%%%%%%%%%%%%%%%%%%%%%%%%%%%%%%%%%%%%%%%%%%%%%%%%%%%%%%%%%%%%%%%%%%%

\begin{tabular}{p{\textwidth}}
  \textbf{To\_DWord}\ \\
  \begin{lstlisting}{} 
    function To_DWord (X : Byte_DWord) return DWord;
  \end{lstlisting}\\
  Diese Funktion wandelt ein Byte\_DWord in ein DWord um.
  Dabei wird X(Byte\_Word'First) zum h�chstwertigesten Byte des DWords und 
  X(Byte\_Word'Last) zu dessen niederwertigsten Byte.\\ \ \\
  \hline\\
\end{tabular}

%%%%%%%%%%%%%%%%%%%%%%%%%%%%%%%%%%%%%%%%%%%%%%%%%%%%%%%%%%%%%%%%%%%%%%%%%%%
  
\begin{tabular}{p{\textwidth}}   
\textbf{R\_To\_DWord}\ \\
  \begin{lstlisting}{}
    function R_To_DWord (X : Byte_DWord) return DWord;  
  \end{lstlisting}
  Diese Funktion wandelt ein Byte\_DWord in ein DWord um.
  Dabei wird X(Byte\_Word'First) zum niederwertigsten Byte und 
  X(Byte\_Word'Last) zum h�chstwertigsten Byte des Resultats.\\ \ \\
  \hline\\
\end{tabular}

%%%%%%%%%%%%%%%%%%%%%%%%%%%%%%%%%%%%%%%%%%%%%%%%%%%%%%%%%%%%%%%%%%%%%%%%%%%

\begin{tabular}{p{\textwidth}}
  \textbf{To\_DWords}\ \\
  \begin{lstlisting}{}
  function To_DWords  (Byte_Array : Bytes) return DWords;
  \end{lstlisting}
  Diese Funktion wandelt ein Byte\_Array B in ein DWord-Array um.
  Dabei wird B als n-Bit Zahl interpretiert bei der  B(B'First)
  das h�chstwertigste Byte und B(B'Last) das niederwerigste Byte dieser
  Zahl darstellt. Diese Zahl wird in ein DWord-Array konvertiert, das ebenfalls
  als Zahl interpretiert wird, bei der das erste Element (First) den 
  h�chstwertigsten und das letzte Element (Last) den niederwertigsten Teil der
  Zahl darstellt.\\ \ \\
  \textbf{Beispiel:}
\begin{lstlisting}{}
    B : Bytes(1..10) : (10, 20, 30, 40, 50, 60, 70, 80, 90, 100);
    D : DWords := Byte_to_Dwords(B);
    -- D(D'First) =             16#0A_14_1E_28_32_3C_46_50#
    -- D(D'Last) = D(First+1) = 16#5A_64_00_00_00_00_00_00#
  \end{lstlisting}\\ \ \\
  \hline\\
\end{tabular}


%%%%%%%%%%%%%%%%%%%%%%%%%%%%%%%%%%%%%%%%%%%%%%%%%%%%%%%%%%%%%%%%%%%%%%%%%%%

\begin{tabular}{p{\textwidth}}
  \textbf{To\_String}\ \\
  \begin{lstlisting}{}
    function Bytes_To_String(ASCII : Bytes) return String;
  \end{lstlisting}
  Diese Funktion wandelt ein Byte-Array in einen String um, indem sie die 
  einzelnen Bytes als ASCII-Code interpretiert.\\ \ \\
  \hline\\
\end{tabular}\ \\


%%%%%%%%%%%%%%%%%%%%%%%%%%%%%%%%%%%%%%%%%%%%%%%%%%%%%%%%%%%%%%%%%%%%%%%%%%%

\begin{tabular}{p{\textwidth}}
  \textbf{To\_Hex}\ \\
  \begin{lstlisting}{}
    function To_Hex(B : Byte)  return Hex_Byte;
    function To_Hex(W : Word)  return Hex_Word;
    function To_Hex(D : DWord) return Hex_DWord;
  \end{lstlisting}
  Diese Funktionen wandelt ein Byte, Word oder DWord in einen Hex-String
  der L�nge 2, 8 oder 16 um.
  Beispielsweise entspricht \texttt{Put(To\_Hex(W));} in C dem Ausdruck
  \texttt{ printf(``\%08X'', i);}.\\ \ \\
  \textbf{Beispiel:}
\begin{lstlisting}{}
    B : Word := 0;
    W : Word := 16#AABBCCDDEEFF#;
    
    HB:Hex_Byte:=To_Hex(W); -- Es gilt: HB="00".
    HW:Hex_Word:=To_Hex(W); -- Es gilt: HW="0000AABBCCDDEEFF".
  \end{lstlisting}\ \\
\end{tabular}\ \\


%%%%%%%%%%%%%%%%%%%%%%%%%%%%%%%%%%%%%%%%%%%%%%%%%%%%%%%%%%%%%%%%%%%%%%%%%%%
%%%%%%%%%%%%%%%%%%%%%%%%%%%%%%%%%%%%%%%%%%%%%%%%%%%%%%%%%%%%%%%%%%%%%%%%%%%

\subsection{Is\_Zero Funktionen}
\begin{tabular}{p{\textwidth}}
  \begin{lstlisting}{}
   function Is_Zero(Byte_Array  : Bytes)  return Boolean;
   function Is_Zero(Word_Array  : Words)  return Boolean;
   function Is_Zero(DWord_Array : DWords) return Boolean;
  \end{lstlisting}
  Diese Funktionen geben ``True'' zur�ck, wenn alle Felder des �bergebenen
  Arrays den Wert ``0'' enthalten. Ansonsten geben diese Funktionen ``False''
  zur�ck.\\
\end{tabular}\ \\

%%%%%%%%%%%%%%%%%%%%%%%%%%%%%%%%%%%%%%%%%%%%%%%%%%%%%%%%%%%%%%%%%%%%%%%%%%%
%%%%%%%%%%%%%%%%%%%%%%%%%%%%%%%%%%%%%%%%%%%%%%%%%%%%%%%%%%%%%%%%%%%%%%%%%%%

\subsection{Padding Prozeduren}
 \begin{lstlisting}{}
procedure Padding(Data           : in out Bytes;
                  Message_Length : in  Word;
                  Data2          : out Bytes);

procedure Padding(Data           : in  out Words;
                  Message_Length : in  Word;
                  Data2          : out Words);

 procedure Padding(Data           : in  out DWords;
                   Message_Length : in  Word;
                   Data2          : out DWords); 
 \end{lstlisting}
 \textbf{Parameter:}\\ \ \\
 \textit{Data:}\\
 Data muss ein Array sein, das eine Nachricht enth�lt.
 Die Nachricht muss folgendem Bereich von Data entsprechen:\\
\texttt{Data(Data'First)..Data(Data'First+(Message\_Length-1))}\\ \ \\
\textit{Message\_Length:}\\
Die L�nge der Nachricht die Data enth�lt.\\ \ \\
\textit{Data2:}\\
Wird f�r Sonderf�lle ben�tigt. Im Normalfall gilt nach dem 
Prozeduraufruf Is\_Zero(Data2) = true.\\ 
\ \\
\textbf{Vorbedingungen:}\\
\begin{enumerate}
 \item Data'Length = Data2'Length
 \item Message\_Length $\le$ Data'Length
 \item Word(Data'Length) - Message\_Length $>$ 2**16-1\footnotemark 
 \end{enumerate}\ \\
 \textbf{Nachbedingung:}
 \begin{enumerate}
 \item Message\_Length $<$ Data'Length-2 $\Rightarrow$ Is\_Zero(Data2) = True
 \item Is\_Zero(Data2) = False  $\Longleftrightarrow$ Message'Length $>$ 
   Data'Length-2
 \end{enumerate}\ \\
 \textbf{Exceptions:}\\
 \addtocounter{footnote}{-1} 
\begin{tabular}{l@{ : }l}
 Constraint\_Message\_Length\_Error &  Message\_Length $>$ Data'Length\\
 Constraint\_Length\_Error &  Data'Length $>$ (2 * Byte'Last)
 \footnote{Gilt nur wenn Data vom Type Byte ist}\\
 \end{tabular}\ \\ \ \\
 Diese Funktionen f�gt einer Nachricht Null-Bytes/Words/Dwords an gefolgt
 von einem Byte/Word/DWord das der Anzahl der eingef�gten Null-Bytes
 entspricht. Dabei k�nnen folgende Sonderf�lle auftreten:
 \begin{enumerate}
 \item Message\_Length+1=Data'Length\\
   Nachdem ein Null-Byte an die Nachricht angef�gt wurde ist kein Platz mehr 
   vorhanden um die Anzahl der eingef�gten Null-Bytes/Words/DWords an die
   Nachricht anzuh�ngen. Data(Data'Last) wurde ja mit einem 
   Null-Byte/Word/DWord aufgef�llt. Um dieses Problem zu l�sen werden 
   alle Felder von Data2 au�er Data2'Length auf 0 gesetzt. Data2'Last wird 
   auf den Wert Data2'Length gesetzt.
 \item Message\_Length=Data'Length\\
   Analog zu dem ersten Sonderfall bis auf folgende Abweichungen:
   \begin{enumerate}
   \item Es wird kein Null-Byte/Word/DWord an die Nachricht angeh�ngt  
   \item Data2'Last := Data2'Length-1
   \end{enumerate}
 \end{enumerate}\ \\

\subsubsection{Beispiel}
\begin{lstlisting}[frame=brtl]{}
  ... 
  package BIO is new Ada.Text_Io.Modular_IO (Byte);
 
  X : Bytes(1..9) := (1 => 2, 2 => 4, 3 => 6, 
                      4 => 8, 5 => 9, others =>1);
  Y : Bytes(1..9);
begin
  Padding(X, N, Y);
  Put("X:");
  for I in X'Range loop
     BIO.Put(X(I));
  end loop;
  New_Line;
  if Is_Zero(Y) = False  then
      Put("Y:");
      for I in Y.all'Range loop
	 BIO.Put(Y.all(I));
      end loop;
   end if;
...
\end{lstlisting}\ \\
\underline{Ausgabe:}\\
\begin{itemize}
\item N = 5\\
  X : 2   4   6  8  9   0   0   0   3
\item N = 8\\
  X : 2   4   6   8   9   1   1   1   0\\
  Y : 0   0   0   0   0   0   0   0   9
\item N = 9\\
  X : 2   4   6   8   9   1   1   1   1\\
  Y : 0   0   0   0   0   0   0   0   8
\end{itemize}





  \chapter{Crypto.Random}
\section{Einf�hrung}
Dieses Paket greift defaultm��ig auf ``dev/random/'' zu.
Unter Linux ist \\``/dev/random'' ein Pseudodevice das kryptographisch sichere
Pseudozufallsbits liefert. Wenn sie ein Betriebsystem verwenden das kein 
solches Pseudodevice besitzt, dann k�nnen sie dies mit Hilfe von mit 
Brian Warners ``Entropy Gathering Daemon''(EGD), der unter
\textit{http://www.lothar.com/tech/crypto/} zu finden 
ist,  nachinstallieren. Der EGD ist ein Userspace-Implementierung des 
Linux-Kernel-Devices ``dev/random/'', das kryptograpisch sichere
Zuallsbits generiert.\\
Im privaten Teil befindet sich die Variable ``Path''. Diese Variable gibt den
Pfad zu einer Datei, Named Pipe oder Ger�t das krytograpisch sichere
Zufallsbits enth�lt an.  Dadurch sind Sie in der Lage die ACL auch an eine
kryptograpisch sichere (Pseudo-)Zufallsbitquelle eines nicht POSIX kompatiblen
Betriebsystems wie Windows XP anbinden.\\
Weiterhin besteht die M�glichkeit unter Windows den Marsaglia- RNG zu benutzen.
Er ist ein lagged- Fibonacci Sequenz Generator und erzeugt 24-Bit real numbers
im Intervall [0,1].\\
Eigenschaften: Kombination zweier unterschiedlicher Generatoren, welche fort-
laufende Bits erzeugen. Diese werden durch zwei weitere Generatoren zu einer 
Tabelle Kombiniert.
Die Prozedur ''Start'' initialisiert diese Tabelle, welche von der Funktion 
''Next'' genutzt wird um die n�chste Zufallszahl zu generieren.

\section{API}
\subsection{Prozeduren}
\begin{tabular}{p{\textwidth}}
  \begin{lstlisting}{}
    procedure Read(B : out Byte);
    procedure Read(W : out Word);
    procedure Read(D : out DWord);
  \end{lstlisting}
  Bei diesen drei Prozeduren wird B, W oder D mit Bits aus der
  verwendeten Pseudozufallsbit-Quelle gef�llt.\\ \ \\
  \hline\\
\end{tabular}

\begin{tabular}{p{\textwidth}}
  \begin{lstlisting}{}
    procedure Read(Byte_Array  : out Bytes);
    procedure Read(Word_Array  : out Words);
    procedure Read(DWord_Array : out DWords);
  \end{lstlisting}\\
  Diese drei Prozeduren wirden die �bergebenen Arrays mit Bits aus der 
  verwendeten Pseudozufallsbit-Quelle gef�llt. \\ \ \\
\end{tabular}

\begin{tabular}{p{\textwidth}}
  \begin{lstlisting}{}
    procedure Start(
    New_I : Seed_Range_1 := Default_I;
    New_J : Seed_Range_1 := Default_J;
    New_K : Seed_Range_1 := Default_K;
    New_L : Seed_Range_2 := Default_L);
  \end{lstlisting}\\
  \\ \ \\
\end{tabular}

\subsection{Exceptions}
\begin{tabular}{p{\textwidth}}
  \begin{lstlisting}{}
    Random_Source_Does_Not_Exist_Error : exception;
  \end{lstlisting}\\
  Diese Ausnahme wird aufgerufen, wenn kein (kryptograpisch sicherer) 
  Pseudozufallsbit-Quelle gefunden wurde.\\ \ \\
  \hline\\
  \begin{lstlisting}{}
    Random_Source_Read_Error : exception;
  \end{lstlisting}\\
  Wenn ein Lesefehler bei einem Zugriff auf die Pseudozufallsbit-Quelle 
  auftritt, wird die folgende Aufnahme ausgeworfen. \\ \ \\
  \hline\\
\end{tabular}

\begin{lstlisting}{}
\end{lstlisting}

   \chapter{Crypto.Symmetric}

\subsubsection{Description}
This is the root package for symmetric cryptography.\\

\subsubsection{Function}
This package provides the symmetric part of the ACL with direct access to
Crypto.Types, which contains the fundamental types and corresponding base
functions.
 
\section{Logical Setup}
A block cipher is separated into three logical layers. An application programmer
should solely use the API of the upmost layer. Only with this API, a secure
application of a block cipher can be guaranteed.

\subsubsection{Layer 1: Algorithm}
This layer provides the pure algorithm of a symmetric cipher. The API of
the different algorithms is identical and provides as an interface for the
block cipher layer.
With each algorithm, a reference to the corresponding source (e.g. a paper)
is provided. As a result, it is possible to verify the implementation
by consulting this reference. Generally, this is always a good thing to do.
Additional information regarding the particular algorithms and ciphers can be
found within the next chapter. \\ \ \\
\textbf{Comment:}\\
The API of this layer should only be used for generating a block cipher.
It should by no means be used for any other purpose. There is no scenario
where this would make any sense. This API only serves as an interface for
layer 2.

\subsubsection{Layer 2: Block Cypher}
At this layer, a block cipher is generated from the algorithm in the
insecure ECB-mode (Electronic Code Book). You should only use this API when
you are familiar with symmetric block ciphers. Actually, this API serves as
an interface for layer 3.
Chapter \ref{block} explains how a block cipher is
generated from an algorithm.

\subsubsection{Layer 3: Mode}
This layer transforms a block cipher into a reasonable mode. There are different
modes for different use cases, each of them having there particular assets and
drawbacks. Chapter \ref{modus} provides a detailed description.

\begin{figure}
  \begin{center}
    \huge
    \begin{tabular}{|c @{\ } c|}\hline
      III. & Mode\\
      \hline
      II. & Block Cypher\\
      \hline
      I. & Algorithm\\
    \hline
    \end{tabular}
  \end{center}
\caption{The three-layered architecture of a symmetric (block) cipher}
\end{figure}

\subsubsection{Comment}
By using this three-layered architecture, it is also possible to implement
stream ciphers. In order to do this, you have to provide the first layer with
a procedure for receiving the next n bits of the key stream. Afterwards, you
can construct an oneway block cipher (Chapter \ref{oneblock}) and and transform
it to the oneway-OFB or oneway-counter mode. The ACL currently does not provide
any stream cipher. However, there are plans for implementing the stream cipher
Helix \cite{helix}.

   \chapter{Crypto.Symmetric}

\subsubsection{Description}
This is the root package for symmetric cryptography.\\

\subsubsection{Function}
This package provides the symmetric part of the ACL with direct access to
Crypto.Types, which contains the fundamental types and corresponding base
functions.
 
\section{Logical Setup}
A block cipher is separated into three logical layers. An application programmer
should solely use the API of the upmost layer. Only with this API, a secure
application of a block cipher can be guaranteed.

\subsubsection{Layer 1: Algorithm}
This layer provides the pure algorithm of a symmetric cipher. The API of
the different algorithms is identical and provides as an interface for the
block cipher layer.
With each algorithm, a reference to the corresponding source (e.g. a paper)
is provided. As a result, it is possible to verify the implementation
by consulting this reference. Generally, this is always a good thing to do.
Additional information regarding the particular algorithms and ciphers can be
found within the next chapter. \\ \ \\
\textbf{Comment:}\\
The API of this layer should only be used for generating a block cipher.
It should by no means be used for any other purpose. There is no scenario
where this would make any sense. This API only serves as an interface for
layer 2.

\subsubsection{Layer 2: Block Cypher}
At this layer, a block cipher is generated from the algorithm in the
insecure ECB-mode (Electronic Code Book). You should only use this API when
you are familiar with symmetric block ciphers. Actually, this API serves as
an interface for layer 3.
Chapter \ref{block} explains how a block cipher is
generated from an algorithm.

\subsubsection{Layer 3: Mode}
This layer transforms a block cipher into a reasonable mode. There are different
modes for different use cases, each of them having there particular assets and
drawbacks. Chapter \ref{modus} provides a detailed description.

\begin{figure}
  \begin{center}
    \huge
    \begin{tabular}{|c @{\ } c|}\hline
      III. & Mode\\
      \hline
      II. & Block Cypher\\
      \hline
      I. & Algorithm\\
    \hline
    \end{tabular}
  \end{center}
\caption{The three-layered architecture of a symmetric (block) cipher}
\end{figure}

\subsubsection{Comment}
By using this three-layered architecture, it is also possible to implement
stream ciphers. In order to do this, you have to provide the first layer with
a procedure for receiving the next n bits of the key stream. Afterwards, you
can construct an oneway block cipher (Chapter \ref{oneblock}) and and transform
it to the oneway-OFB or oneway-counter mode. The ACL currently does not provide
any stream cipher. However, there are plans for implementing the stream cipher
Helix \cite{helix}.
 
  \include{acl.crypto.symmetric.algorithm.oneway}
  \chapter{Crypto.Symmetric.Blockcipher}\label{block}
Mit Hilfe dieses generischen Paketes l�sst sich aus dem Algorithmus einer
symmetrischen Chiffre (siehe Kapitel \ref{algobc}) eine Blockchiffre erstellen.
Sie sollten davon Abstand nehmen die API dieses Paketes direkt zu 
verwenden, denn dieses Paket implementiert eine Blockchiffre im ``unsicheren''
ECB-Modus (Electronic Codebook Modus). 
Wenn man in diesem Modus zwei identische Klartextbl�cke  p1=p2 mit dem selben
Schl�ssel chiffriert, dann erh�lt man zwei identische Chiffretextbl�cke  c1=c2.
Dadurch kann der Chiffretext noch Informationen �ber die Struktur des 
Klartextes enthalten. 

\section{API}

Die API eine Blockchiffre besteht aus drei Prozeduren.

\begin{enumerate}
\item Die Prozedur \textbf{Prepare\_Key} weist einer Blockchiffre
  einen Schl�ssel \textit{Key} zu.
  \begin{lstlisting}{}
 procedure Prepare_Key(Key : in Key_Type);
  \end{lstlisting}
  
\item Die Prozedur \textbf{Encrypt} verschl�sselt (mit Hilfe eines vorher 
  zugewiesenen Schl�ssels) einen Klartextblock  (Plaintext) in einen 
  Chiffretextblock (Ciphertext) 
  \begin{lstlisting}{}
 procedure Encrypt(Plaintext : in Block; Ciphertext : out Block);
  \end{lstlisting}

\item Die Prozedur \textbf{Decrypt} entschl�sselt (mit Hilfe eines vorher
  zugewiesenen Schl�ssels) einen Chiffretextblock (Ciphertext) in einen 
  Klartextblock (Plaintext).
  \begin{lstlisting}{}
 procedure Decrypt(Ciphertext : in Block; Plaintext  : out Block);
  \end{lstlisting}

\end{enumerate}

\section{Generischer Teil}

 \begin{lstlisting}{}
 type Block is private;
 type Key_Type is private;
 type Cipherkey_Type is private;

 with procedure Prepare_Key(Key       : in  Key_Type;
                            Cipherkey : out Cipherkey_Type);

 with procedure Encrypt(Cipherkey  : in  Cipherkey_Type;
                        Plaintext  : in  Block;
                        Ciphertext : out Block);

 with procedure Decrypt(Cipherkey  : in  Cipherkey_Type;
                        Ciphertext : in  Block;
                        Plaintext  : out Block);
 \end{lstlisting}

\section{Anwendungsbeispiel}
\begin{lstlisting}{generic TDES}
with Crypto.Types;
with Crypto.Symmetric.Blockcipher;
with Crypto.Symmetric.Algorithm.TripleDES;

procedure Generic_TripleDES is
use Crypto.Types;
use Crypto.Symmetric.Algorithm.TripleDES;

   package Generic_TripleDES is 
      new  Crypto.Symmetric.Blockcipher
      (Block          => B_Block64,
       Key_Type       => B_Block192,
       Cipherkey_Type => Cipherkey_TDES,
       Prepare_Key    => Prepare_Key,
       Encrypt        => Encrypt,
       Decrypt        => Decrypt);
begin
   ...
end Generic_TripleDES;
\end{lstlisting}

\section{Anmerkung}
Sie m�ssen nicht jedes mal von neuem aus einem symmetrischen Algorithmus eine
Blockchiffre generiern. Statdessen k�nnen Sie auch eine der folgenden 
vorgefertigten  Blockchiffren verwenden.
\begin{itemize}
\item Crypto.Symmetric.Blockcipher\_AES128
\item Crypto.Symmetric.Blockcipher\_AES192
\item Crypto.Symmetric.Blockcipher\_AES256
\item Crypto.Symmetric.Blockcipher\_Serpent256
\item Crypto.Symmetric.Blockcipher\_Tripledes
\item Crypto.Symmetric.Blockcipher\_Twofish128
\item Crypto.Symmetric.Blockcipher\_Twofish192
\item Crypto.Symmetric.Blockcipher\_Twofish256
\end{itemize}




  \chapter{Crypto.Symmetric.Oneway\_Blockcipher}\label{oneblock}

Mit Hilfe dieses generischen Paketes k�nnen Sie aus einem symmetrischen 
Einweg-Algorithmus (\textbf{Kapitel} \ref{onealg}) oder einer Hashfunktion 
(\textbf{Kapitel} Kapitel \ref{hash}) eine Einweg-Blockchiffre
(oneway-Blockcipher) generieren.
Sie sollten aber davon Abstand nehmen die API dieses Paketes direkt zu 
verwenden. Diese Paket implementiert n�mlich eine Blockchiffre ``unsicheren''
im ECB-Modus (Electronic Codebook Modus).  In diesem Modus werden zwei
identische Klartextbl�cke  p1=p2, die mit dem selben Schl�ssel chiffriert
werden, zu zwei identischen Chiffretextbl�cken c1,c2 mit c1=c2. 
Der Chiffretext kann also immer noch Informationen �ber die Struktur des 
Klartextes enthalten. \\
Bei Einweg-Blockchiffre die auf Hashfunktionen beruhen wurde die Schl�ssell�nge
(mind. 256 Bit) so gew�hlt, dass es unm�glich ist den kompletten
Schl�sselraum abzusuchen. 
Ein (irreversibler) 219-Bit Z�hler der all seine Zust�nde durchl�uft ben�tigt 
min. $10^{51} erg.$. Dies entspricht der Energie die eine typische Supernova
abgibt. Damit ein 256-Bit Z�hler alle Zust�nde durchl�uft ben�tigt er min. die
Energie von 64 Milliarden Supernoven. Der Energieausto� unserer Sonne betr�gt
im Jahr gerade mal $1,21 \cdot 10^{41}$ erg. \cite{schneier}\\
Damit ist eine solche Einweg-Blockchiffre so (un-)sicher wie die 
zugrundeliegende Hashfunktionen.

\section{API}

Die API eine Blockchiffre besteht den folgenden beiden Prozeduren.

\begin{enumerate}
\item Die Prozedur \textbf{Prepare\_Key} weisen sie einer Blockchiffre
  einen Schl�ssel \textit{Key} zu.
  \begin{lstlisting}{}
 procedure Prepare_Key(Key  : in Key_Type);
  \end{lstlisting}
  
\item Die Prozedur \textbf{Encrypt} verschl�sselt (mit Hilfe eines vorher 
  zugewiesenen Schl�ssels) einen Klartextblock  (Plaintext) in einen 
  Ciffretextblock (Ciphertext) 
  \begin{lstlisting}{}
 procedure Encrypt(Plaintext : in Block; Ciphertext : out Block);
  \end{lstlisting}

\end{enumerate}


\section{Generischer Teil}

 \begin{lstlisting}{}
generic
   type Block is private;
   type Key_Type is private;
   type Cipherkey_Type is private;

   with procedure Prepare_Key(Key : in Key_Type;
                              Cipherkey : out Cipherkey_Type);

   with procedure Encrypt(Cipherkey  : in Cipherkey_Type;
                          Plaintext  : in Block;
                          Ciphertext : out Block);
    
 \end{lstlisting}

\section{Anwendungsbeispiel}
\begin{lstlisting}{generic onewayTDES}
with Crypto.Types;
with Crypto.Symmetric.Oneway_Blockcipher;
with Crypto.Symmetric.Algorithm.TripleDES.Oneway;

procedure Generic_Oneway_TripleDES is
use Crypto.Types;
use Crypto.Symmetric.Algorithm.TripleDES.Oneway;

   package Oneway_TripleDES is 
      new  Crypto.Symmetric.Oneway_Blockcipher
      (Block           => B_Block64,
       Key_Type        => B_Block192,
       Cipherkey_Type  => Cipherkey_Oneway_TDES,
       Prepare_Key     => Prepare_Oneway_Key,
       Encrypt         => Encrypt_Oneway);
                                  
begin
   ...
end Generic_Oneway_TripleDES;
\end{lstlisting}


\section{Anmerkung}
Sie m�ssen nicht jedes mal auf neue aus einem Algorithmus eine 
Einweg-Blockchiffre generieren. Stattdessen k�nnen Sie auch eine der folgenden 
vorgefertigten Blockchiffren verwenden.
\begin{itemize}
\item Crypto.Symmetric.Oneway\_Blockcipher\_AES128
\item Crypto.Symmetric.Oneway\_Blockcipher\_AES192
\item Crypto.Symmetric.Oneway\_Blockcipher\_AES256
\item Crypto.Symmetric.Oneway\_Blockcipher\_SHA1
\item Crypto.Symmetric.Oneway\_Blockcipher\_SHA256
\item Crypto.Symmetric.Oneway\_Blockcipher\_SHA512
\item Crypto.Symmetric.Oneway\_Blockcipher\_Serpent256
\item Crypto.Symmetric.Oneway\_Blockcipher\_Tripledes
\item Crypto.Symmetric.Oneway\_Blockcipher\_Twofish128
\item Crypto.Symmetric.Oneway\_Blockcipher\_Twofish192
\item Crypto.Symmetric.Oneway\_Blockcipher\_Twofish256
\item Crypto.Symmetric.Oneway\_Blockcipher\_Whirlpool
\end{itemize}

  \chapter{Crypto.Symmetric.Mode}\label{modus}

Mit Hilfe von generischen Betriebsmodi ist es m�glich eine Blockchiffre in 
einem sicheren Betriebsmodus zu betreiben. Ein Betriebsmodus verkn�pft
f�r gew�hnlich eine Blockchiffre mit einer R�ckkopplung und einigen einfachen
Operationen (+, xor), und wird mit Hilfe eines zuf�lligen �ffentlichen 
Startwertes (Initial Value (IV)) initialisiert. Der Chiffretext ist damit
nicht  nur von dem verwendeten Chiffre, Klartext und Schl�ssel abh�ngig,
sondern auch von dem zuf�lligen Startwert. 
Wenn Sie einen Klartext mehrmals mit der gleichen Chiffre und dem gleichen
Schl�ssel aber unterschiedlichen IVs verschl�sselt, erhalten Sie
unterschiedliche Chiffretexte. Durch die R�ckkopplung werden gleiche
Klartextbl�cke zu unterschiedlichen Chiffretextbl�cken chiffriert, d.h ein 
Betriebsmodus verschl�sselt zwei Klartextbl�cke p1 und p2 mit p1=p2, mit 
�berw�ltigender Wahrscheinlichkeit zu zwei Chiffretextbl�cke c1 und c2 mit 
c1$\not=$c2. Damit ist es nun m�glich mehrere Nachrichten sicher
mit dem selben Schl�ssel zu verschl�sseln.\\
\textbf{Vorsicht: Um einen Chiffretext zu entschl�sseln ben�tigt Sie den 
gleichen Schl�ssel und Startwert wie bei der Verschl�sselung.}
Aus diesem Grund sollte Sie den Startwert immer mit dem zugeh�rigen Chiffretext
aufbewahren. \textbf{Die Sicherheit eines Modus ist unabh�ngig vom 
\glqq Bekanntheitsgrad\grqq des Startwertes}.  Daher ist es �blich, dass man
den Startwert mit dem Chiffretext multipliziert ($IV*C = C'= IV||C$),
d.h. man h�ngt den Chiffretext einfach an den Startwert an um das 
finale Chiffrat C' zu erhalten.\\
In diesem Kapitel wird auf die einzelnen Modi ihre API, Einsatzweck und 
Merkmale eingegangen.

%%%%%%%%%%%%%%%%%%%%%%%%%%%%%%%%%%%%%%%%%%%%%%%%%%%%%%%%%%%%%%%%%%%%%%%%%%%

\section{API}\label{mode-api}

Die API eine Blockchiffre besteht aus den folgenden 4 Prozeduren.

\begin{enumerate}
\item Die Prozedur \textbf{Init} initialisiert einen Blockchiffre und Modus 
indem sie der Blockchiffre einen Schl�ssel \textit{Key} und dem Modus einen
 Startwert  \textit{Initial\_Value} zuweist.
  \begin{lstlisting}{}
    procedure Init(Key           : in Key_Type;
                   Initial_Value : in Block);
  \end{lstlisting}

\item Die Prozedur \textbf{Encrypt} verschl�sselt einen Klartextblock 
  (Plaintext) in einen Chiffretextblock (Ciphertext) 
  \begin{lstlisting}{}
    procedure Encrypt(Plaintext  : in Block; 
                      Ciphertext : out Block);
  \end{lstlisting}

\item Die Prozedur \textbf{Decrypt} entschl�sselt einen Chiffretextblock 
  (Ciphertext) in einen Klartextblock (Plaintext).
  \begin{lstlisting}{}
    procedure Decrypt(Ciphertext : in Block;
                      Plaintext  : out Block);
  \end{lstlisting}

\item Die Prozedur \textbf{Set\_IV} weist dem Modus den Startwert 
  \textit{Initial\_Value} zu. D.h. mit Hilfe dieser Prozedur l�sst sich eine
  Modus reinitialisieren. Nach jeder ver- bzw. entschl�sselten Nachricht 
  (Eine Nachricht besteht aus n-Klartextbl�cken,dies entspricht n-Aufrufe der 
  \textit{Encrypt} bzw. \textit{Decrypt}-Prozedur) muss eine Chiffre 
  reinitialisiert werden. Au�erdem muss eine Chiffre jedes mal reinitialisieren
  werden wenn man den Richtung (Verschl�sselung, Entschl�sselung) des Modus
  �ndert
 \begin{lstlisting}{}
procedure Set_IV(Initial_Value : in Block);
 \end{lstlisting}
\end{enumerate}

%%%%%%%%%%%%%%%%%%%%%%%%%%%%%%%%%%%%%%%%%%%%%%%%%%%%%%%%%%%%%%%%%%%%%%%%%%%

\section{Cipher-Block-Chaining-Modus (CBC)}
\subsubsection{Paket: Crypto.Symmetric.Mode.CBC}
Bei diesem Betriebsmodus flie�t das Ergebnis der Verschl�sselung fr�herer 
Bl�cke in  die Verschl�sselung des aktuellen Blockes mit ein, 
indem der Klartext vor der Verschl�sselung mit dem vorherigen
(zwischengespeicherten) Chiffretextblock mit der XOR-Operation verkn�pft wird. 

\subsection{Verschl�sselung}
Bei der Initialisierung wird der Startwert IV als $C_0$ zwischengespeichert. 
Wird nun ein Klartextblock $P_i$ verschl�sselt, so wird dieser zuerst 
mit $C_0$ mittels der XOR-Operation verkn�pft und danach mit Hilfe der 
Blockchiffre zu dem Chiffretextblock $C_1$ verschl�sselt und 
zwischengespeichert. Der n�chste Klartextblock $P_2$ wird zuerst mit $C_1$ 
mittels der XOR-Operation verkn�pft und danach zu $C_2$ verschl�sselt und 
zwischengespeichert usw.\\ \ \\
Mathematische Beschreibung: $C_i = E_K(P\oplus C_{i-1})$ 


\subsection{Entschl�sselung}
Die Entschl�sselung verl�uft analog zur Verschl�sselung. Zu Beginn wird der
Betriebsmodus mit dem Startwert $C_0$ initialisiert bzw. mittels 
\textit{Set\_IV} reinitialisiert. Der Chiffretextblock $C_1$ wird nun ganz
normal entschl�sselt und das Ergebnis der Entschl�sselung dann mit $C_0$
mittels der XOR-Operation verkn�pft. Das Ergebnis dieser XOR-Operation
ist der Klartextblock $P_i$. Als n�chstes wird der Chiffretextblock $C_2$
entschl�sselt und das Ergebnis dieser Entschl�sselung wird dann mit $C_1$,
mittels der XOR-Operation, zu dem Klartextblock $P_2$, verkn�pft usw. \\ \ \\

Mathematische Beschreibung: $P_i = C_{i-1} \oplus D_K(C_i)$ 

\subsection{Verwendungszweck}
\begin{itemize}
\item \textbf{Verschl�sselung von Dateien}\\
  Da mit diesem Modus keine Synchronisationsfehler behoben werden k�nnen, 
  eignet er sich besonders f�r die Verschl�sselung von Dateien.
  Dabei kommt es praktisch nie zu synchronisations Fehlern, daf�r aber 
  gelegentlich zu Bitfehler (durch defekte Hardware o.�.). Ein Bitfehler in 
  einem Chiffretextblock $C_i$ betrifft den kompletten Klartextblock $P_i$ 
  sowie das  entsprechenden Bit im n�chsten Klartextblock $P_{i+1}$.\\
\item \textbf{Integrit�tspr�fung von Nachrichten}\\
  Um die Integrit�t einer Nachricht M zu �berpr�fen verschl�sseln sie diese und
  merken sich nur die beiden Chiffretextbl�cke $C_0=IV$ und $C_n$. 
  Die restlichen Chiffretextbl�cke werden nicht ben�tigt. Jetzt sind Sie in 
  der Lage jederzeit festzustellen ob die Nachricht manipuliert wurde,
  indem Sie die Nachricht  M' nochmals  mit dem Startwert $C_0$ verschl�sseln 
  $C_n'$ berechnen und �berpr�ft  ob $C_n=C_n'$ gilt. Ist dies der Fall, dann 
  wurde die M nicht manipuliert.
  Wenn $C_n\not=C_n'$ gilt, dann wurde entweder $IV$, $C_n$ oder M manipuliert.
  Unter manipuliert versteht man hier das zuf�llige Kippen eines oder mehrer 
  Bits.
\item \textbf{Authentifizierung}\\
  Angenommen sie teilen mit Alice, die Sie noch nie zuvor getroffen haben, 
  ein Geheimnis  (Key). Eines Tages wollen Sie sich mit Alice treffen um mit 
  ihr vertrauliche Daten  auszutauschen. Um sicher zugehen das sich es bei der
  Person am vereinbarten Treffpunkt wirklich um Alice handelt nehmen Sie 
  einfach eine Nachricht M und einen Zuf�lligen Startwert IV mit. Bitten Sie 
  ``Alice'' M mit ihrem Geheimnis und dem Startwert IV $C_n$ zu  berechnen.
  Wenn dieser Wert mit dem von ihnen berechneten $C_n$  �bereinstimmen 
  (und weder Alice noch Sie ihr  Geheimnis jemand anderem verraten hat), 
  dann handelt es sich bei der Person mit �berw�ltigender Wahrscheinlichkeit 
  um Alice. Wenn die Werte nicht �bereinstimmen, dann handelt es sich bei 
  dieser Person nicht um Alice.\\
  \textbf{VORSICHT: Der CBC-MAC ist nur sicher, wenn die alle ausgetauschten
    Nachrichten gleich lang sind.} Aus diesem Grund sollte man davon Abstand
    nehmen  den CBC-MAC zu verwendet.     
\end{itemize}

%%%%%%%%%%%%%%%%%%%%%%%%%%%%%%%%%%%%%%%%%%%%%%%%%%%%%%%%%%%%%%%%%%%%%%%%%%%
  
\section{Cipher-Feedback-Modus (CFB)}\label{CFB}
\subsubsection{Paket: Crypto.Symmetric.Mode.CFB}
Der CFB-Modus wandelt eine Blockchiffre in eine selbst synchronisierte
Blockchiffre um. Z.B. versetzt dies ein Terminal in die Lage, jedes 
eingetippte Zeichen sofort zum Host zu �bertragen. \\
\textbf{Achtung: Bei diesem Modus muss der Startwert nach jeder Nachricht
ge�ndert werden.} Ansonsten ist dieses Verfahren nicht sicher.

\subsection{Verschl�sselung}
Bei der Initialisierung wird der Startwert IV als $C_0$ zwischengespeichert.
Wenn nun n-Bytes (n $<$ Block'Size) verschl�sselt werden soll, dann werden die 
n-Bytes an den Anfang des Klartextblockes $P_0$ kopiert. Dieser wird dann mit
Nullen aufgef�llt. Bei der Verschl�sselung von $P_1$ wird $C_0$ 
verschl�sselt und der dadurch generierte Chiffretext $C_1$ wird mittels
der XOR-Operation mit $P_1$ verkn�pft. Das Ergebnis dieser Operation wird
ausgegeben und als $C_1$ zwischengespeichert. Der n�chste aufgef�llte 
Klartextblock $P_2$ wird nach der Verschl�sselung zu $C_2$ indem $C_1$
verschl�sselt und mit Hilfe der XOR-Operation mit $P_2$ verkn�pft wird. Dies 
wird solange wiederholt bis die gesamte Nachricht �bermittelt wurde.\\ \ \\
Mathematische Beschreibung: $C_i = P_i \oplus E_K(C_{i-1})$    

\subsection{Entschl�sselung}
Die Entschl�sselung verl�uft analog zur Verschl�sselung. Zu Beginn wird der
Betriebsmodus mit dem Startwert IV initialisiert bzw. mittels 
\textit{Set\_IV} reinitialisiert und als $C_0$ zwischengespeichert.
Bei der Entschl�sselung eines Chiffretextblockes $C_i$ wird zun�chst $C_{i-1}$
entschl�sselt und  mit Hilfe der XOR-Operation mit $C_i$ verkn�pft. Das
Ergebnis dieser Operation ist $P_i$. Zum Schluss wird noch $C_i$ 
zwischengespeichert.\\ \ \\
Mathematische Beschreibung: $P_i = C_{i} \oplus E_K(C_{i-1})$ 

\subsection{Verwendungszweck}  
Im Gegensatz zum CBC-Modus wo die Verschl�sselung erst dann
beginnen kann, wenn ein vollst�ndiger Datenblock vorliegt, k�nnen im 
CFB-Modus Daten z.B. auch Byteweise (8-CFB) verschl�sselt werden. Dadurch 
eignet sich dieses Verfahren
hervorragend f�r die \textbf{Verschl�sselung von Bytestr�me (z.B. Remoteshell)}

\subsection{Anmerkungen}  
Beim n-CFB-Modus
\begin{itemize}
\item wirkt sich ein Fehler im Klartext auf den gesamten nachfolgenden
  Chiffretext aus und macht bei der Entschl�sselung selbst wieder 
  r�ckg�ngig.
\item wirkt sich ein Fehler im Chiffretext $C_i$ auf den  
  Klartextblock $P_i$ und die folgenden  $\frac{m}{n}-1$ Klartextbl�cke aus,
  wobei m die Blockgr��e ist. 
\item Ein Angreifer kann die Nachrichtenbits im letzten Chiffretextblock 
  ver�ndern ohne dabei entdeckt zu werden.
\end{itemize}

%%%%%%%%%%%%%%%%%%%%%%%%%%%%%%%%%%%%%%%%%%%%%%%%%%%%%%%%%%%%%%%%%%%%%%%%%%%

\section{Couter-Modus (CTR)}\label{CTR}
\subsubsection{Paket: Crypto.Symmetric.Mode.CTR}
Beim Counter-Modus wird die die Blockchiffre in einen Schl�sselstromgenerator
umgewandelt. Die R�ckkopplung h�ngt also nicht vom Klartext ab sondern von 
einem Z�hler, der nach jeder Verschl�sselungsoperation um eins erh�ht wird.


\subsection{Verschl�sselung}
Bei der Initialisierung wird der Z�hler auf den Startwert (IV) gesetzt. 
Bei der Verschl�sselung des Klartextblockes $P_i$ wird $IV+i-1$ mit Hilfe der
Blockchiffre zu dem Schl�sselstromblock $K_i$ verschl�sselt.
Danach wird mit Hilfe der
XOR-Operation $K_i$ mit $P_i$ verkn�pft. Das Ergebnis dieser Verkn�pfung ist 
der Chiffretext $C_i$ .\\ \ \\
Mathematische Beschreibung: $C_i = P_{i} \oplus E_K(IV+i-1)$   

\subsection{Entschl�sselung}
Die Entschl�sselung verl�uft analog zur Verschl�sselung. Zu Beginn wird der
Z�hler mit dem Startwert IV initialisiert bzw. mittels \textit{Set\_IV} 
reinitialisiert. Bei der Entschl�sselung eines Chiffretextblockes $C_i$ wird
zun�chst $IV+i-1$ mittels der Blockchiffre zu dem Schl�sselstromblock $K_i$
verschl�sselt. Danach wird $C_i$ mit Hilfe der XOR-Operation zu mit $K_i$
verkn�pft. Das Ergebnis dieser Operation ist $P_i$ .\\ \ \\
Mathematische Beschreibung: $P_i = C_{i} \oplus E_K(IV+i-1)$   

\subsection{Verwendungszweck}  
\begin{itemize}
\item  \textbf{Ent/Verschl�sselung von Nachrichten mit wahlfreiem Zugriff}
  Da Sie mit dem Counter-Modus in der Lage sind gezielt 
  einzelne Chiffretextblock zu entschl�sseln eignet sich dieses Verfahren 
  f�r die Verschl�sselung von Dateien mit wahlfreiem Zugriff wie z.B. 
  Datenbanken. Hier k�nnen   sie bei einer Anfrage an eine verschl�sselte
  Datenbank genau die Daten, die angefragt wurden.  
\item \textbf{Parallele Ent/Verschl�sselung}\\
  Eine Parallelisierung ist m�glich indem man aus dem Startwert des Z�hlers 
  IV und  der L�nge der Nachricht L folgendes Intervall berechnet: 
  $[IV...IV+L]$
  Das Intervall l�sst sich in max. L disjunkte Teilintervalle zerlegen. 
  Die Nachrichtenbl�cke der Teilintervalle k�nnen parallel ver- bzw. 
  entschl�sselt werden.
\item \textbf{Phasenweise ``High Speed''-Verschl�sselung}\\
  Die ist eine Expertenanwendung die auf der ``Low-Level-API'' (\ref{ctrllapi})
  des CTR-Modus beruht. Verwenden sie diese nur wenn Sie genau wissen, 
  was Sie tun.\\
  Beim Counter-Modus ist es m�glich beliebig viele Schl�sselstrombits, ohne
  das ein Nachrichtenblock ben�tigt wird, zu generiere. Wenn Sie mit 
  Hilfe des Counter-Modus gen�gend Schl�sselstrombits generieren, 
  dann sind Sie in  der Lage  Nachrichten sehr schnell zu verschl�sseln,
  indem Sie sie einfach mit den vorher erzeugten Schl�sselstrombits 
  XOR-Verkn�pft. 
\end{itemize}


\subsection{Anmerkungen}
\begin{itemize}
\item Ein Bit-Fehler im Klartext wirkt sich nur auf ein Bit im Chiffretext aus 
  und umgekehrt.
\item Manipulationen am Klartext sind sehr einfach, da jede �nderung des 
  Chiffretextes beeinflusst direkt den Klartext.
\item Synchronisationsfehler (Alice und Bob haben unterschiedliche 
  Counterst�nde) k�nnen nicht behoben werden.
\end{itemize}

\subsection{Low-Level-API}\label{ctrllapi}
Die folgende API sollten Sie nur dann verwenden, wenn Sie genau wissen was Sie
tun.\\
\begin{tabular}{p{\textwidth}}
\begin{lstlisting}{}
    procedure Next_Block(Keystream : out Block);
  \end{lstlisting}\\
Diese Prozedur generiert einen Schl�ssselstromblock (\textit{Keystream})
indem sie den Wert des internen Z�hler Counter zuerst zu C verschl�sselt,
ihn dann um Eins erh�ht und zu guter letzt C ausgibt.
Bei der Initialisierung wird Counter auf IV gesetzt.\\
Mathematische Beschreibung: $C = E_K(Counter); \quad Counter:=Counter+1$\\
\end{tabular}


%%%%%%%%%%%%%%%%%%%%%%%%%%%%%%%%%%%%%%%%%%%%%%%%%%%%%%%%%%%%%%%%%%%%%%%%%%%

\section{Output-Feedback-Modus (OFB)}\label{OFB}
\subsubsection{Paket: Crypto.Symmetric.Mode.OFB}
Der OFB-Modus transformiert wie der Counter-Modus eine Blockchiffre in eine 
Stromchiffre. D.h. die interne R�ckkopplung ist hier unabh�ngig vom Klartext.

\subsection{Verschl�sselung}
Bei der Initialisierung wird der interne Schl�sselstromblock $K_0$ auf IV
gesetzt.
Bei der Verschl�sselung eines Klartextblock $P_i$ wird $K_{i-1}$ zu $K_i$
verschl�sselt und mit $P_i$ XOR verkn�pft. Das Ergebnis dieser Operation, ist
der Chiffretextblock $C_i$\\
Mathematische Beschreibung: $C_i =  P_i \oplus K_i$

\subsection{Entschl�sselung}
Die Entschl�sselung verl�uft analog zur Verschl�sselung. Zu Beginn wird der
Schl�sselstromblock $K_0$ mit dem Startwert $IV$ initialisiert bzw. mittels 
\textit{Set\_IV} reinitialisiert. 
Bei der Entschl�sselung eines Chiffretextblock $C_i$ wird $K_{i-1}$ zu $K_i$
verschl�sselt und mit $C_i$ XOR verkn�pft. Das Ergebnis dieser Operation, ist
der Klartextblock $P_i$. Die Chiffretextbl�cke m�ssen in der gleichen
Reihenfolge 
in der sie geniertet wurden entschl�sselt werden.\\
Mathematische Beschreibung: $P_i = C_i \oplus K_i$


\subsection{Verwendung}
Diesen Modus macht eigentlich nur mit der ``Low-Level-API'' f�r Experten 
(\ref{ofbllapi})
Sinn. Mit Hilfe dieser k�nnen Sie einen Schl�sselstrom ohne Klartextbl�cke 
generieren. Dadurch sind Sie in der Lage Klartextbl�cke sehr schnell zu  
verschl�sseln. Zum Beispiel w�re es denkbar nachts Schl�sselstrombl�cke zu 
generieren und mit Hilfe dieser dann Klartextbl�cke am Tag zu verschl�sselt. 
Dieser Modus eignet sich daher besonders gut, wenn 
\textbf{phasenweise  sehr schnell Klartextbl�cke verschl�sseln} werden m�ssen. 

\subsection{Anmerkungen}
\begin{itemize}
\item Der Schl�sselstrom wiederholt sich irgendwann. 
  D.h. $\exists\; L: K_0=K_L$
  Wenn m die Blockgr��e in Bits ist, betr�gt die durchschnittliche L�nge eines
  Zyklus $2^m-1$ Bits
\item Ein Bit-Fehler im Klartext wirkt sich nur auf ein Bit im Chiffretext aus 
  und umgekehrt.
\item Manipulationen am Klartext sind sehr einfach, da jede �nderung des 
  Chiffretextes beeinflusst direkt den Klartext.
\item Synchronisationsfehler (Alice und Bob haben unterschiedliche 
  Counterst�nde) k�nnen nicht behoben werden.
\end{itemize}


\subsection{Low-Level-API}\label{ofbllapi}
 Die folgende API sollten Sie nur dann verwenden, wenn Sie genau wissen was 
 Sie tun.\\
\begin{tabular}{p{\textwidth}}
\begin{lstlisting}{}
  procedure Next_Block(Keystream : out Block);
\end{lstlisting}\\
Bei der Initialisierung mittels der \textit{Init}-Prozedur wird der Startwert
(IV) als Schl�sselstromblock $K_0$ zwischengespeichert. Jedes mal wenn nun die 
\textit{Next\_Block}-Prozedur aufgerufen wird, passiert folgendes:  Der 
Schl�sselstromblock $K_i$ wird zu $K_{i+1}$ verschl�sselt, zwischengespeichert
und als Keystream ausgegeben.\\ \ \\
Mathematische Beschreibung: $K_i =  E_K(K_{i-1})$
\end{tabular}

\section{Generischer Teil}
\begin{lstlisting}{}
generic
   with package C is
      new Crypto.Symmetric.Blockcipher(<>);

   with function "xor" (Left, Right  : in C.Block) 
                         return C.Block is <>;

    -- Diese Funktion wird nur beim Counter-Mode benoetigt
   with function "+"   (Left : C.Block; Right : Byte)
                        return C.Block is <>;
\end{lstlisting}



\section{Anwendungsbeispiel}
\begin{lstlisting}{}
with Crypto.Types;
with Ada.Text_IO;
with Crypto.Symmetric.Blockcipher_Tripledes;
with Crypto.Symmetric.Mode.CBC;


procedure Bsp_Modus_CBC is
   use Ada.Text_IO;
   use Crypto.Types;

  package TDES renames Crypto.Symmetric.Blockcipher_Tripledes;

  --Benutze die TDES im sicheren CBC-Modus
   package TDES_CBC is new Crypto.Symmetric.Mode.CBC(TDES);

  use TDES_CBC;

  -- Schluessel
  Key : B_Block192 := 
              (16#00#, 16#00#, 16#00#, 16#00#, 16#00#, 16#00#,
               16#00#, 16#00#, 16#00#, 16#00#, 16#00#, 16#00#,
               16#00#, 16#00#, 16#00#, 16#00#, 16#01#, 16#23#,
               16#45#, 16#67#, 16#89#, 16#ab#, 16#cd#, 16#ef#);

   --Startwert
  IV : B_Block64 := (16#12#, 16#34#, 16#56#, 16#78#,
                     16#90#, 16#ab#, 16#cd#, 16#ef#);

   -- Klartext
   P_String : String :="Now is the time for all .";

   --Klartext wird in drei 64-Bit Bloecke unterteilt
   P : array (1..3) of B_Block64 :=
     ((To_Bytes(P_String(1..8))),
      (To_Bytes(P_String(9..16))),
      (To_Bytes(P_String(17..24))));

   -- Chiffretrext
   C : array (0..3) of B_Block64;
  begin
    --1. Initialisierung
    Init(Key, IV);

    -- 1a) Chiffreblock = Startwert.
    C(0) := IV;

   -- 2. Verschluesselung
   for I in P'Range loop
      Encrypt(P(I), C(I));
   end loop;

   -- Fuer die Entschluesselung wird die Chiffre mit dem
   -- gleichen Startwert wie bei der Entschluesselung reinitalisiert
     Set_IV(C(0));

   -- 3. Entschluesselung
   for I in P'Range loop
      Decrypt(C(I), P(I));
      Put(To_String(P(I)));
   end loop;
end  Bsp_Modus_CBC;
\end{lstlisting}

  \chapter{Crypto.Symmmetric.Mode.Oneway}

Dieses generische Paket betreibt eine Einwegblockchiffre in einen bestimmten 
Einweg-(Betriebs-)modus. Dieser verkn�pft gew�hnlich eine Einweg-Blockchiffre 
mit einer R�ckkopplung und einigen einfachen Operationen (+, xor). Ein 
Einweg-Modus wird mit
Hilfe eines zuf�lligen �ffentlichen Startwert (Initial Value (IV)) 
initialisiert. Der Chiffretext ist damit nicht nur von der verwendeten Chiffre,
Klartext und Schl�ssel abh�ngig, sondern auch von dem zuf�lligen Startwert. 
Wenn Sie nun einen Klartext mehrmals mit der gleichen Chiffre und dem gleichen
Schl�ssel aber unterschiedlichen IVs verschl�sselt, erhalten Sie
unterschiedliche Chiffretext und durch die R�ckkopplung in einem Einweg-Modus
werden gleiche Klartextbl�cke zu unterschiedlichen Chiffretextbl�cken 
chiffriert, d.h ein Einweg-Betriebsmodus verschl�sselt zwei Klartextbl�cke
p1 und p2 mit p1=p2, mit  �berw�ltigender Wahrscheinlichkeit, zu zwei 
Chiffretextbl�cke c1 und c2 mit c1$\not=$c2. Damit ist es nun m�glich mehrere 
Nachrichten mit dem selben Schl�ssel zu verschl�sseln.\\
\textbf{Vorsicht: Um einen Chiffretext zu entschl�sseln ben�tigen Sie den 
gleichen Schl�ssel und Startwert wie bei der Verschl�sselung.}
Aus diesem Grund sollte der Startwert immer mit dem zugeh�rigen Chiffretext
aufbewahrt werden. \textbf{ Die Sicherheit eines Modus ist unabh�ngig vom 
\glqq Bekanntheitsgrad\grqq des Startwertes}. Daher Multipliziert man den
Startwert meist mit dem Chiffretext zu dem endg�ltigen Chiffrat indem man den
Startwert vor dem Chiffretext h�ngt ($IV*C = C'= IV||C$).\\

\section{Anmerkungen}
\begin{itemize}
\item 
  F�r die Einweg-Modus gilt das gleiche wie f�r einen normalen Modus.
  Falls ein normaler Modus auch als Einweg-Modus zur Verf�gung steht, dann
  sollte Sie diesen dem normalen Modus vorziehen, da dieser etwas schlanker
  ist. 
\item Die \textbf{API} ist identisch zu den normalen Modi [ \ref{mode-api} ]
\item Unterst�tze Modi
  \begin{itemize}
  \item Cipher-Feedback-Modus (CFB)  [ \ref{CFB} ]
  \item Counter-Modus (CTR) [ \ref{CTR} ]
  \item Output-Feedback-Modus (OFB) [ \ref{OFB} ]
  \end{itemize}
\end{itemize}

\section{Anwendungsbeispiel}
\begin{lstlisting}{}
with Crypto.Types, Ada.Text_IO, Crypto.Symmetric.Mode.Oneway_CTR;
with Crypto.Symmetric.Oneway_Blockcipher_Twofish128;

procedure Bsp_Oneway_Modus_CTR
   package TF128 renames Crypto.Symmetric.Oneway_Blockcipher_Twofish128;

   -- Benutze die Blockchiffre in einem sicheren Modus
   package Twofish128 is new Crypto.Symmetric.Mode.Oneway_CTR(TF128);

   use Ada.Text_IO, Crypto.Types, Twofish128;

   --Schluessel
   Key : B_Block128:=(16#2b#, 16#7e#, 16#15#, 16#16#, 16#28#, 16#Ae#,
                      16#D2#, 16#A6#, 16#Ab#, 16#F7#, 16#15#, 16#88#,
                      16#09#, 16#Cf#, 16#4f#, 16#3c#);

   --Startwert
   IV : B_Block128 := (15 => 1, others => 0);

   --Nachricht
   Message : String :="All your Base are belong to us! ";

   -- Nachrichtenbloecke
   P : array(1..2) of B_Block128 :=
     (To_Bytes(Message(1..16)), To_Bytes(Message(17..32)));

   --Chiffretextbloecke
   C : array(0..2) of B_Block128;

begin
   --Initialisierung
   Init(Key, IV);

   -- 1. Chiffreblock = Startwert.
   C(0) := IV;

    -- Verschluesselung
   for I in P'Range loop
      Encrypt(P(I), C(I));
   end loop;

   -- Fuer die Entschluesselung wird die Chiffre mit dem
   -- gleichen Startwert wie bei der Entschluesselung reinitalisiert
   Set_IV(C(0));

   -- Entschluesselung
   for I in P'Range loop
      Decrypt(C(I), P(I));
      Put(To_String(P(I)));
   end loop;

end Bsp_Oneway_Modus_CTR;
\end{lstlisting}





  \chapter{Crypto.Hashfunction}\label{hash}
Mit Hilfe dieses generischen Paketes l�sst sich aus dem Algorithmus einer
kryptographsichen Hashfunktion (siehe Kapitel \ref{algoh})
eine krypto. Hashfunktion erstellen die sich hervorragend f�r folgende Zwecke 
einsetzen:
\begin{itemize}
\item Integrit�ts�berpr�fung von Nachrichten
\item Generierung und Verifizierung von digitalen Signaturen.
\item Generierung von Zufallszahlen bzw. Zufallsbits
\end{itemize}
Der Zweck dieses Paketes ist es die API f�r Hashfunktionen zu vereintheilichen
und zu vereinfachen. Aus Sicht der Sicherheit spricht hier nichst dagegen 
direkt die native API aus \texttt{Crypto.Symmetric.Algorithm} zu verwenden.\\

%%%%%%%%%%%%%%%%%%%%%%%%%%%%%%%%%%%%%%%%%%%%%%%%%%%%%%%%%%%%%%%%%%%%%%%%%%%%
%%%%%%%%%%%%%%%%%%%%%%%%%%%%%%%%%%%%%%%%%%%%%%%%%%%%%%%%%%%%%%%%%%%%%%%%%%%%

\section{Generischer Teil}
 \begin{lstlisting}{}
generic
   type Hash_Type                 is private;
   type Message_Type              is private;
   type Message_Block_Length_Type is private;
   
   with procedure Init(Hash_Value : out Hash_Type) is <>;

   with procedure Round(Message_Block : in     Message_Type;
                        Hash_Value    : in out Hash_Type) is <>;

   with function Final_Round(Last_Message_Block  : Message_Type;
                             Last_Message_Length :
                             Message_Block_Length_Type;
                             Hash_Value          : Hash_Type)
                             return Hash_Type is <>;

   with procedure Hash(Message    : in Bytes;
                       Hash_Value : out  Hash_Type) is <>;

   with procedure Hash(Message    : in String;
                       Hash_Value : out  Hash_Type) is <>;

   with procedure F_Hash(Filename : in String;
                         Hash_Value : out  Hash_Type) is <>;

 \end{lstlisting}\ \\

%%%%%%%%%%%%%%%%%%%%%%%%%%%%%%%%%%%%%%%%%%%%%%%%%%%%%%%%%%%%%%%%%%%%%%%%%%%
%%%%%%%%%%%%%%%%%%%%%%%%%%%%%%%%%%%%%%%%%%%%%%%%%%%%%%%%%%%%%%%%%%%%%%%%%%%

\section{API}
Die API einer generischen Hashfunktion besteht aus einer High- und einer 
Low-Level-API. Die Low-Level-API sollen sie nur verwenden, wenn sie sich mit
krptographischen Hashfunktionen auskennen. Wenn dies nicht der Fall ist, dann
verwenden sie bitte die folgende High-Level-API.

\subsection{High-Level-API}
\begin{lstlisting}{}
  function Hash  (Message  : Bytes)  return Hash_Type;
  function Hash  (Message  : String) return Hash_Type;
  
  function F_Hash(Filename : String) return Hash_Type;
\end{lstlisting}
Die Funktion \textbf{Hash} liefert den Hashwert einer Nachricht 
(\textit{Message}). Bei der Nachricht kann es sich dabei entweder um ein 
Byte-Array oder einen String handeln. Die Funktion \textbf{F\_Hash} gibt den 
Hashwert der Datei \textit{Filename} zur�ck.  Beispielsweise liefert 
die Codezeile
\begin{lstlisting}{}
  H := F_Hash("/bin/ls")
\end{lstlisting}
den Hashwert von \texttt{/bin/ls}.\\ \ \\


\subsection{Low-Level-API}
Die Low-Level-API besteht aus einer Funktion und zwei Prozeduren. 

\begin{itemize}
\item Die Prozedur \textbf{Init} initialisiert bzw. reinitailisert die 
  Hashfunktion. Jedesmal wenn eine Nachricht gehashed werde soll muss 
  zun�chst diese Prozedure aufgerufen werden.
  \begin{lstlisting}{}
 procedure Init;
  \end{lstlisting}

\item Mit der Prozedure \textbf{Round}, k�nnen iterativ Nachrichtenbl�cke
  gehashed werden. 
  \begin{lstlisting}{}
procedure Round(Message_Block : in Message_Type);
 \end{lstlisting}
 
\item Die Funktion \textbf{Final\_Round} padded und hashed anschlie�end einen
  Nachrichtenblock \textit{Last\_Message\_Block}. Auf Grund des Paddings muss
  die Bytel�nge des Nachrichtenmaterials \textit{Last\_Message\_Length} 
  angegeben werden.  Denn eine Nachricht ist i.d.R.  k�rzer als eine 
  Nachrichtenblock vom Typ \textit{Message\_Type}. Der R�ckgabewert dieser 
  Funktion entspricht dem finalen Hashwert einer Nachricht.
   \begin{lstlisting}{}
function Final_Round(Last_Message_Block  : Message_Type;
                     Last_Message_Length : Message_Block_Length_Type)
                     return Hash_Type;
  \end{lstlisting}
\end{itemize}\ \\

%%%%%%%%%%%%%%%%%%%%%%%%%%%%%%%%%%%%%%%%%%%%%%%%%%%%%%%%%%%%%%%%%%%%%%%%%%%
%%%%%%%%%%%%%%%%%%%%%%%%%%%%%%%%%%%%%%%%%%%%%%%%%%%%%%%%%%%%%%%%%%%%%%%%%%%
%%%%%%%%%%%%%%%%%%%%%%%%%%%%%%%%%%%%%%%%%%%%%%%%%%%%%%%%%%%%%%%%%%%%%%%%%%%

\section{Anwendungsbeispiel}
\subsection{High-Level-API}
Das folgende Beispiel gibt den SHA-256 Hashwert von \textbf{/bin/ls} aus.
 \begin{lstlisting}{}
with Ada.Text_IO;
with Crypto.Types,
with Crypto.Hashfunction_SHA256;


use Crypto.Types, Ada.Text_IO;

procedure example is
   package SHA256 renames Crypto.Hashfunction_SHA256;

   Hash : W_Block256 := SHA256.F_Hash("/bin/ls");
begin
   for I in Hash'Range loop
      Put(To_Hex(Hash(I)));
   end loop;
   Put_Line(" /bin/ls");
end example;
 \end{lstlisting}

\pagebreak

\subsection{Low-Level-API}
\begin{lstlisting}{}
with Ada.Text_IO;
with Crypto.Types;
with Crypto.Hashfunction;
with Crypto.Symmetric.Algorithm.SHA256;

use Ada.Text_IO;
use Crypto.Types;
use Crypto.Symmetric.Algorithm.SHA256;

pragma Elaborate_All (Crypto.Hashfunction);

procedure Example_Hashing is
   package WIO is new  Ada.Text_IO.Modular_IO(Word);

    package SHA256 is
      new Crypto.Hashfunction(Hash_Type    => W_Block256,
                              Message_Type => W_Block512,
                              Message_Block_Length_Type =>
                              Crypto.Types.Message_Block_Length512);

   Message : String :=  "All your base are belong to us!";

   W : Words := To_Words(To_Bytes(Message));

   M : W_Block512 := (others => 0);
   H : W_Block256;
begin
   M(W'Range) := W;

   SHA256.Init;

   -- Berechne den Hashwert von Message
   H := SHA256.Final_Round(M, Message'Last);

   -- Gib den Hashwert von Message aus
   for I in W_Block256'Range loop
      WIO.Put(H(I), Base => 16);
      New_line;
   end loop;
   New_Line;
end Example_Hashing;
\end{lstlisting}

\pagebreak

\section{Anmerkung}
Sie m�ssen nicht jedes mal von neuem aus einem daf�r vorgesehenen symmetrischen
Algorithmus eine Hashfunktion generieren. Stattdessen k�nnen Sie auch eine der 
vorgefertigten Hashfunktionen verwenden.
\begin{itemize}
\item Crypto.Hashfunktion\_SHA1
\item Crypto.Hashfunktion\_SHA256
\item Crypto.Hashfunktion\_SHA512
\item Crypto.Hashfunktion\_Whirlpool
\end{itemize}



  \chapter{Crypto.Symmetric.MAC}
Ein Message Authentication Code (MAC) dient zur Sicherung der Integrit�t und
Authentizit�t einer Nachricht. Er gew�hrt keine Verbindlichkeit, da er nur 
einen symmetrischen Schl�ssel verwendet. Bei den asymmertrischen Verfahren 
spricht man von digitalen Signaturen. Bei MACs von \glqq authentication Tags
\grqq. Der Autor dieser Bibliothek bezeichnet diese meist als (digitiale)
Stempel.

%%%%%%%%%%%%%%%%%%%%%%%%%%%%%%%%%%%%%%%%%%%%%%%%%%%%%%%%%%%%%%%%%%%%%%%%%%%
%%%%%%%%%%%%%%%%%%%%%%%%%%%%%%%%%%%%%%%%%%%%%%%%%%%%%%%%%%%%%%%%%%%%%%%%%%%

\section{Randomized MAC (RMAC)}
Der RMAC wurde von Eliane Jaulmes, Antoine Joux and Frederic
Valette entwickelt. Er basiert auf dem CBC-MAC und basiert daher auf einer 
Einweg-Blockchiffre (Kapitel \ref{oneblock}) . Er ist beweisbar sicher 
gegen�ber \glqq Geburtstagsparadoxon-Angriffe \grqq. Er ben�tigt zwei 
Schl�ssel.
Der digitale Stempel ist doppelt so lange wie die verwendete Blockchiffre.\\
\ \\
Mathematische Beschreibung:
$$\mbox{RMAC}_{K_1,K_2}(E,M) = E_{K2 \oplus R}(C_n)
\mbox{ mit } C_i = E_{K_1}(M_i \oplus C_{i-1}), \quad R \in_R \{0,1\}^{|K_1|}
\mbox{ und } C_0=0$$

%%%%%%%%%%%%%%%%%%%%%%%%%%%%%%%%%%%%%%%%%%%%%%%%%%%%%%%%%%%%%%%%%%%%%%%%%%%
%%%%%%%%%%%%%%%%%%%%%%%%%%%%%%%%%%%%%%%%%%%%%%%%%%%%%%%%%%%%%%%%%%%%%%%%%%%

\subsection{Generischer Teil}
\begin{lstlisting}{}
generic
   with package C is 
       new Crypto.Symmetric.Oneway_Blockcipher(<>);

   with procedure Read (Random : out C.Key_Type) is <>;
   with function "xor" (Left, Right : C.Block)
                         return C.Block is <>;
   with function "xor" (Left, Right : C.Key_type) 
                         return C.Key_Type is <>;
\end{lstlisting}

%%%%%%%%%%%%%%%%%%%%%%%%%%%%%%%%%%%%%%%%%%%%%%%%%%%%%%%%%%%%%%%%%%%%%%%%%%%
%%%%%%%%%%%%%%%%%%%%%%%%%%%%%%%%%%%%%%%%%%%%%%%%%%%%%%%%%%%%%%%%%%%%%%%%%%%

\subsection{API}
\subsubsection{High-Level-API}
\begin{lstlisting}{}
  type Blocks  is array (Integer range <>) of Block;

  procedure Sign(Message : in Blocks;
                 Key1, Key2 : in Key_Type;
                 R   : out Key_Type;
                 Tag : out Block);


  function Verify(Message : in Blocks;
                  Key1, Key2 : in Key_Type;
                  R   : in Key_Type;
                  Tag : in Block) return Boolean;
\end{lstlisting}\ \\
Die Funktion \textbf{Sign} signiert eine Nachricht \textbf{Message} unter
den beiden Schl�ssel \textbf{Key1} und \textbf{Key2}. Sie liefert einen 
digitiale Stempel $S=Tag||R$. Wobei es sich bei \textbf{R} um eine zuf�llig
generierte Zahl handelt.\\ \ \\
Die Funktion \textbf{Verify} liefert \textit{true} zur�ck, wenn $S=Tag||R$
ein g�ltiger Stempel ist ansonsten  \textit{false} zur�ck. D.h. wenn einer oder
Mehrere Parameter nicht mit denen aus der Prozedur \textbf{Sign} �bereinstimmt,
dann liefert die Funktion mit �berw�ltigender Wahrscheinlichkeit
\textit{false} zur�ck.\\ \ \\

%%%%%%%%%%%%%%%%%%%%%%%%%%%%%%%%%%%%%%%%%%%%%%%%%%%%%%%%%%%%%%%%%%%%%%%%%%%

\subsubsection{Low-Level-API}
\begin{lstlisting}{}
   procedure Init(Key1, Key2 : in Key_Type);

   procedure Sign(Message_Block : in Block);
   procedure Final_Sign(Final_Message_Block : in Block;
                        R   : out Key_Type;
                        Tag : out Block);

   procedure Verify(Message_Block : in Block);
   function Final_Verify(Final_Message_Block  : in Block;
                         R   : in Key_Type;
                         Tag : in Block)
                        return Boolean;
\end{lstlisting}\ \\
\begin{itemize}
\item Die Prozedur \textbf{Init} initalisiert einen RMAC mit den beiden 
  Schl�sseln \textbf{Key1} und \textbf{Key2} und einem internen Anfangszustand.
\item Die Prozedur \textbf{Sign} \glqq hashed \grqq einen Nachrichtenblock 
  (\textbf{Message\_Block}).
\item Die Prozedur \textbf{Final\_Sign} hashed den finalen Nachrichtenblock 
  \textbf{Final\_Message\_Block}, generiert eine Zufallszahl \textbf{R}
  berechnet \textbf{Tag} und setzt den RMAC auf seinen Anfangszustand zur�ck.
\item  Die Prozedur \textbf{Verify} \glqq hashed \grqq einen Nachrichtenblock 
  (\textbf{Message\_Block}).
\item Die Funktion \textbf{Final\_Verify} hashed den finalen Nachrichtenblock 
  \textbf{Final\_Message\_Block} und verifiziert mit Hilfe eines internen 
  Zustandes und \textbf{R}, ob es sich bei \textbf{Tag} bzw. $\mathbf{R||Tag}$ 
  um einen g�ltigen Stempel handelt. Falls ja gibt sie \textit{true}, ansonsten
  \textit{false} zur�ck.
\end{itemize}

%%%%%%%%%%%%%%%%%%%%%%%%%%%%%%%%%%%%%%%%%%%%%%%%%%%%%%%%%%%%%%%%%%%%%%%%%%%
%%%%%%%%%%%%%%%%%%%%%%%%%%%%%%%%%%%%%%%%%%%%%%%%%%%%%%%%%%%%%%%%%%%%%%%%%%%

\subsection{Anwendungsbeispiel}
\begin{lstlisting}{}
  with Crypto.Types; use Crypto.Types;
  with Ada.Text_IO;  use Ada.Text_IO;
  with Crypto.Symmetric.MAC.RMAC;
  with Crypto.Symmetric.Oneway_Blockcipher_AES128;

pragma Elaborate_All (Crypto.Symmetric.MAC.RMAC);
 
procedure Example is
  package AES128 renames  Crypto.Symmetric.Oneway_Blockcipher_AES128;
  package RMAC is new Crypto.Symmetric.MAC.RMAC(AES128);
  use RMAC;

  Key1 : B_Block128 :=
        (
         16#00#, 16#01#, 16#02#, 16#03#, 16#04#, 16#05#,
         16#06#, 16#07#, 16#08#, 16#09#, 16#0a#, 16#0b#,
         16#0c#, 16#0d#, 16#0e#, 16#0f#
        );
  
  Key2 : B_Block128 :=
        (
         16#00#, 16#11#, 16#22#, 16#33#, 16#44#, 16#55#, 16#66#,
         16#77#, 16#88#, 16#99#, 16#aa#, 16#bb#, 16#cc#, 16#dd#,
         16#ee#, 16#ff#
        );

      R    : B_Block128;
      Tag  : B_Block128;

      M : B_Block256 := To_Bytes("ALL YOUR BASE ARE BELONG TO US! ");
      Message : RMAC.Blocks(0..1) := (0 =>  M(0..15), 1 => M(16..31));
begin
  Init(K1,K2);

  Sign(Message(0));
  Final_Sign(Message(1), R, Tag) ;

  Verify(Message(0));
  Put_Line(Final_Verify(Message(1), R, Tag));
 end Example;
\end{lstlisting}\ \\

%%%%%%%%%%%%%%%%%%%%%%%%%%%%%%%%%%%%%%%%%%%%%%%%%%%%%%%%%%%%%%%%%%%%%%%%%%%
%%%%%%%%%%%%%%%%%%%%%%%%%%%%%%%%%%%%%%%%%%%%%%%%%%%%%%%%%%%%%%%%%%%%%%%%%%%
%%%%%%%%%%%%%%%%%%%%%%%%%%%%%%%%%%%%%%%%%%%%%%%%%%%%%%%%%%%%%%%%%%%%%%%%%%%

\section{Hashfunction MAC (HMAC)}
Bei Mihir Bellare, Ran Canettiy und Hugo Krawczykz handelt es sich um die HMAC
Designer. Der HMAC basiert auf einer Hashfunktion.\\ \ \\
Mathematische Beschreibung: 
$$\mbox{HMAC}_K(H,M) =  H(K \oplus opad, H(K \oplus ipad || M)) \mbox{ mit } 
opad = \{0x5C\}^n \mbox{ und } ipad = \{0x36\}^n$$

%%%%%%%%%%%%%%%%%%%%%%%%%%%%%%%%%%%%%%%%%%%%%%%%%%%%%%%%%%%%%%%%%%%%%%%%%%%
%%%%%%%%%%%%%%%%%%%%%%%%%%%%%%%%%%%%%%%%%%%%%%%%%%%%%%%%%%%%%%%%%%%%%%%%%%%

\subsection{Generischer Teil}
\begin{lstlisting}{}
  with package H is new Crypto.Hashfunction(<>);

   with function "xor"
     (Left, Right : H.Message_Type)  return H.Message_Type is <>;
   with procedure Fill36 (Ipad : out  H.Message_Type) is <>;
   with procedure Fill5C (Opad : out  H.Message_Type) is <>;

   with procedure Copy
     (Source : in H.Hash_Type; Dest : out H.Message_Type) is <>;

    with function M_B_Length
       (Z : H.Hash_Type) return H.Message_Block_Length_Type is <>;
\end{lstlisting}\ \\
\begin{itemize}
\item Die beiden Prozeduren \textbf{Fill36} und \textbf{Fill5C} f�llen
  einen \textit{Message\_Type} \textbf{Ipad} bzw. \textbf{Opad}  mit dem Wert 
  \texttt{0x36} bzw. \texttt{0x5C} auf.
\item  Die Prozedur \textbf{Copy} kopiert den Inhalt von \textbf{Soucre} nach
\textbf{Destination}. Ist der  \textit{Message\_Type} \glqq kleiner \grqq
als der \textit{Hash\_Type} der Hashfunktion, dann wird der Rest mit Nullen
aufgef�llt. 
\item Die Funktion \textbf{M\_B\_Length} gibt die L�nge in Bytes von von 
  \textbf{Z} bzw. dem \textit{Hash\_Typ} zur�ck.
\end{itemize}
All diese Hilfsfmethoden sind in \textit{Crypto.Symmetric.MAC} definiert.\\ 
\ \\

%%%%%%%%%%%%%%%%%%%%%%%%%%%%%%%%%%%%%%%%%%%%%%%%%%%%%%%%%%%%%%%%%%%%%%%%%%%
%%%%%%%%%%%%%%%%%%%%%%%%%%%%%%%%%%%%%%%%%%%%%%%%%%%%%%%%%%%%%%%%%%%%%%%%%%%

\subsection{API}
\begin{lstlisting}{}
    procedure Init(Key : in Message_Type);
    
    procedure Sign(Message_Block : in Message_Type);

    procedure Final_Sign
      (Final_Message_Block        : in Message_Type;
       Final_Message_Block_Length : in Message_Block_Length_Type;
       Tag                        : out Hash_Type);
    
    
    procedure Verify(Message_Block : in Message_Type);

    function Final_Verify
      (Final_Message_Block        : Message_Type;
       Final_Message_Block_Length : Message_Block_Length_Type;
       Tag                        : Hash_Type) return Boolean;
\end{lstlisting}\ \\
\begin{itemize}
\item Die Prozedur \textbf{Init} initialisiert den HMAC mit dem Schl�ssel
  \textbf{Key} und einem internen Anfangszustand.
\item Die Prozedur \textbf{Sign} hashed einen Nachrichtenblock.
\item \item Die Prozedur \textbf{Final\_Sign} hashed einen Nachrichtenblock
  der Bytel�nge  \textbf{Final\_Message\_Block\_Length}, gibt den digitalen 
  Stempel \textbf{Tag} zur�ck und setzt den HMAC auf seine Anfangszustand 
  zur�ck.
\item Die Prozedur \textbf{Verify} hashed einen Nachrichtenblock.
\item Die Funktion  \textbf{Final\_Verify} hashed einen Nachrichtenblock 
  der Bytel�nge  \textbf{Final\_Message\_Block\_Length} und verifiziert mit 
  Hilfe eines internen Zustandes ob es sich bei \textbf{Tag} um einen g�ltigen 
  Stempel handelt. Falls ja gibt sie \textit{true}, ansonsten
  \textit{false} zur�ck.
\end{itemize}\ \\  

%%%%%%%%%%%%%%%%%%%%%%%%%%%%%%%%%%%%%%%%%%%%%%%%%%%%%%%%%%%%%%%%%%%%%%%%%%%
%%%%%%%%%%%%%%%%%%%%%%%%%%%%%%%%%%%%%%%%%%%%%%%%%%%%%%%%%%%%%%%%%%%%%%%%%%%

\subsection{Anwendungsbeispiel}
 \begin{lstlisting}{}
with Ada.Text_IO;
with Crypt.Types;
with Crypto.Symmetric.MAC;
with Crypto.Symmetric.MAC.HMAC;
with Crypto.Hashfunction_SHA256;

use  Ada.Text_IO;
use  Crypt.Types;

pragma Elaborate_All (Crypto.Symmetric.MAC.HMAC);


procedure Example is

  package RMAC is new Crypto.Symmetric.MAC.HMAC
           (H          => Crypto.Hashfunction_SHA256,
            Copy       => Crypto.Symmetric.Mac.Copy,
            Fill36     => Crypto.Symmetric.Mac.Fill36,
            Fill5C     => Crypto.Symmetric.Mac.Fill5C,
            M_B_Length => Crypto.Symmetric.Mac.M_B_Length);

  use RMAC;

  -- 160-Bit Key
  Key1 : W_Block512 := (0 => 16#0b_0b_0b_0b#, 1 => 16#0b_0b_0b_0b#,
                        2 => 16#0b_0b_0b_0b#, 3 => 16#0b_0b_0b_0b#,
                        4 => 16#0b_0b_0b_0b#, others  => 0);

   -- "Hi There" Len = 8 Byte
   Message : W_Block512 := (0 => 16#48_69_20_54#, 1 => 16#68_65_72_65#,
                            others => 0);

   Tag : W_Block256;
begin
  Init(Key1);
  Final_Sign(Message, 8, Tag);
  
  Put_Line(Final_Verify(Message1, 8, Tag));
end Example;
 \end{lstlisting}\ \\


%%%%%%%%%%%%%%%%%%%%%%%%%%%%%%%%%%%%%%%%%%%%%%%%%%%%%%%%%%%%%%%%%%%%%%%%%%%
%%%%%%%%%%%%%%%%%%%%%%%%%%%%%%%%%%%%%%%%%%%%%%%%%%%%%%%%%%%%%%%%%%%%%%%%%%%

\subsection{Anmerkungen}
Sie m�ssen nicht jedes mal von neuem aus einer Hashfunktion einen HMAC
generieren. Stattdessen k�nnen Sie auch eine der vorgefertigten HMACs 
verwenden.
\begin{itemize}
\item Crypto.Symmetric.MAC.HMAC\_SHA1
\item Crypto.Symmetric.MAC.HMAC\_SHA256
\item Crypto.Symmetric.MAC.HMAC\_SHA512
\item Crypto.Symmetric.MAC.HMAC\_Whirlpool
\end{itemize}

  \chapter{Crypto.Types.Big\_Numbers}
Dieses Paket stellt den generischen Typen \textit{Big\_Unsigned}, mit einem 
ganzen Satz von Prozeduren und Funktionen, zur Verf�gung. Dieser Typ verwendet
intern ein Array das aus k CPU-W�rtern besteht. Dieses Array wird als ein
modularer Typ interpretiert. Aus diesem Grund es ist nur m�glich 
Big\_Unsigneds deren Bitl�nge einem Vielfachem der CPU-Wortl�nge entspricht zu
generieren. Das dieses Paket ohne Zeiger und Inline-Assembler arbeitet, ist
es um ein vielfaches langsamer als z.B. die auf Effizienz optimierte MPI des
GnuPG\cite{gpg}.\\
Dieses Paket basiert auf Jerome Delcourts Big\_Number-Bibliothek\cite{bignum}.
Urspr�nglich sollte  diese Bibliothek an dieser Stelle verwendet werden.
Es stellte sich aber heraus, das diese doch nicht nicht den gew�nschten
Anforderungen entsprach. Aus diesem Grund wurde dieses Paket noch mal komplett
neu geschrieben. Einen wesentlichen Beitrag zu dem derzeitigen Code
trugen die Analyse des Quellcodes von java.math.BigInteger\cite{bigint} und
Bob Debliers beecyrpt\cite{beecrypt} sowie folgende Quellen
\cite{handout, cormen, schneier, wiki, 2004-hankerson} bei.


\subsubsection{Notation}
\begin{itemize}
\item $|X| \hat{=}$ : Bitl�nge von X.
\item $|CPU|$ \quad : L�nge eines CPU-Wortes.\\ 
  (Bei einem n-Bit Prozessor gilt i.d.R. $|CPU|=n$) 
\end{itemize}



%%%%%%%%%%%%%%%%%%%%%%%%%%%%%%%%%%%%%%%%%%%%%%%%%%%%%%%%%%%%%%%%%%%%%%%%%%%

\section{API}

\subsection{Generischer Teil}
\begin{lstlisting}{}
generic
   Size : Positive;
\end{lstlisting}
\textbf{Vorbedingung:}
$Size = k \cdot |CPU| \quad k \in N$\\ \ \\
\textbf{Exception:}
$Size \not= k \cdot |CPU| \quad k \in N$ : Constraint\_Size\_Error.\\ \ \\

Size gibt die Bitl�nge des Types \textit{Big\_Unsigned} an.\\

%%%%%%%%%%%%%%%%%%%%%%%%%%%%%%%%%%%%%%%%%%%%%%%%%%%%%%%%%%%%%%%%%%%%%%%%%%%

\subsection{Typen}\ 

\begin{tabular}{p{\textwidth}}
\begin{lstlisting}{}
  type Big_Unsigned is private;
\end{lstlisting}\\
Dies ist dar Basis-Typ des Paketes. Big\_Unsigned repr�sentiert eine 
modulare n-Bit Zahl. ($n = k\cdot m$ wobei m die L�nge eines CPU-Wortes ist).
Eine Variable von diesem Typ wir immer mit der Konstanten Big\_Unsigned\_Zero
(\ref{buz}) initialisiert.\\ \ \\
 \hline
\end{tabular}

\begin{tabular}{p{\textwidth}}
\begin{lstlisting}{}
   subtype Number_Base is Integer range 2 .. 16;
\end{lstlisting}\\
Dieser Typ wird sp�ter bei der Konvertierung einer Big\_Unsigned in einen 
String ben�tigt. Ein Big\_Unsigned Variable kann nur als eine Zahl zu einer
Basis dieses Types dargestellt werden.\\ \ \\
\hline
\end{tabular}

\begin{tabular}{p{\textwidth}}
\begin{lstlisting}{}
  type Mod_Type is mod 2**System.Word_Size;
   for Mod_Type'Size use System.Word_Size;
\end{lstlisting}\\
Mod\_Type ist ein Modulare Typ der die L�nge eines CPU-Wortes hat. Bei einem
24-Bit Prozessor ist System.Word\_Size = 24 bei einem 32-Bit Prozessor ist
System.Word\_Size = 32 usw.\\ \ \\ \ \\
\end{tabular}

%%%%%%%%%%%%%%%%%%%%%%%%%%%%%%%%%%%%%%%%%%%%%%%%%%%%%%%%%%%%%%%%%%%%%%%%%%%

\subsection{Konstanten}\label{buz}
\begin{lstlisting}{}
 Big_Unsigned_Zero    : constant Big_Unsigned; -- = 0
 Big_Unsigned_One     : constant Big_Unsigned; -- = 1
 Big_Unsigned_Two     : constant Big_Unsigned; -- = 2
 Big_Unsigned_Three   : constant Big_Unsigned; -- = 3
 Big_Unsigned_Four    : constant Big_Unsigned; -- = 4
 Big_Unsigned_Sixteen : constant Big_Unsigned; -- = 16
 Big_Unsigned_First   : constant Big_Unsigned; -- = 0
 Big_Unsigned_Last    : constant Big_Unsigned; -- = "Big_Unsigned'Last"
\end{lstlisting}\ \\ 


%%%%%%%%%%%%%%%%%%%%%%%%%%%%%%%%%%%%%%%%%%%%%%%%%%%%%%%%%%%%%%%%%%%%%%%%%%%

\section{Vergleichsoperationen}
\begin{lstlisting}{}
 -- Vergleiche Big_Unsigned mit Big_Unsigned

 function "="(Left, Right : Big_Unsigned) return Boolean;
 function "<"(Left, Right : Big_Unsigned) return Boolean;
 function ">"(Left, Right : Big_Unsigned) return Boolean;

 function "<="(Left, Right : Big_Unsigned) return Boolean;
 function ">="(Left, Right : Big_Unsigned) return Boolean;

 function Min(X, Y : in Big_Unsigned) return Big_Unsigned;
 function Max(X, Y : in Big_Unsigned) return Big_Unsigned;


  -- Vergleiche Big_Unsigned mit Mod_Type

 function "=" (Left  : Big_Unsigned;
               Right : Mod_Type)
               return  Boolean;

 function "=" (Left  : Mod_Type;
               Right : Big_Unsigned) 
	       return  Boolean;

 function "<" (Left  : Big_Unsigned;
               Right : Mod_Type)
               return Boolean;

 function "<" (Left  : Mod_Type;
               Right : Big_Unsigned)
               return  Boolean;

 function ">" (Left  : Big_Unsigned; 
               Right : Mod_Type)
               return  Boolean;

 function ">" (Left  : Mod_Type;
               Right : Big_Unsigned)
               return  Boolean;

 function "<=" (Left  : Big_Unsigned;
                Right : Mod_Type)
                return  Boolean;

 function "<="(Left  : Mod_Type;
               Right : Big_Unsigned)
               return  Boolean;
  
 function ">=" (Left  : Big_Unsigned;
                Right : Mod_Type)
                return  Boolean;

 function ">=" (Left  : Mod_Type; 
                Right : Big_Unsigned) 
                return  Boolean;
\end{lstlisting}\ \\

\subsection{Elementare Operationen}
\begin{lstlisting}{}
function "+" (Left, Right : Big_Unsigned) return Big_Unsigned;

function "+" (Left  : Big_Unsigned;
              Right : Mod_Type) 
              return  Big_Unsigned;

function "+" (Left :  Mod_Type;
              Right : Big_Unsigned)
              return  Big_Unsigned;


function "-" (Left, Right : Big_Unsigned) return Big_Unsigned;

function "-" (Left  : Big_Unsigned;
              Right : Mod_Type)
              return  Big_Unsigned;

function "-" (Left  : Mod_Type;
              Right : Big_Unsigned) 
              return  Big_Unsigned;


function "*" (Left, Right : Big_Unsigned) return Big_Unsigned;

function "*" (Left  : Big_Unsigned; 
              Right : Mod_Type) 
              return  Big_Unsigned;

function "*" (Left  : Mod_Type; 
              Right : Big_Unsigned)
              return  Big_Unsigned;


function "/" (Left, Right : Big_Unsigned) return Big_Unsigned;

function "/" (Left  : Big_Unsigned; 
              Right : Mod_Type)
              return  Big_Unsigned;

function "/" (Left  : Mod_Type; 
              Right : Big_Unsigned)
              return  Big_Unsigned;


function "xor" (Left, Right : Big_Unsigned) return Big_Unsigned;

function "and" (Left, Right : Big_Unsigned) return Big_Unsigned;

function "or"  (Left, Right : Big_Unsigned) return Big_Unsigned;

function "**"  (Left, Right : Big_Unsigned) return Big_Unsigned;

function "mod" (Left, Right : Big_Unsigned) return Big_Unsigned;

function "mod" (Left  : Big_Unsigned;
                Right : Mod_Type) 
                return  Big_Unsigned;
\end{lstlisting} \ \\


\subsubsection{Multiplikation}
Zus�tzlich zu der elementaren Multiplikation wurden noch weitere Algorithmen 
implementiert, die ihr Potential bei gr��eren Zahlen entfalten. Leider lassen
die theoretischen Laufzeiten nicht in der ACL erreichen. Je komplexer die 
Algorithmen werden, desto mehr Rechenoperationen und Variablen werden ben�tigt.
An dieser Stelle l�sst sich der Flaschenhals erkennen, welcher in der 
Initialisierung von \textit{Big\_Unsigned} steckt.\\

Weitere Multiplikationsalgorihmen sind:\\
\hline

\begin{lstlisting}{}
 function Russ        (Left, Right : Big_Unsigned)
                      return Big_Unsigned;
\end{lstlisting}
Die Russische Bauernmultiplikation $O(N^2)$ basierend auf Bitoperationen.\\

\hline
\begin{lstlisting}{}
 function Karatsuba   (Left, Right : Big_Unsigned)
                      return Big_Unsigned;
\end{lstlisting}
Der Karatsuba $ \left(O\left(N^{log_2 3}\right) = O\left(N^{1.585}\right)\right)$ 
Algorithmus teilt die Faktoren intern in Polynome 1. Grades und brechnet die 
Teilprodukte mit der Schulmethode.\\

\hline
\begin{lstlisting}{}
 function Karatsuba_P (Left, Right : Big_Unsigned)
                      return Big_Unsigned;
\end{lstlisting}
Die Berechnung der Teilprodukte wird parallel in Tasksausgef�hrt.\\

\hline
\begin{lstlisting}{}
 function Toom_Cook   (Left, Right : Big_Unsigned)
                      return Big_Unsigned;
\end{lstlisting}
Die hier Implementierte variante ist der Toom-Cook-3-Way 
$\left(O\left(N^{1,465}\right)\right)$ Algorithmus nach D. Knuth. Er teilt die 
Faktoren intern in Polynome 2. Grades und brechnet die 
Teilprodukte mit der Schulmethode.\\

\hline
\begin{lstlisting}{}
 function Toom_Cook_P (Left, Right : Big_Unsigned)
                      return Big_Unsigned;
\end{lstlisting}
Die Berechnung der Teilprodukte wird parallel in Tasksausgef�hrt.\\
\hline
\\ \ \\
Alle Algorithmen lassen sich explizit Aufrufen und zur Multiplikation zweier 
\textit{Big\_Unsigned} benutzen.\\

Eine Performance-Steigerung l�sst sich dennoch durch die Verzahnung der
Algorithmen erreichen, welche versucht die St�rken der verschiedenen Algorithmen
zu kombinieren. Intern ruft dazu der �berladene Operator $"*"$ f�r kleine Faktoren
die Schulmethode, f�r gr��ere (ca. 3100 Bit) den parallelen Karatsuba und ab
ca. 3900 Bit L�nge den parallelen Toom-Cook-Algorithmus auf.

%%%%%%%%%%%%%%%%%%%%%%%%%%%%%%%%%%%%%%%%%%%%%%%%%%%%%%%%%%%%%%%%%%%%%%%%%%%
%%%%%%%%%%%%%%%%%%%%%%%%%%%%%%%%%%%%%%%%%%%%%%%%%%%%%%%%%%%%%%%%%%%%%%%%%%%

\section{Utils}
In dem separatem Body Crypto.Types.Big\_Numbers.Utils verbergen sich sehr
viele n�tzliche Funktionen und Prozeduren. Der Zugriff erfolgt �ber das Pr�fix
\textbf{Utils.}\\

\begin{tabular}{p{\textwidth}}
\begin{lstlisting}{}
procedure Swap(X, Y : in out Big_Unsigned);
\end{lstlisting}\\
Diese Prozedur vertauscht \textit{X} mit \textit{Y}.\\ \ \\
\hline
\end{tabular}

%%%%%%%%%%%%%%%%%%%%%%%%%%%%%%%%%%%%%%%%%%%%%%%%%%%%%%%%%%%%%%%%%%%%%%%%%%%

\begin{tabular}{p{\textwidth}}
\begin{lstlisting}{}
procedure Set_Least_Significant_Bit(X : in out Big_Unsigned);
\end{lstlisting}\\
Diese Prozedur setzt das niederwertigste Bit von \textit{X} auf 1.\\
Dadurch ist \textit{X} nach diesem Prozeduraufruf immer ungerade. \\ \ \\
\hline
\end{tabular}

%%%%%%%%%%%%%%%%%%%%%%%%%%%%%%%%%%%%%%%%%%%%%%%%%%%%%%%%%%%%%%%%%%%%%%%%%%%

\begin{tabular}{p{\textwidth}}
\begin{lstlisting}{}
 procedure Set_Most_Significant_Bit(X : in out Big_Unsigned); 
\end{lstlisting}\\
Diese Prozedur setzt das h�chstwertigste Bit von \textit{X} auf 1.\\
Damit ist \textit{X} nach dem Prozeduraufruf eine Size-Bit Zahl.\\ \ \\
\hline
\end{tabular}

%%%%%%%%%%%%%%%%%%%%%%%%%%%%%%%%%%%%%%%%%%%%%%%%%%%%%%%%%%%%%%%%%%%%%%%%%%%

\begin{tabular}{p{\textwidth}}
\begin{lstlisting}{}
  function Is_Odd(X : Big_Unsigned) return Boolean;
\end{lstlisting}\\
Diese Funktion liefert  \textit{True} zur�ck, wenn
\textit{X} ungerade ist, ansonsten \textit{False}.  \\ \ \\
\hline
\end{tabular}

%%%%%%%%%%%%%%%%%%%%%%%%%%%%%%%%%%%%%%%%%%%%%%%%%%%%%%%%%%%%%%%%%%%%%%%%%%%

\begin{tabular}{p{\textwidth}}
\begin{lstlisting}{}
  function Is_Even(X : Big_Unsigned) return Boolean;
\end{lstlisting}\\
Diese Funktion liefert \textit{True} zur�ck, wenn
X gerade ist, ansonsten \textit{False}. \\ \ \\
\hline
\end{tabular}

%%%%%%%%%%%%%%%%%%%%%%%%%%%%%%%%%%%%%%%%%%%%%%%%%%%%%%%%%%%%%%%%%%%%%%%%%%%

\begin{tabular}{p{\textwidth}}
\begin{lstlisting}{}
 procedure Inc(X : in out Big_Unsigned);
\end{lstlisting}\\
Diese Prozedur erh�ht X um 1.\\ \ \\
\hline
\end{tabular}

%%%%%%%%%%%%%%%%%%%%%%%%%%%%%%%%%%%%%%%%%%%%%%%%%%%%%%%%%%%%%%%%%%%%%%%%%%%


\begin{tabular}{p{\textwidth}}
\begin{lstlisting}{}
 procedure Dec(X : in out Big_Unsigned);
\end{lstlisting}\\
Diese Prozedur vermindert X um 1.\\ \ \\
\hline
\end{tabular}

%%%%%%%%%%%%%%%%%%%%%%%%%%%%%%%%%%%%%%%%%%%%%%%%%%%%%%%%%%%%%%%%%%%%%%%%%%%

\begin{tabular}{p{\textwidth}}
\begin{lstlisting}{}
 function Shift_Left (Value  : Big_Unsigned; 
                      Amount : Natural) 
                      return   Big_Unsigned;

\end{lstlisting}\\
Diese Funktion berechnet  $Value * 2^{Amount}$.\\ \ \\
\hline
\end{tabular}

%%%%%%%%%%%%%%%%%%%%%%%%%%%%%%%%%%%%%%%%%%%%%%%%%%%%%%%%%%%%%%%%%%%%%%%%%%%

\begin{tabular}{p{\textwidth}}
\begin{lstlisting}{}
function Shift_Right (Value  : Big_Unsigned; 
                      Amount : Natural)
                      return    Big_Unsigned;
\end{lstlisting}\\
Diese Funktion berechnet  $\lfloor Value / 2^{Amount}\rfloor$.\\ \ \\
\hline
\end{tabular}

%%%%%%%%%%%%%%%%%%%%%%%%%%%%%%%%%%%%%%%%%%%%%%%%%%%%%%%%%%%%%%%%%%%%%%%%%%%


\begin{tabular}{p{\textwidth}}
\begin{lstlisting}{}
function Rotate_Left(Value  : Big_Unsigned;
                     Amount : Natural)
                     return   Big_Unsigned;
\end{lstlisting}\\
Diese Funktion berechnet
$((Value * 2^{Amount})\; \oplus\; (\lfloor Value / 2^{|Value|-Amount}\rfloor))$
$\bmod\; 2^{|Value|+1}$.\\ \ \\

\hline
\end{tabular}

%%%%%%%%%%%%%%%%%%%%%%%%%%%%%%%%%%%%%%%%%%%%%%%%%%%%%%%%%%%%%%%%%%%%%%%%%%%

\begin{tabular}{p{\textwidth}}
\begin{lstlisting}{}
function Rotate_Right(Value  : Big_Unsigned; 
                      Amount : Natural)
                      return   Big_Unsigned;
\end{lstlisting}\\
Diese Funktion berechnet
$((Value * 2^{|Value|-Amount})\; \oplus\; (\lfloor Value / 2^{Amount}\rfloor))$
$\bmod\; 2^{|Value|+1}$.\\ \ \\
\hline
\end{tabular}

%%%%%%%%%%%%%%%%%%%%%%%%%%%%%%%%%%%%%%%%%%%%%%%%%%%%%%%%%%%%%%%%%%%%%%%%%%%

\begin{tabular}{p{\textwidth}}
\begin{lstlisting}{}
function Get_Random return Big_Unsigned;
\end{lstlisting}\\
Diese Funktion generierte eine Zufallszahl aus dem Intervall 
\{$0\ldots ,$Big\_Unsigned\_Last\}.\\ \ \\
\hline
\end{tabular}

%%%%%%%%%%%%%%%%%%%%%%%%%%%%%%%%%%%%%%%%%%%%%%%%%%%%%%%%%%%%%%%%%%%%%%%%%%%

\begin{tabular}{p{\textwidth}}
\begin{lstlisting}{}
function Bit_Length(X : Big_Unsigned) return Natural;
\end{lstlisting}\\
Diese Funktion berechnet die Bitl�nge von X.\\ \ \\
\hline
\end{tabular}

%%%%%%%%%%%%%%%%%%%%%%%%%%%%%%%%%%%%%%%%%%%%%%%%%%%%%%%%%%%%%%%%%%%%%%%%%%%

\begin{tabular}{p{\textwidth}}
\begin{lstlisting}{}
function Lowest_Set_Bit(X : Big_Unsigned) return Natural;
\end{lstlisting}\\
Diese Funktion berechnet die Position des niederwertigsten Bits von X, das
den Wert ``1'' hat.\\ \ \\
\textbf{Exception:}\\
$X =$ Big\_Unsigned\_Zero : Is\_Zero\_Error\\ \ \\   
\hline
\end{tabular}

%%%%%%%%%%%%%%%%%%%%%%%%%%%%%%%%%%%%%%%%%%%%%%%%%%%%%%%%%%%%%%%%%%%%%%%%%%%

\begin{tabular}{p{\textwidth}}
\begin{lstlisting}{}
function Gcd(Left, Right : Big_Unsigned) return Big_Unsigned;
\end{lstlisting}\\
Diese Funktion berechnet den gr��ten gemeinsamen Teiler von \textit{Left}
und \textit{Right}.\\ \ \\
\hline
\end{tabular}

%%%%%%%%%%%%%%%%%%%%%%%%%%%%%%%%%%%%%%%%%%%%%%%%%%%%%%%%%%%%%%%%%%%%%%%%%%%

\begin{tabular}{p{\textwidth}}
\begin{lstlisting}{}
function Length_In_Bytes(X : Big_Unsigned) return Natural;
\end{lstlisting}\\
Diese Funktion berechnet die Anzahl der Bytes die man ben�tigt um X als
Byte-Array auszugeben.\\ \ \\
\hline
\end{tabular}

%%%%%%%%%%%%%%%%%%%%%%%%%%%%%%%%%%%%%%%%%%%%%%%%%%%%%%%%%%%%%%%%%%%%%%%%%%%

\begin{tabular}{p{\textwidth}}
\begin{lstlisting}{}
function To_Big_Unsigned(X : Bytes) return Big_Unsigned;
\end{lstlisting}\\
Diese Funktion wandelt ein Byte-Array \textit{X} in eine Big\_Unsigned B um.
Dabei wird X(X'First) zum h�chstwertigstem Byte von B und X(X'Last)
zum niederwertigstem Byte von B.\\ \ \\
\textbf{Exception:}\\
$X'Length * Byte'Size > Size$  : Constraint\_Error\\ \ \\
\hline
\end{tabular}

%%%%%%%%%%%%%%%%%%%%%%%%%%%%%%%%%%%%%%%%%%%%%%%%%%%%%%%%%%%%%%%%%%%%%%%%%%%

\begin{tabular}{p{\textwidth}}
\begin{lstlisting}{}
function To_Bytes(X : Big_Unsigned) return Bytes;
\end{lstlisting}\\
Diese Funktion wandelt \textit{X} in ein Byte-Array B um.
Dabei wird das h�chstwertigste Byte von X zu B(B'First), und das 
niederwertigste Byte von X wird zu B(B'Last).\\ \ \\  
\hline
\end{tabular}

%%%%%%%%%%%%%%%%%%%%%%%%%%%%%%%%%%%%%%%%%%%%%%%%%%%%%%%%%%%%%%%%%%%%%%%%%%%

\begin{tabular}{p{\textwidth}}
\begin{lstlisting}{}
function To_String (Item : Big_Unsigned;
                    Base : Number_Base := 10)
		    return String;
\end{lstlisting}\\
Diese Funktion wandelt \textit{Item} in einen String um. Die Umwandlung 
erfolgt dabei zur Basis \textit{Base} ($Base \in \{2,\ldots ,16\}$)
Der String hat folgenden Aufbau:
\begin{description}
\item[Base=10: ] Einen Zahl zur Basis 10
  (Bsp. ``1325553'')
\item[Base/=10: ] \textit{Base}\# eine Zahl zur Basis B\#
  (Bsp. ``12\#AB45623A3402\#'')
\end{description}\ \\ \ \\
\hline
\end{tabular}

%%%%%%%%%%%%%%%%%%%%%%%%%%%%%%%%%%%%%%%%%%%%%%%%%%%%%%%%%%%%%%%%%%%%%%%%%%%

\begin{tabular}{p{\textwidth}}
\begin{lstlisting}{}
function To_Big_Unsigned(S : String) return Big_Unsigned;
\end{lstlisting}\\
Dies Funktion wandelt eine Zeichenkette \textit{S} in eine Big\_Unsigned um.
Die �bergebene Zeichenkette wird als Zahl in dezimaler Notation oder als 
eine Zahl zur Basis B ($B \in \{2,...,16\}$) betrachtet und muss folgenden
Aufbau haben: 
\begin{itemize}
\item B\#eine Zahl zur Basis B\#
  (Bsp. S = ``16\#FF340A12B1\#'')
\item eine Zahl zur Basis 10 
  (Bsp. S = ``333665'')
\end{itemize} \ \\
\textbf{Exception:}\\
\begin{tabular}{l @{\ :\ } l}
  S ist eine leere Zeichenkette & Conversion\_Error\\
  S hat eine ung�ltige Basis & Conversion\_Error\\
  S hat ung�ltige Ziffern &  Conversion\_Error
\end{tabular}\ \\ \ \\
\hline
\end{tabular}

\begin{tabular}{p{\textwidth}}
\begin{lstlisting}{}
procedure Put (Item : in Big_Unsigned;
               Base : in Number_Base := 10);
\end{lstlisting}\\
Diese Prozedur gibt die Big\_Unsigned \textit{Item} auf der Standardausgabe
zur Basis \textit{Base} ($Base \in \{2,...,16\}$) aus.\\ \ \\ 
\hline
\end{tabular}

%%%%%%%%%%%%%%%%%%%%%%%%%%%%%%%%%%%%%%%%%%%%%%%%%%%%%%%%%%%%%%%%%%%%%%%%%%%

\begin{tabular}{p{\textwidth}}
\begin{lstlisting}{}
procedure Put_Line(Item : in Big_Unsigned;
                   Base : in Number_Base := 10);
\end{lstlisting}\\
Diese Prozedur gibt die Big\_Unsigned \textit{Item} auf der Standardausgabe,
inklusive Zeilenumbruch, zur Basis \textit{Base} 
($Base \in \{2,...,16\}$) aus.\\ \ \\ 
\hline
\end{tabular}

%%%%%%%%%%%%%%%%%%%%%%%%%%%%%%%%%%%%%%%%%%%%%%%%%%%%%%%%%%%%%%%%%%%%%%%%%%%

\begin{tabular}{p{\textwidth}}
\begin{lstlisting}{}
procedure Big_Div (Dividend, Divisor : in Big_Unsigned;
                   Quotient, Remainder : out Big_Unsigned);
\end{lstlisting}\\
Diese Prozedur berechnet den Quotienten \textit{Quotient} und den Rest
\textit{Remainder} einer ganzzahligen Division. Es gilt:
\begin{itemize}
\item $Quotient := \lfloor \frac{Dividend}{Divisor} \rfloor$
\item $Remainder :=  Dividend\; \bmod\; Divisor$
\end{itemize}\ \\
\textbf{Exception:}\\
$X =$ Big\_Unsigned\_Zero : Is\_Zero\_Error\\ \ \\   
\hline
\end{tabular}

%%%%%%%%%%%%%%%%%%%%%%%%%%%%%%%%%%%%%%%%%%%%%%%%%%%%%%%%%%%%%%%%%%%%%%%%%%%


\begin{tabular}{p{\textwidth}}
\begin{lstlisting}{}
procedure Short_Div (Dividend  : in  Big_Unsigned;
                     Divisor   : in  Mod_Type;
                     Quotient  : out Big_Unsigned;
                     Remainder : out Mod_Type);
\end{lstlisting}\\
Diese Prozedur berechnet den Quotienten \textit{Quotient} und den Rest
\textit{Remainder} einer ganzzahligen Division. Es gilt:
\begin{itemize}
\item $Quotient := \lfloor \frac{Dividend}{Divisor} \rfloor$
\item $Remainder :=  Dividend\; \bmod\; Divisor$
\end{itemize}\ \\
\textbf{Exception:}\\
$X =$ Big\_Unsigned\_Zero : Is\_Zero\_Error\\ \ \\   
\end{tabular}\ \\

%%%%%%%%%%%%%%%%%%%%%%%%%%%%%%%%%%%%%%%%%%%%%%%%%%%%%%%%%%%%%%%%%%%%%%%%%%%
%%%%%%%%%%%%%%%%%%%%%%%%%%%%%%%%%%%%%%%%%%%%%%%%%%%%%%%%%%%%%%%%%%%%%%%%%%%

\section{Mod\_Utils}
In dem separatem Body Crypto.Types.Big\_Numbers.Mod\_Utils befinden sich 
Funktionen und Prozeduren die man h�ufig in der Public-Key-Kryptographie 
ben�tigt. Der Zugriff erfolgt �ber das Pr�fix \textbf{Mod\_Utils.}\\

\begin{tabular}{p{\textwidth}}
\begin{lstlisting}{}
function Add (Left, Right, N : Big_Unsigned)
              return Big_Unsigned;
\end{lstlisting}
Diese Funktion berechnet $Left + Right \pmod{N}$.\\ \ \\
\hline
\end{tabular}

%%%%%%%%%%%%%%%%%%%%%%%%%%%%%%%%%%%%%%%%%%%%%%%%%%%%%%%%%%%%%%%%%%%%%%%%%%%

\begin{tabular}{p{\textwidth}}
\begin{lstlisting}{}
function Sub (Left, Right, N : Big_Unsigned)
              return Big_Unsigned;
\end{lstlisting}
Diese Funktion berechnet $Left - Right \pmod{N}$.\\ \ \\
\hline
\end{tabular}

%%%%%%%%%%%%%%%%%%%%%%%%%%%%%%%%%%%%%%%%%%%%%%%%%%%%%%%%%%%%%%%%%%%%%%%%%%%

\begin{tabular}{p{\textwidth}}
\begin{lstlisting}{}
function Div (Left, Right, N : Big_Unsigned)
              return Big_Unsigned;
\end{lstlisting}
Diese Funktion berechnet $Left / Right \pmod{N}$.\\ \ \\
\textbf{Exception:}\\
$Right =$ Big\_Unsigned\_Zero : Constraint\_Error.\\ \ \\
\hline
\end{tabular}

%%%%%%%%%%%%%%%%%%%%%%%%%%%%%%%%%%%%%%%%%%%%%%%%%%%%%%%%%%%%%%%%%%%%%%%%%%%

\begin{tabular}{p{\textwidth}}
\begin{lstlisting}{}
function Mult (Left, Right, N : Big_Unsigned) 
               return Big_Unsigned;
\end{lstlisting}
Diese Funktion berechnet $Left \cdot Right \pmod{N}$.\\ \ \\
\hline
\end{tabular}

%%%%%%%%%%%%%%%%%%%%%%%%%%%%%%%%%%%%%%%%%%%%%%%%%%%%%%%%%%%%%%%%%%%%%%%%%%%

\begin{tabular}{p{\textwidth}}
\begin{lstlisting}{}
function Pow (Base, Exponent, N : Big_Unsigned) 
              return Big_Unsigned;
\end{lstlisting}
Diese Funktion berechnet $Base^{Exponent} \pmod{N}$.\\ \ \\
\hline
\end{tabular}

%%%%%%%%%%%%%%%%%%%%%%%%%%%%%%%%%%%%%%%%%%%%%%%%%%%%%%%%%%%%%%%%%%%%%%%%%%%

\begin{tabular}{p{\textwidth}}
\begin{lstlisting}{}
function Get_Random (N : Big_Unsigned) return Big_Unsigned;
\end{lstlisting}
Diese Funktion berechnet eine zuf�llige Big\_Unsigned\textit{B} mit
$B < N$.\\ \ \\
\hline
\end{tabular}

%%%%%%%%%%%%%%%%%%%%%%%%%%%%%%%%%%%%%%%%%%%%%%%%%%%%%%%%%%%%%%%%%%%%%%%%%%%

\begin{tabular}{p{\textwidth}}
\begin{lstlisting}{}
function Inverse (X, N : Big_Unsigned) return Big_Unsigned;
\end{lstlisting}
Diese Funktion berechnet das Inverse (bez�glich der Multiplikation)
von \textit{X} mod \textit{N}. Falls kein Inverses von \textit{X} mod 
\textit{N} existiert gibt diese Funktion Big\_Unsigned\_Zero zur�ck. \\ \ \\ 
\hline
\end{tabular}

%%%%%%%%%%%%%%%%%%%%%%%%%%%%%%%%%%%%%%%%%%%%%%%%%%%%%%%%%%%%%%%%%%%%%%%%%%%

\begin{tabular}{p{\textwidth}}
\begin{lstlisting}{}
function Get_Prime(N : Big_Unsigned) return Big_Unsigned;
\end{lstlisting}
Diese Funktion berechnet mit einer �berw�ltigenden Wahrscheinlichkeit eine
Primzahl P mit $P < N$.  Sie benutzt dazu die Funktion \textit{Is\_Prime}
(\ref{isprime}).\\ \ \\
\textbf{Exception:}\\
  $N <=$  Big\_Unsigned\_Two :  Constraint\_Error\\ \ \\
\hline
\end{tabular}

%%%%%%%%%%%%%%%%%%%%%%%%%%%%%%%%%%%%%%%%%%%%%%%%%%%%%%%%%%%%%%%%%%%%%%%%%%%

\begin{tabular}{p{\textwidth}}
\begin{lstlisting}{}
function Get_N_Bit_Prime(N : Positive) return Big_Unsigned;
\end{lstlisting}
Diese Funktion berechnet mit einer �berw�ltigenden Wahrscheinlichkeit eine
N-Bit-Primzahl. Sie benutzt dazu die Funktion \textit{Is\_Prime}
(\ref{isprime}).\\ \ \\
\textbf{Exception:}\\
$(N = 2) \vee (N > Size)$  : Constraint\_Error \\ \ \\
\hline
\end{tabular}

%%%%%%%%%%%%%%%%%%%%%%%%%%%%%%%%%%%%%%%%%%%%%%%%%%%%%%%%%%%%%%%%%%%%%%%%%%%

\begin{tabular}{p{\textwidth}}\label{isprime}
\begin{lstlisting}{}
Is_Prime(X : Big_Unsigned) return Boolean;
\end{lstlisting}
Dies Funktion gibt mit �berw�ltigender Wahrscheinlichkeit \textit{False}
zur�ck, wenn es sich bei \textit{X} nicht um eine Primzahl handelt.\\
\textbf{Funktionsweise:}
\begin{enumerate}
\item Es wird getestet ob \textit{X} durch eine einstellige Primzahl 
(2,3,5,7) teilbar ist.
\item Es wird getestet ob \textit{X} durch eine zweistellige Primzahl 
  teilbar ist.
\item Es wird getestet ob es sich bei \textit{X} um eine 
  Lucas-Lemehr-Primzahl handelt.
\item Es wird getestet  ob \textit{X} durch eine dreistellige Primzahl 
teilbar ist.
\item Es werden 2-50 Miller-Rabin-Test durchgef�hrt. Wobei die Anzahl der 
  Tests von X abh�ngig ist. Je gr��er $|X|$, desto weniger Test werden
  durchgef�hrt.
\end{enumerate}\ \\
\hline
\end{tabular}

\begin{tabular}{p{\textwidth}}
\begin{lstlisting}{}
function Looks_Like_A_Prime(X : Big_Unsigned) return Boolean;
\end{lstlisting}
Dies Funktion gibt mit hoher Wahrscheinlichkeit \textit{False}
zur�ck, wenn es sich bei \textit{X} nicht um eine Primzahl handelt.\\ \ \\
\textbf{Funktionsweise:}\\
Wie \textit{Is\_Prime} (\ref{isprime}) mit dem Unterschied, da� anstelle des
Miller-Rabin-Tests ein einfacherer aber unzuverl�ssiger
Primzahlentest, der 2-50 Zufallszahlen zieht und testet ob es sich dabei
um einen Miller-Rabin-Zeugen oder einen echten Teiler von \textit{X} handelt.
Wenn keiner dieser Zufallszahlen ein Miller-Rabin-Zeuge oder ein echter Teiler
von \textit{X} ist, geht die Funktion davon aus, das es sich bei \textit{X} 
um eine Primzahl handelt.\\ \ \\
\hline
\end{tabular}

\begin{tabular}{p{\textwidth}}
\begin{lstlisting}{}
 function Passed_Miller_Rabin_Test (X : Big_Unsigned;
                                    S : Positive) 
                                    return Boolean;
\end{lstlisting}
Dies Funktion gibt \textit{True} zur�ck, wenn \textit{X} eine gewisse Anzahl 
von  Miller-Rabin-Test (\textit{S}-St�ck) �berlebt. Die Wahrscheinlichkeit
das eine Pseudoprimzahl diesen Test besteht ist kleiner als 
$\frac{1}{2^{2S}}$.\\ \ \\
\hline
\end{tabular}


 %%%%%%%%%%%%%%%%%%%%%%%%%%%%%%%%%%%%%%%%%%%%%%%%%%%%%%%%%%%%%%%%%%%%%%%%%%%

\begin{tabular}{p{\textwidth}}
  \begin{lstlisting}{}
function Jacobi(X, N : Big_Unsigned) return Integer;
  \end{lstlisting}
  Diese Funktion berechnet das Jacobisymbol f�r eine Element \textit{X} aus 
  $\mathbf{Z}_N$.\\ \ \\
  \underline{Vorbedingung:}\\
  N muss ungerade sein.\\ \ \\
  \underline{R�ckgabewerte:}\\
  \begin{tabular}{l@{\ :\ }l}
  Jacobi(N,C) = 0 & X mod N = 0\\
  Jacobi(N,C) = 1 & X ist ein quadratischer Rest mod N \\ 
  Jacobi(N,C) = -1 & X ist ein quadratischer Rest mod N \\ 
  \end{tabular}\\ \ \\
  \textbf{Ausnahme:}\\
  N ungerade :  Constraint\_Error\\ \ \\
  \hline
\end{tabular}
  

  \section{Bin�re K�rper GF($2^m$)}
\subsection{Erl�uterungen}
In dem separatem Body Crypto.Types.Big\_Numbers.Mod\_Binfield\_Utils befinden 
sich Funktionen und Prozeduren bei denen eine \texttt{Big\_Unsigned} Variable 
als Element aus dem K�rper GF($2^m$) interpretiert wird.\\ \ \\
\textbf{Beispiel:}
\begin{lstlisting}[frame=tlrb]{}
  -- Es wird das 1., 2. und 6. Bit von A gesetzt.
  -- Danach enspricht A dem Element/Polynom z^5 + z + 1
  A := Big_Unsigned_One;
  A := A xor Shift_left(Big_Unsigned_One,1);
  A := A xor Shift_left(Big_Unsigned_One,5);
\end{lstlisting}
Die \texttt{Big\_Unsigned} Variable F ist im folgenden immer ein irreduzible 
Polynom $f(z)$ vom Grad m.
Der Zugriff erfolgt �ber das Pr�fix \textbf{Binfield\_Utils.}\\

%%%%%%%%%%%%%%%%%%%%%%%%%%%%%%%%%%%%%%%%%%%%%%%%%%%%%%%%%%%%%%%%%%%%%%%%%%%
\subsection{API}

\begin{tabular}{p{\textwidth}}
\begin{lstlisting}{}
function B_Add(Left,Right : Big_Unsigned) return Big_Unsigned;
function B_Sub(Left,Right : Big_Unsigned) return Big_Unsigned;
\end{lstlisting}\\
Diese beiden Funktion berechnen Left xor Right.
Dies entspricht einer Addtion bzw. Subtraktion in GF($2^m$).\\ \ \\
\hline
\end{tabular}

%%%%%%%%%%%%%%%%%%%%%%%%%%%%%%%%%%%%%%%%%%%%%%%%%%%%%%%%%%%%%%%%%%%%%%%%%%%


\begin{tabular}{p{\textwidth}}
\begin{lstlisting}{}
  function B_Mult(Left, Right, F : Big_Unsigned) return Big_Unsigned;
\end{lstlisting}\\
Diese Funktion berechnet Left $\cdot$ Right mod F.\\ \ \\
\hline
\end{tabular}

%%%%%%%%%%%%%%%%%%%%%%%%%%%%%%%%%%%%%%%%%%%%%%%%%%%%%%%%%%%%%%%%%%%%%%%%%%%

\begin{tabular}{p{\textwidth}}
  \begin{lstlisting}{}
function B_Square(A, F : Big_Unsigned)    return Big_Unsigned;
  \end{lstlisting}\\
  Diese Funktion berechnet $A^2\; \bmod\; F$.\\ \ \\
  \hline
\end{tabular}

%%%%%%%%%%%%%%%%%%%%%%%%%%%%%%%%%%%%%%%%%%%%%%%%%%%%%%%%%%%%%%%%%%%%%%%%%%%

\begin{tabular}{p{\textwidth}}
\begin{lstlisting}{}
function B_Div (Left, Right, F : Big_Unsigned) return Big_Unsigned;
\end{lstlisting}\\
Diese Funktion berechnet Left/Right mod F.\\ \ \\
\hline
\end{tabular}

%%%%%%%%%%%%%%%%%%%%%%%%%%%%%%%%%%%%%%%%%%%%%%%%%%%%%%%%%%%%%%%%%%%%%%%%%%%

\begin{tabular}{p{\textwidth}}
  \begin{lstlisting}{}
function B_Mod(Left, Right  : Big_Unsigned) return Big_Unsigned;
  \end{lstlisting}\\
  Diese Funktion berechnet Left mod Right.\\ \ \\
  \hline
\end{tabular}


%%%%%%%%%%%%%%%%%%%%%%%%%%%%%%%%%%%%%%%%%%%%%%%%%%%%%%%%%%%%%%%%%%%%%%%%%%%


\begin{tabular}{p{\textwidth}}
  \begin{lstlisting}{}
function B_Inverse(X, F : Big_Unsigned) return Big_Unsigned;
  \end{lstlisting}\\
  Diese Funktion berechnet $X^-1\; \bmod\; F$.\\ \ \\
  \hline
\end{tabular}


%%%%%%%%%%%%%%%%%%%%%%%%%%%%%%%%%%%%%%%%%%%%%%%%%%%%%%%%%%%%%%%%%%%%%%%%%%%

\section{Exceptions}

\begin{lstlisting}{}
  Constraint_Size_Error : exception;
  Conversion_Error      : exception;
  Division_By_Zero      : exception;
  Is_Zero_Error         : exception;
\end{lstlisting}

  \chapter{Crypto}

Crypto is an empty root package. This package does not contain any code.
It's sole purpose is to provide an unique namespace for all ACL packages.
As all packages are descendants of this package, their names start with
the prefix Crypto (e.g. Crypto.Foo.Bar).


 

  \include{acl.crypto.asymmetric.dsa}
  \include{acl.crypto.asymmetric.rsa}
  \chapter{Ellyptische Kurven}
Die ACL unterst�tzt die beiden folgenden Klasen von ellyptischen Kurven
\begin{itemize}
\item $y^2 = x^3 + ax + b\; \bmod\; p \mbox{ mit } p \in \mathbf{P}
  \setminus\{2,3\}$
  \\
  Hier handelt es sich um ellyptische Kurven �ber dem endlichen K�rper $Z_p$.
\item  $y^2 + xy = x^3 + a(z)x^2 + b(z) \;\bmod\; f(z)$\\
  Hier handelt es sich �ber nicht supersingulare ellyptische Kurven �ber 
  den endlichen K�rper $GF(2^{\deg{f(z)}})$.\\
  \textbf{Vorraussetzung:} $f(z)$ ein irreduzibles Polynom.\\
\end{itemize}

\section{Wurzelpaket: Types.Elliptic\_Curves}
Dieses generische Paket ist das Wurzelpaket f�r ellyptische Kurven.
In ihm ist der Basistype \texttt{EC\_Point} f�r ellyptische Kurven definiert.

\subsection{API}

\subsubsection{Generischer Teil}
\begin{lstlisting}{}
  generic
  with package Big is new Crypto.Types.Big_Numbers(<>);
\end{lstlisting}

\subsubsection{Typen}
Dieses Paket stellt den Typ \textit{EC\_Point} zur Verf�gung.
\begin{lstlisting}{}
  -- (x,y)
  type EC_Point is record
     X : Big.Big_Unsigned;
     Y : Big.Big_Unsigned;
  end record;
\end{lstlisting}\ \\
Die Konstante \texttt{EC\_Point\_Infinity} entspricht dem Punkt $\infty$.
Dieser Punkt befindet sich auf jeder ellyptischen Kurve und entspricht dem
neutrale Element der Addition.


\subsubsection{Prozeduren}
\begin{lstlisting}{}
  procedure Put(Item : in EC_Point; Base : in Big.Number_Base := 10;
  
  procedure Put_Line(Item : in EC_Point; Base : in Big.Number_Base := 10);
\end{lstlisting}
Die beiden Prozeduren geben einen \texttt{EC\_Point} auf der Standardausgabe
in Form eines Tupels aus.\\ 
\ \\
\textbf{Beispiel:}
\begin{lstlisting}{}
  procedure Example is
    P : EC_Point(X => Big_unsigned_Three, Y => Big_Unsigned_one)
  begin
    Put(P);             -- Ausgabe: "(3,1)"
    Put(P, Base => 2);  -- Ausgabe: "(2#11, 2#1#)"
  end Example;
\end{lstlisting}\ \\ \ \\

\section{Kinderpakete}
Die beiden Kinderpakete
\begin{itemize}
  \item Types.Elliptic\_Curves.Zp
  \item Types.Elliptic\_Curves.NSS\_BF
\end{itemize}
verf�gen �ber die folgende Typen

\subsubsection{Typen}
Das Paket \textit{Types.Elliptic\_Curves.Zp} stellt den Typ \textit{Elliptic\_Curve\_Zp} zur Verf�gung.
\begin{lstlisting}{}
  -- (A,B,P)
  type Elliptic_Curve_Zp is record
    A : Big_Unsigned;
    B : Big_Unsigned;
    P : Big_Unsigned;
  end record;
\end{lstlisting}\ \\


\subsubsection{API}

\begin{tabular}{p{\textwidth}}
\begin{lstlisting}{}
   procedure Init(A, B, P : in Big_Unsigned);
   procedure Init(ECZ : in Elliptic_Curve_Zp);

   procedure Init(A, B, F : in Big_Unsigned);
\end{lstlisting}\\
Diese Prozeduren initialiseren die ellyptische Kurven.
Die ersten beiden bilden die ellyptische Kurve 
$y^2 = x^3 + Ax + B\; \bmod\; P$, die dritte die Kurve
$y^2 + xy = x^3 + Ax^2 + B \;\bmod\; F.$\\
Diese Funktion testet nicht, ob es sich bei \texttt{P} bzw. \texttt{F}
wirklich um eine Primzahl bzw. irreduzibles Polynom handelt oder die
Parameter tats�chlich eine ellyptische Kurve bilden.\\ \ \\
\hline
\end{tabular}   

%%%%%%%%%%%%%%%%%%%%%%%%%%%%%%%%%%%%%%%%%%%%%%%%%%%%%%%%%%%%%%%%%%%%%%%%%%%

\begin{tabular}{p{\textwidth}}
  \begin{lstlisting}{}
    function Is_Elliptic_Curve return Boolean;
  \end{lstlisting}\\
  Diese Funktion testet, ob durch die Initalisierung mittels \texttt{Init}
  ein ellyptische Kurve konstruiert worden ist. Wenn dies nicht der Fall ist,
  dann liefern alle folgenden Funktionen und Prozeduren wahrscheinlich 
  nicht das gew�nschte Ergebnis.\\ \ \\
  \hline
\end{tabular}   

%%%%%%%%%%%%%%%%%%%%%%%%%%%%%%%%%%%%%%%%%%%%%%%%%%%%%%%%%%%%%%%%%%%%%%%%%%%

\begin{tabular}{p{\textwidth}}
  \begin{lstlisting}{}
    function On_Elliptic_Curve(X : EC_Point) return Boolean;
  \end{lstlisting}
  Diese Funktion testet, ob \texttt{X} ein Punkt auf der eyllptischen Kurve 
  ist.\\ \ \\
\end{tabular}

%%%%%%%%%%%%%%%%%%%%%%%%%%%%%%%%%%%%%%%%%%%%%%%%%%%%%%%%%%%%%%%%%%%%%%%%%%%

\begin{tabular}{p{\textwidth}}
  \begin{lstlisting}{}
    function Negative(X : EC_Point) return EC_Point;
  \end{lstlisting}
  Dies Funktion berechnet \texttt{-X}.\\ \ \\
\end{tabular}

%%%%%%%%%%%%%%%%%%%%%%%%%%%%%%%%%%%%%%%%%%%%%%%%%%%%%%%%%%%%%%%%%%%%%%%%%%%

\begin{tabular}{p{\textwidth}}
  \begin{lstlisting}{}
    function "+"(Left, Right : EC_Point) return EC_Point;
  \end{lstlisting}
  Dies Funktion berechnet \texttt{Left + Right}.\\ \ \\
  \hline
\end{tabular}

%%%%%%%%%%%%%%%%%%%%%%%%%%%%%%%%%%%%%%%%%%%%%%%%%%%%%%%%%%%%%%%%%%%%%%%%%%%

   \begin{tabular}{p{\textwidth}}
  \begin{lstlisting}{}
    function Double(X : EC_Point) return EC_Point;
  \end{lstlisting}
  Dies Funktion berechnet \texttt{2X}.\\ \ \\
  \hline
   \end{tabular}



  \chapter{Ellyptische Kurven Datenbank ZP}
Die ACL bietet f�r die ellyptischen Kurven der Form
\begin{itemize}
\item $y^2 = x^3 + ax + b\; \bmod\; p \mbox{ mit } p \in \mathbf{P}
  \setminus\{2,3\}$
  \\
  Hier handelt es sich um ellyptische Kurven �ber dem endlichen K�rper $Z_p$.
\end{itemize}
eine Datenbank, welche die durch NIST (National Institute of Standards and Technology) herausgegebenen elliptischen Kurven enth�lt.
\\~\\
Es sind die folgenden Kurven enthalten.
\begin{itemize}
	\item Curve P-192
	\item Curve P-224
	\item Curve P-256
	\item Curve P-384
	\item Curve P-521
	\item Test Curve (5 bit)
\end{itemize}
 
\section{API}

\subsection{Generischer Teil}
\begin{lstlisting}{}
  generic
  with package Big is new Crypto.Types.Big_Numbers(<>);
\end{lstlisting}

\subsection{Typen}
Dieses Paket stellt den Typ \textit{Bit\_Length} und den Typ \textit{Precomputed\_Elliptic\_Curve} zur Verf�gung.
\begin{lstlisting}{}
  type Bit_Length is new natural;
  
 	type Precomputed_Elliptic_Curve is record
 	  -- prime modulus
		p      : String(1..192) := (others=>' ');
		-- order	
		r      : String(1..192) := (others=>' ');
		-- 160-bit input seed to SHA-1	
		s      : String(1..192) := (others=>' ');	
		-- output of SHA-1
		c      : String(1..192) := (others=>' ');	
		-- coefficient b (satisfying b*b*c = -27 (mod p))
		b      : String(1..192) := (others=>' ');
		-- base point x coordinate	
		Gx     : String(1..192) := (others=>' ');	
		-- base point x coordinate
		Gy     : String(1..192) := (others=>' ');
		-- Bit lenght	
		length : Bit_Length;                      
	end record;
\end{lstlisting}\ \\


\subsection{Prozeduren}

\begin{tabular}{p{\textwidth}}
\begin{lstlisting}{}
  procedure Get_Elliptic_Curve(ECZ    : out Elliptic_Curve_Zp; 
                               ECP    : out EC_Point; 
                               order  : out Big_Unsigned; 
                               length : in  Bit_Length);
\end{lstlisting}\\
Diese Prozedur holt alle Variablen aus der Datenbank, die zum Rechnen mit Elliptische Kurven ben�tigt werden. Die Kurve hat mindestens die kryptografische Sicherheit von \textit{length}.\\ \ \\
\textbf{Exception:}\\
\begin{tabular}{l @{\ :\ } l}
  BitLength is not supported. (Max BitLength = 521) & LEN\_EX\\
\end{tabular}\ \\ \ \\
\hline
\end{tabular}

\begin{tabular}{p{\textwidth}}
\begin{lstlisting}{}
  procedure Set_Elliptic_Curve_Map;
\end{lstlisting}\\
Diese Prozedur initiert die Datenbank. Sie ist nur f�r den internen Gebrauch notwendig.
\\ \ \\
\end{tabular}

\section{Anwendungsbeispiel}
\begin{lstlisting}{}
with Crypto.Types; use Crypto.Types;
with Crypto.Types.Big_Numbers;
with Crypto.Types.Elliptic_Curves.Zp;
with Crypto.Types.Elliptic_Curves.Zp.Database;

procedure Example_EC_DB_ZP is
  package Big is new Crypto.Types.Big_Numbers(Size);
  use Big;
  package EC  is new Crypto.Types.Elliptic_Curves(Big);
  use EC;
  package Zp is new EC.Zp;
  use Zp;
	package DB is new Zp.Database;
	use DB;
  
  EC_ZP : Public_Key_ECNR;
  EC_P  : Private_Key_ECNR;
  order : Signature_ECNR;

begin
	Get_Elliptic_Curve(EC_ZP, EC_P, order, 168);
end Example_EC_DB_ZP;
\end{lstlisting}



  \include{acl.crypto.asymmetric.ecdsa}
  \include{acl.crypto.asymmetric.ecnr}
  \include{acl.crypto.symmetric.algorithm.ecdh}
  \include{acl.crypto.symmetric.algorithm.ecmqv}
  \include{acl.crypto.symmetric.algorithm.ecies}  
  \chapter{Crypto.Certificate}
\label{sec:CryptoCertificate}
Mit Hilfe eines asymmetrischen Kryptosystems k�nnen Nachrichten in einem Netzwerk digital signiert und verschl�sselt werden. F�r jede verschl�sselte �bermittlung ben�tigt der Sender allerdings den �ffentlichen Schl�ssel (Public-Key) des Empf�ngers. Um zu �berpr�fen, ob es sich tats�chlich um den Schl�ssel des Empf�ngers handelt und nicht um eine F�lschung, benutzt man digitale Zertifikate, welche die Authentizit�t eines �ffentlichen Schl�ssels und seinen zul�ssigen Anwendungs- und Geltungsbereich best�tigen. X.509 ist derzeit der wichtigste Standard f�r digitale Zertifikate. Die aktuelle Version ist X.509v3.


\section{X.509}
Durch den Einsatz des Crypto.Certificate werden Zertifikate verifiziert und im Falle einer erfolgeichen Pr�fung k�nnen sie f�r ihre Zwecke weiter benutzt werden. Andernfalls gilt das Zertifikat als nicht vertrauensw�rdig und wird nicht weiter benutzt.\\ Unter solchen umst�nden sollten in Zukunft die Kinderzertifikate, also die Zertifikate die von einem hierarchisch h�heren Signiert wurden, auch nicht mehr vertraut werden.


\section{API}
\begin{tabular}{p{\textwidth}}
\begin{lstlisting}{}
procedure Get_Cert(Filename: String);
\end{lstlisting}\\
Die Prozedur dient zum einlesen eines Zertifikats. Akzeptiert werden *.txt und *.cer Files. Allerdings k�nnen nur die CER-Files auf Echtheit �berpr�ft werden.\\
Beim Importieren wird die Struktur und die G�ltigkeitsdauer des Zertifikats �berpr�ft. Ist das Zertifikat noch aktuell, wird die Signatur mit dem �ffentlichen Schl�ssel des Austellers entschl�sselt und mit dem Zertifikat-Hash verglichen. Bei einem Wurzelzertifikat befindet sich der Public Key in dem selbigen. So wird die Korrektheit der Werte sichergestellt.\\ \ \\
\hline
\end{tabular}
\begin{tabular}{p{\textwidth}}
\begin{lstlisting}{}
procedure CertOut;
\end{lstlisting}\\
Die Prozedur \textit{CertOut} gibt das Zertifikat in Textform aus.\\ \ \\
\hline
\end{tabular}
\begin{tabular}{p{\textwidth}}
\begin{lstlisting}{}
function AnalyseSigned return Unbounded_String;
\end{lstlisting}\\
Die Funktion \textit{AnalyseSigned} entschl�sselt die Signatur des in \textit{Get\_Cert} Importierten Zertifikats und gibt einen Unbounded\_String zur�ck. Man erh�lt bei erfolgreicher Dekodierung einen Hash-Wert (die L�nge variiert je nach verwendeten Hash-Typ MD5 oder Sha-1). Ansonsten erh�lt man einen Null\_Unbounded\_String.\\ \ \\
\end{tabular}


\section{Exceptions}
\begin{tabular}{p{\textwidth}}
\begin{lstlisting}{}
Cert_Typ_Error	:	exception;
\end{lstlisting}\\
Diese Ausnahme wird geworfen wenn es sich beim Importieren um ein falsches Format handelt.\\ \ \\
\hline
\end{tabular}

\begin{tabular}{p{\textwidth}}
\begin{lstlisting}{}
Cert_Structure_Error : exception;
\end{lstlisting}\\
Diese Ausnahme wird geworfen wenn die Struktur des Zertifikats nicht stimmt.\\ \ \\
\hline
\end{tabular}

\begin{tabular}{p{\textwidth}}
\begin{lstlisting}{}
ASN1_Structure_Error : exception;
\end{lstlisting}\\
Diese Ausnahme wird im Kinderpaket crypto.certificate.asn1 geworfen, wenn die Struktur der Bin�rdaten nicht nach dem asn1 Standart vorliegen.\\ \ \\
\hline
\end{tabular}

\begin{tabular}{p{\textwidth}}
\begin{lstlisting}{}
Conversion_Error  : exception;
\end{lstlisting}\\
Diese Ausnahme wird im Kinderpaket crypto.certificate.asn1 geworfen, wenn keine Bin�rdaten vorliegen.\\ \ \\
\end{tabular}


\section{Kinderpaket}
Das Paket Crypto.Certificate.ASN1 dient lediglich zur Konvertierung eines Zertifikats im Format CER.\\
Es verf�gt �ber folgende High-Level-API\\
\begin{tabular}{p{\textwidth}}
\begin{lstlisting}{}
procedure ASN1_Out;
\end{lstlisting}\\
Mit Hilfe dieser Prozedur wird das Zertifikat in Bin�r Darstellung ausgegeben. Es muss vorher ein Zertifikat mittels \textit{Get\_Cert(Filename: String)} eingelesen sein.\\ \ \\
\hline
\end{tabular}
\\ \ \\
Die Low-Level API\\
\begin{tabular}{p{\textwidth}}
\begin{lstlisting}{}
procedure Parse(ASN1Data : Unbounded_String);;
\end{lstlisting}\\
Diese Prozedur wird direkt beim Einlesen eines Zertifikats im CER-Format benutzt und dient zum Entschl�sseln der Werte eines nach dem Standard ASN1 (mit Distinct Encoding Rules (DER)) bin�r codierten Zertifikat.\\ \ \\
\hline
\end{tabular}


\section{Anwendungsbeispiel}
\begin{lstlisting}{}
with Ada.Text_IO; use Ada.Text_IO;
with Crypto.Certificate; 
use Crypto.Certificate;

procedure Example_Cert is
   
begin
   --Zertifikat importiern
   Get_Cert("1.cer");

   --Zertifikat ausgeben
   CertOut;

end Example_Cert;
\end{lstlisting}


% Literaturdatenbank
\bibliographystyle{plain} 
\bibliography{crypto}

\end{document}

    
    


