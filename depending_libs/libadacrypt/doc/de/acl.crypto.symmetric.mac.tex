\chapter{Crypto.Symmetric.MAC}
Ein Message Authentication Code (MAC) dient zur Sicherung der Integrit�t und
Authentizit�t einer Nachricht. Er gew�hrt keine Verbindlichkeit, da er nur 
einen symmetrischen Schl�ssel verwendet. Bei den asymmertrischen Verfahren 
spricht man von digitalen Signaturen. Bei MACs von \glqq authentication Tags
\grqq. Der Autor dieser Bibliothek bezeichnet diese meist als (digitiale)
Stempel.

%%%%%%%%%%%%%%%%%%%%%%%%%%%%%%%%%%%%%%%%%%%%%%%%%%%%%%%%%%%%%%%%%%%%%%%%%%%
%%%%%%%%%%%%%%%%%%%%%%%%%%%%%%%%%%%%%%%%%%%%%%%%%%%%%%%%%%%%%%%%%%%%%%%%%%%

\section{Randomized MAC (RMAC)}
Der RMAC wurde von Eliane Jaulmes, Antoine Joux and Frederic
Valette entwickelt. Er basiert auf dem CBC-MAC und basiert daher auf einer 
Einweg-Blockchiffre (Kapitel \ref{oneblock}) . Er ist beweisbar sicher 
gegen�ber \glqq Geburtstagsparadoxon-Angriffe \grqq. Er ben�tigt zwei 
Schl�ssel.
Der digitale Stempel ist doppelt so lange wie die verwendete Blockchiffre.\\
\ \\
Mathematische Beschreibung:
$$\mbox{RMAC}_{K_1,K_2}(E,M) = E_{K2 \oplus R}(C_n)
\mbox{ mit } C_i = E_{K_1}(M_i \oplus C_{i-1}), \quad R \in_R \{0,1\}^{|K_1|}
\mbox{ und } C_0=0$$

%%%%%%%%%%%%%%%%%%%%%%%%%%%%%%%%%%%%%%%%%%%%%%%%%%%%%%%%%%%%%%%%%%%%%%%%%%%
%%%%%%%%%%%%%%%%%%%%%%%%%%%%%%%%%%%%%%%%%%%%%%%%%%%%%%%%%%%%%%%%%%%%%%%%%%%

\subsection{Generischer Teil}
\begin{lstlisting}{}
generic
   with package C is 
       new Crypto.Symmetric.Oneway_Blockcipher(<>);

   with procedure Read (Random : out C.Key_Type) is <>;
   with function "xor" (Left, Right : C.Block)
                         return C.Block is <>;
   with function "xor" (Left, Right : C.Key_type) 
                         return C.Key_Type is <>;
\end{lstlisting}

%%%%%%%%%%%%%%%%%%%%%%%%%%%%%%%%%%%%%%%%%%%%%%%%%%%%%%%%%%%%%%%%%%%%%%%%%%%
%%%%%%%%%%%%%%%%%%%%%%%%%%%%%%%%%%%%%%%%%%%%%%%%%%%%%%%%%%%%%%%%%%%%%%%%%%%

\subsection{API}
\subsubsection{High-Level-API}
\begin{lstlisting}{}
  type Blocks  is array (Integer range <>) of Block;

  procedure Sign(Message : in Blocks;
                 Key1, Key2 : in Key_Type;
                 R   : out Key_Type;
                 Tag : out Block);


  function Verify(Message : in Blocks;
                  Key1, Key2 : in Key_Type;
                  R   : in Key_Type;
                  Tag : in Block) return Boolean;
\end{lstlisting}\ \\
Die Funktion \textbf{Sign} signiert eine Nachricht \textbf{Message} unter
den beiden Schl�ssel \textbf{Key1} und \textbf{Key2}. Sie liefert einen 
digitiale Stempel $S=Tag||R$. Wobei es sich bei \textbf{R} um eine zuf�llig
generierte Zahl handelt.\\ \ \\
Die Funktion \textbf{Verify} liefert \textit{true} zur�ck, wenn $S=Tag||R$
ein g�ltiger Stempel ist ansonsten  \textit{false} zur�ck. D.h. wenn einer oder
Mehrere Parameter nicht mit denen aus der Prozedur \textbf{Sign} �bereinstimmt,
dann liefert die Funktion mit �berw�ltigender Wahrscheinlichkeit
\textit{false} zur�ck.\\ \ \\

%%%%%%%%%%%%%%%%%%%%%%%%%%%%%%%%%%%%%%%%%%%%%%%%%%%%%%%%%%%%%%%%%%%%%%%%%%%

\subsubsection{Low-Level-API}
\begin{lstlisting}{}
   procedure Init(Key1, Key2 : in Key_Type);

   procedure Sign(Message_Block : in Block);
   procedure Final_Sign(Final_Message_Block : in Block;
                        R   : out Key_Type;
                        Tag : out Block);

   procedure Verify(Message_Block : in Block);
   function Final_Verify(Final_Message_Block  : in Block;
                         R   : in Key_Type;
                         Tag : in Block)
                        return Boolean;
\end{lstlisting}\ \\
\begin{itemize}
\item Die Prozedur \textbf{Init} initalisiert einen RMAC mit den beiden 
  Schl�sseln \textbf{Key1} und \textbf{Key2} und einem internen Anfangszustand.
\item Die Prozedur \textbf{Sign} \glqq hashed \grqq einen Nachrichtenblock 
  (\textbf{Message\_Block}).
\item Die Prozedur \textbf{Final\_Sign} hashed den finalen Nachrichtenblock 
  \textbf{Final\_Message\_Block}, generiert eine Zufallszahl \textbf{R}
  berechnet \textbf{Tag} und setzt den RMAC auf seinen Anfangszustand zur�ck.
\item  Die Prozedur \textbf{Verify} \glqq hashed \grqq einen Nachrichtenblock 
  (\textbf{Message\_Block}).
\item Die Funktion \textbf{Final\_Verify} hashed den finalen Nachrichtenblock 
  \textbf{Final\_Message\_Block} und verifiziert mit Hilfe eines internen 
  Zustandes und \textbf{R}, ob es sich bei \textbf{Tag} bzw. $\mathbf{R||Tag}$ 
  um einen g�ltigen Stempel handelt. Falls ja gibt sie \textit{true}, ansonsten
  \textit{false} zur�ck.
\end{itemize}

%%%%%%%%%%%%%%%%%%%%%%%%%%%%%%%%%%%%%%%%%%%%%%%%%%%%%%%%%%%%%%%%%%%%%%%%%%%
%%%%%%%%%%%%%%%%%%%%%%%%%%%%%%%%%%%%%%%%%%%%%%%%%%%%%%%%%%%%%%%%%%%%%%%%%%%

\subsection{Anwendungsbeispiel}
\begin{lstlisting}{}
  with Crypto.Types; use Crypto.Types;
  with Ada.Text_IO;  use Ada.Text_IO;
  with Crypto.Symmetric.MAC.RMAC;
  with Crypto.Symmetric.Oneway_Blockcipher_AES128;

pragma Elaborate_All (Crypto.Symmetric.MAC.RMAC);
 
procedure Example is
  package AES128 renames  Crypto.Symmetric.Oneway_Blockcipher_AES128;
  package RMAC is new Crypto.Symmetric.MAC.RMAC(AES128);
  use RMAC;

  Key1 : B_Block128 :=
        (
         16#00#, 16#01#, 16#02#, 16#03#, 16#04#, 16#05#,
         16#06#, 16#07#, 16#08#, 16#09#, 16#0a#, 16#0b#,
         16#0c#, 16#0d#, 16#0e#, 16#0f#
        );
  
  Key2 : B_Block128 :=
        (
         16#00#, 16#11#, 16#22#, 16#33#, 16#44#, 16#55#, 16#66#,
         16#77#, 16#88#, 16#99#, 16#aa#, 16#bb#, 16#cc#, 16#dd#,
         16#ee#, 16#ff#
        );

      R    : B_Block128;
      Tag  : B_Block128;

      M : B_Block256 := To_Bytes("ALL YOUR BASE ARE BELONG TO US! ");
      Message : RMAC.Blocks(0..1) := (0 =>  M(0..15), 1 => M(16..31));
begin
  Init(K1,K2);

  Sign(Message(0));
  Final_Sign(Message(1), R, Tag) ;

  Verify(Message(0));
  Put_Line(Final_Verify(Message(1), R, Tag));
 end Example;
\end{lstlisting}\ \\

%%%%%%%%%%%%%%%%%%%%%%%%%%%%%%%%%%%%%%%%%%%%%%%%%%%%%%%%%%%%%%%%%%%%%%%%%%%
%%%%%%%%%%%%%%%%%%%%%%%%%%%%%%%%%%%%%%%%%%%%%%%%%%%%%%%%%%%%%%%%%%%%%%%%%%%
%%%%%%%%%%%%%%%%%%%%%%%%%%%%%%%%%%%%%%%%%%%%%%%%%%%%%%%%%%%%%%%%%%%%%%%%%%%

\section{Hashfunction MAC (HMAC)}
Bei Mihir Bellare, Ran Canettiy und Hugo Krawczykz handelt es sich um die HMAC
Designer. Der HMAC basiert auf einer Hashfunktion.\\ \ \\
Mathematische Beschreibung: 
$$\mbox{HMAC}_K(H,M) =  H(K \oplus opad, H(K \oplus ipad || M)) \mbox{ mit } 
opad = \{0x5C\}^n \mbox{ und } ipad = \{0x36\}^n$$

%%%%%%%%%%%%%%%%%%%%%%%%%%%%%%%%%%%%%%%%%%%%%%%%%%%%%%%%%%%%%%%%%%%%%%%%%%%
%%%%%%%%%%%%%%%%%%%%%%%%%%%%%%%%%%%%%%%%%%%%%%%%%%%%%%%%%%%%%%%%%%%%%%%%%%%

\subsection{Generischer Teil}
\begin{lstlisting}{}
  with package H is new Crypto.Hashfunction(<>);

   with function "xor"
     (Left, Right : H.Message_Type)  return H.Message_Type is <>;
   with procedure Fill36 (Ipad : out  H.Message_Type) is <>;
   with procedure Fill5C (Opad : out  H.Message_Type) is <>;

   with procedure Copy
     (Source : in H.Hash_Type; Dest : out H.Message_Type) is <>;

    with function M_B_Length
       (Z : H.Hash_Type) return H.Message_Block_Length_Type is <>;
\end{lstlisting}\ \\
\begin{itemize}
\item Die beiden Prozeduren \textbf{Fill36} und \textbf{Fill5C} f�llen
  einen \textit{Message\_Type} \textbf{Ipad} bzw. \textbf{Opad}  mit dem Wert 
  \texttt{0x36} bzw. \texttt{0x5C} auf.
\item  Die Prozedur \textbf{Copy} kopiert den Inhalt von \textbf{Soucre} nach
\textbf{Destination}. Ist der  \textit{Message\_Type} \glqq kleiner \grqq
als der \textit{Hash\_Type} der Hashfunktion, dann wird der Rest mit Nullen
aufgef�llt. 
\item Die Funktion \textbf{M\_B\_Length} gibt die L�nge in Bytes von von 
  \textbf{Z} bzw. dem \textit{Hash\_Typ} zur�ck.
\end{itemize}
All diese Hilfsfmethoden sind in \textit{Crypto.Symmetric.MAC} definiert.\\ 
\ \\

%%%%%%%%%%%%%%%%%%%%%%%%%%%%%%%%%%%%%%%%%%%%%%%%%%%%%%%%%%%%%%%%%%%%%%%%%%%
%%%%%%%%%%%%%%%%%%%%%%%%%%%%%%%%%%%%%%%%%%%%%%%%%%%%%%%%%%%%%%%%%%%%%%%%%%%

\subsection{API}
\begin{lstlisting}{}
    procedure Init(Key : in Message_Type);
    
    procedure Sign(Message_Block : in Message_Type);

    procedure Final_Sign
      (Final_Message_Block        : in Message_Type;
       Final_Message_Block_Length : in Message_Block_Length_Type;
       Tag                        : out Hash_Type);
    
    
    procedure Verify(Message_Block : in Message_Type);

    function Final_Verify
      (Final_Message_Block        : Message_Type;
       Final_Message_Block_Length : Message_Block_Length_Type;
       Tag                        : Hash_Type) return Boolean;
\end{lstlisting}\ \\
\begin{itemize}
\item Die Prozedur \textbf{Init} initialisiert den HMAC mit dem Schl�ssel
  \textbf{Key} und einem internen Anfangszustand.
\item Die Prozedur \textbf{Sign} hashed einen Nachrichtenblock.
\item \item Die Prozedur \textbf{Final\_Sign} hashed einen Nachrichtenblock
  der Bytel�nge  \textbf{Final\_Message\_Block\_Length}, gibt den digitalen 
  Stempel \textbf{Tag} zur�ck und setzt den HMAC auf seine Anfangszustand 
  zur�ck.
\item Die Prozedur \textbf{Verify} hashed einen Nachrichtenblock.
\item Die Funktion  \textbf{Final\_Verify} hashed einen Nachrichtenblock 
  der Bytel�nge  \textbf{Final\_Message\_Block\_Length} und verifiziert mit 
  Hilfe eines internen Zustandes ob es sich bei \textbf{Tag} um einen g�ltigen 
  Stempel handelt. Falls ja gibt sie \textit{true}, ansonsten
  \textit{false} zur�ck.
\end{itemize}\ \\  

%%%%%%%%%%%%%%%%%%%%%%%%%%%%%%%%%%%%%%%%%%%%%%%%%%%%%%%%%%%%%%%%%%%%%%%%%%%
%%%%%%%%%%%%%%%%%%%%%%%%%%%%%%%%%%%%%%%%%%%%%%%%%%%%%%%%%%%%%%%%%%%%%%%%%%%

\subsection{Anwendungsbeispiel}
 \begin{lstlisting}{}
with Ada.Text_IO;
with Crypt.Types;
with Crypto.Symmetric.MAC;
with Crypto.Symmetric.MAC.HMAC;
with Crypto.Hashfunction_SHA256;

use  Ada.Text_IO;
use  Crypt.Types;

pragma Elaborate_All (Crypto.Symmetric.MAC.HMAC);


procedure Example is

  package RMAC is new Crypto.Symmetric.MAC.HMAC
           (H          => Crypto.Hashfunction_SHA256,
            Copy       => Crypto.Symmetric.Mac.Copy,
            Fill36     => Crypto.Symmetric.Mac.Fill36,
            Fill5C     => Crypto.Symmetric.Mac.Fill5C,
            M_B_Length => Crypto.Symmetric.Mac.M_B_Length);

  use RMAC;

  -- 160-Bit Key
  Key1 : W_Block512 := (0 => 16#0b_0b_0b_0b#, 1 => 16#0b_0b_0b_0b#,
                        2 => 16#0b_0b_0b_0b#, 3 => 16#0b_0b_0b_0b#,
                        4 => 16#0b_0b_0b_0b#, others  => 0);

   -- "Hi There" Len = 8 Byte
   Message : W_Block512 := (0 => 16#48_69_20_54#, 1 => 16#68_65_72_65#,
                            others => 0);

   Tag : W_Block256;
begin
  Init(Key1);
  Final_Sign(Message, 8, Tag);
  
  Put_Line(Final_Verify(Message1, 8, Tag));
end Example;
 \end{lstlisting}\ \\


%%%%%%%%%%%%%%%%%%%%%%%%%%%%%%%%%%%%%%%%%%%%%%%%%%%%%%%%%%%%%%%%%%%%%%%%%%%
%%%%%%%%%%%%%%%%%%%%%%%%%%%%%%%%%%%%%%%%%%%%%%%%%%%%%%%%%%%%%%%%%%%%%%%%%%%

\subsection{Anmerkungen}
Sie m�ssen nicht jedes mal von neuem aus einer Hashfunktion einen HMAC
generieren. Stattdessen k�nnen Sie auch eine der vorgefertigten HMACs 
verwenden.
\begin{itemize}
\item Crypto.Symmetric.MAC.HMAC\_SHA1
\item Crypto.Symmetric.MAC.HMAC\_SHA256
\item Crypto.Symmetric.MAC.HMAC\_SHA512
\item Crypto.Symmetric.MAC.HMAC\_Whirlpool
\end{itemize}
